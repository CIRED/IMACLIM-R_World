% !TEX TS-program = pdflatex
% !TEX encoding = UTF-8 Unicode

% This is a simple template for a LaTeX document using the "article" class.
% See "book", "report", "letter" for other types of document.

\documentclass[11pt,a4paper]{article} % use larger type; default would be 10pt
\usepackage[utf8]{inputenc} % set input encoding (not needed with XeLaTeX)

%%% Examples of Article customizations
% These packages are optional, depending whether you want the features they provide.
% See the LaTeX Companion or other references for full information.

%%% PAGE DIMENSIONS
%\usepackage{geometry} % to change the page dimensions
%\geometry{a4paper} % or letterpaper (US) or a5paper or....
% \geometry{margin=2in} % for example, change the margins to 2 inches all round
% \geometry{landscape} % set up the page for landscape
%   read geometry.pdf for detailed page layout information

\def\hybrid{\topmargin -10pt    \oddsidemargin 0pt
        \headheight 0pt \headsep 0pt
       \textwidth 6.5in        % US paper
        \textheight 9in         % US paper
       \textwidth 6.25in       % A4 paper
      \textheight 9.5in       % A4 paper
  \marginparwidth .875in
       % \parskip 5pt plus 1pt   \jot = 1.5ex
        }
     %The default is set to be hybrid
\hybrid

\usepackage{graphicx} % support the \includegraphics command and options
\usepackage[parfill]{parskip} % Activate to begin paragraphs with an empty line rather than an indent

%%% PACKAGES

\newcommand*\rot{\rotatebox[origin=rB]{90}}

\usepackage[table]{xcolor}
\newcommand\colorTable{\rowcolors{1}{white}{blue!20}}

\usepackage{siunitx}


\usepackage{booktabs} % for much better looking tables
\usepackage{array} % for better arrays (eg matrices) in maths
\usepackage{paralist} % very flexible & customisable lists (eg. enumerate/itemize, etc.)
\usepackage{verbatim} % adds environment for commenting out blocks of text & for better verbatim
\usepackage{subfig} % make it possible to include more than one captioned figure/table in a single float
% These packages are all incorporated in the memoir class to one degree or another...
\usepackage{longtable}
\usepackage{pdflscape}
\usepackage{float}
\usepackage{hyperref}
\usepackage{cleveref}
\usepackage{natbib}
\usepackage{changepage}
\usepackage{rotating}
\usepackage{amsmath, amsthm} 
\usepackage{amsfonts}
\usepackage{amssymb,graphics,psfrag}
%\usepackage[bottom]{footmisc}

\usepackage{mathrsfs}
%\usepackage{cite}
%\usepackage{setspace}
\usepackage{color}
%\usepackage{mathtools}


\textwidth=15cm
\usepackage[textsize=small,textwidth=2.2cm,shadow]{todonotes}
\newcommand\ruben[2][]{\todo[color=green!50,#1]{rb: #2}}
\newcommand\ri[1]{\ruben[inline]{#1}}
\newcommand\rb[1]{\textcolor{green!80!black}{#1}}
\newcommand\ru[1]{\textcolor{green!80!black}{\sout{#1}}}

%\usepackage[marginpar]{todo}


%%% HEADERS & FOOTERS
%\usepackage{fancyhdr} % This should be set AFTER setting up the page geometry
%\pagestyle{fancy} % options: empty , plain , fancy
%\renewcommand{\headrulewidth}{0pt} % customise the layout...
%\lhead{}\chead{}\rhead{}
%\lfoot{}\cfoot{\thepage}\rfoot{}

%%% SECTION TITLE APPEARANCE
\usepackage{sectsty}
\allsectionsfont{\sffamily\mdseries\upshape} % (See the fntguide.pdf for font help)
% (This matches ConTeXt defaults)

%%% ToC (table of contents) APPEARANCE
\usepackage[nottoc,notlof,notlot]{tocbibind} % Put the bibliography in the ToC
\usepackage[titles,subfigure]{tocloft} % Alter the style of the Table of Contents
\renewcommand{\cftsecfont}{\rmfamily\mdseries\upshape}
\renewcommand{\cftsecpagefont}{\rmfamily\mdseries\upshape} % No bold!

%%% END Article customizations

\floatstyle{plaintop}
\restylefloat{table}

\title{Using the KLEMS capital input database to detail capital investment per sector in IMACLIM ; Methodology}
%\author{The Author}
\date{} % Activate to display a given date or no date (if empty),
         % otherwise the current date is printed
\setlength{\tabcolsep}{20pt}
\renewcommand{\arraystretch}{1.3}
\setlength{\parskip}{1em}
\begin{document}
\maketitle
\setcounter{tocdepth}{2}
\tableofcontents

\section{Goals of the study}
\begin{enumerate} 
    \item Format the data so that it can be used
    \item Produce a GFCF matrix in the imaclim format and check whether the assumption of colinearity is valid (hypothesis is that it is not)
    \item Compare to AMECO or other databases
    \item Identify general trends by countries in temporal evolution
    \item Identify group of countries with similar GFCF structure and / or evolution
    \item Provide the GFCF matrices for Imaclim, for France \& the 12 regions and design scenarios for their evolution
    \item Use these matrices to:
        \begin{itemize}
            \item Evaluate compared to standard assumption
            \item Understand the economic impacts
            \item Understand the emission impacts
        \end{itemize}
    \item Link with growth
\end{enumerate}



\section{Description of the KLEMS data}
\subsection{Capital input variables}

The period covered is 1970 - 2007, for the variables summarized in \cref{tbl:kDataSummary}.
Data at a 32 industry-level (table \ref{IndustriesKLEMS}).
The 32 industries invest in 8 types of products (table \ref{CapAssets}).

\subsubsection{Gross fixed capital formation, I} 
GFCF\footnote{
    A more complete definition can be found at \url{http://stats.oecd.org/glossary/detail.asp?ID=1173}.
} is acquisition of fixed assets less disposal of fixed assets, where disposal refers to the sale of assets.
Disposal does not include capital consumption nor losses due to natural disasters.
\subsubsection{Price index, Ip} \label{hedonic}
The price index used in the KLEMS database is the hedonic price index \cite{timmer_towards_2000} \cite{timmer_eu_2007-1}. It differs from a 'conventional' price index in quality adjustement method. Conventional price indexes are based on the matching method which records price changes holding constant the characteristics of the transaction. The quality adjustment problem arrises when matching is not possible because of products leaving or entering the market, which introduces a sampling problem. 

A hedonic price index is based on a hedonic function, wich specifies a relationship between prices and the characteristics of the product. For ICT equipement, specifically computers, the hedonic price index decreases much faster than a conventional price index. Thus, current computers have a much higher value when expressed in an older base year's currency. A detailed account of possible critques of the hedonic price index is given in
\cite{triplett_handbook_2004}, critiques which center around - in the case of computers - what exactly is captured by the very fast decreasing of the index. Leaving this issue at the side, when using the price index or real economic variables corrected by it, one has to keep in mind the consistent and fast quality increase which has been going on in computer technology: investing in 1 computer today is assumed to being equivalent to investing in many more computers from years ago. When doing productivity measurements, it is obviously necessary to take into account quality improvements in quality. When confronting the question of a particular industry's investment pattern, one would want to look at for example what the portion of a budget of that industry is, at that moment, that is allocated to a particular asset. Thus one wants to work with nominal prices and the ratio of those nominal prices to the total amount invested. From this point of view, the time series for real investment, which show an enormous increase in ICT assets, mean that every year more and more computer power is installed. It does not mean that every year, over the years, a bigger portion of capital (in the nominal money sense), goes to computer assets.%, or that since this new asset is amongst us, higher total investment is needed.
%On the other hand, if newer more performant computers require more material and energy input to produce, then there is a possible correlation between the increase in real investment and the increase in capital input where the latter is taken in the sense of materials and energy input, correlation captured via the hedonic price index.
The price index going down fast for IT or computers represents efficiency gains, which in the case of computers are not (only) translated in lowering prices in the sense of what portion of income (national, household) goes to a particular product, but is translated in quality gains, which is apparently what demand dictates.

Countries using a hedonic deflator are USA, Japan, France and Denmark (need to check)\cite{timmer_eu_2007-1}. For many countries, hedonic price indices are not available. In that case, the US hedoncic price index has been used, using a harmonization method based on assuming the difference between price changes in ICT and non-ICT capital the same across countries \cite{schreyer_measuring_2001}. 
\subsubsection{Fixed capital stock, K}
Fixed capital stock is determined by the aggregation of the value of fixed assets, taking into account their age-price profile, see below in the section on Consumption of fixed capital. To aggregate, acquisition prices need to be corrected by a price index. 

\subsubsection{Capital compensation, CAP} \ruben{what does it represent? how is it computed?}
Capital compensation is calculated as value added minus labour compensation. Labour compensation is calculated by supposing all workers are  compensated at the same rate as employees.
\subsubsection{Consumption of fixed capital, D}
Consumption of fixed capital (CFC) is the difference between successive real market values of an asset. It represents the decline in the future benefits of the assets due to their use in the production process. The decline in the value of fixed assets of governments and firms, and of dwellings of households. Unforseen obsolescene (for example due to accidents and natural disasters) is not included. 

The value of an asset is determined by the rentals it will earn, the discount rate and its scrap value. Scrap value is usually small compared to the vlaue of the asset determined by the former two variables. The value of an asset is the actualised sum of the rentals it will generate per year during its service life. The rentals are equal to the quantity of services the asset will generate times the unit price. The decline in future benefits then results form the fact that the asset has a finite
service life and that it will generate a smaller and smaller quantity of services. This relationship between rentals and the value of an assets is used to determine the pattern of the values of an asset over its service life, which is called the age-price profile of an asset.  \cite{oecd_measuring_2001}.

\emph{Value} in the previous paragraph is to be understood as market price, since through rational investor and producer behaviour, market mechanisms should equalize the market price of an asset and its value as determined herabove.

The age-price profile can be used to calculate the difference between beginning and end of year asset-values. In practice, many countries do not use age-price profiles to determin CFC, but apply depreciation functions to the gross value of assets. Using a specific depreciation function implies adopting a specific age-price profile.

\subsubsection{Depreciation rates, Deprate}
The depreciation rates used are specific for asset type and industry, and do not vary per country and over time. The rates used are the ones from the US Bureau of Economic Analysis (BEA).

\subsubsection{Industry rate of return on capital, IRR}

\begin{table}
    \centering
    \begin{tabular}{lp{6cm}p{5cm}}
    Variables &Meaning&Detail\\
    \multicolumn{3}{c}{\textbf{Variables \textit{var}}}\\
    I  & Nominal gross fixed capital formation&per product\\
    Iq & Real gross fixed capital formation, 1995 prices&per product \\
    Ip & Gross fixed capital formation price index (1995=1.00)&per product   \\
    K  & Real fixed capital stock, 1995 prices &per product  \\
    Cap& Capital compensation&per product    \\
    D  & Fixed capital consumption, 1995 prices &per product    \\
    IRR&Industry rate of return on capital&\\
    Deprate&Geometric depreciation rates&\\
    \midrule
    \multicolumn{3}{c}{\textbf{Products (goods bought for investment) \textit{product}}}\\
    IT       & Computing equipment\\
    CT       & Communications equipment\\
    Soft     & Software\\
    TraEq    & Transport Equipment\\
    OMach    & Other Machinery and Equipment\\
    OCon     & Total Non-residential investment\\
    RStruc   & Residential structures\\
    Other    & Other assets   \\
    ICT      & ICT assets     & ICT = IT + CT + Soft\\
    NonICT   & Non-ICT assets & NonICT = TraEq + OMach + OCon + RStruc + Other\\
    GFCF     & All assets     & GFCF = ICT + NonICT\\
    \midrule
\end{tabular}

    \caption{\label{tbl:kDataSummary}Structure of the capital database}
\end{table}

\subsection{Labour and other variables}
Generally data at a 72 industry-level, except USA, Japan, Korea, Malta, Cyprus and Australia. Variables:

\begin{itemize}
    \item Gross output (current basic prices, volume indices)
    \item Intermediate inputs (current purchasers' prices, volume) + energy, material, service. The prices used are purchasers' prices and include margins that should be reallocated to trade and transportation. \ruben{what disaggregation?}
        \begin{itemize}
            \item E = 10-12 (energy mining products) + 23 (oil refining products), 40, electricity and gas products, where the codes used refer to the NACE1 classification of industries, see table \ref{IndustriesKLEMS}.
            \item S = 50-99
            \item M = remaining = AtB + 13t22 + 24t37 + 41 + F
        \end{itemize}
    \item Gross value added (current basic prices, volume, per hour worked)
    \item Gross operating surplus
    \item Taxes minus subsidies on production
    \item Capital compensation (Total, ICT versus non-ICT)
    \item Capital services (Total, ICT versus non-ICT)
\end{itemize}

and for the workforce:
\begin{itemize}
    \item Compensation of employees
    \item Number of persons engaged
    \item Number of employees
    \item Total hours worked by persons engaged (+ differentiation per age, skill, sex)
    \item Total hours worked by employees
    \item Labour compensation (skill differentiation)
    \item Labour services
\end{itemize}


\subsubsection{Data availability per country}

\paragraph{Argentina}

\paragraph{Canada}

Capital data available for 4 asset-levels, $OCon$ (Total Non-residential investment), $I\_RStruc$ (Residential structures), $I\_ICT$ (ICT assets), $I\_NonICT$ (Non-ICT assets), 32 industry-levels, from 1961-2010.
Variables : I, Iq, Ip, K, CAP, IRR.


\paragraph{\href{http://www.rieti.go.jp/en/database/CIP2015/index.html}{China}}


Available data :
Files $CIP\_3.0\_(2015)\_1.01.xlsx - 2.13$ :
\begin{itemize}
\item 1.1a: Value of output by industry in ml current yuan, 1.1b: Value of intermediate input by industry in ml current yuan, 1981- 2010
\item 1.2a: Value of output by industry in ml constant yuan (based on previous year prices), 1.2b: Value of intermediate input by industry in ml constant yuan (based on previous year prices), 1981- 2010
\item 1.3: 1981- 2010 input-output table in ml current yuan
\item 1.4: 1981-2010 input-output table in ml constant yuan (based on previous year prices)
\item 1.5a: Distribution of gross value added by industry: Consumption of fixed capital (ml current yuan), Compensation of employees (ml current yuan), Operating surplus (ml current yuan), Net production tax (ml current yuan)

\item 1.6: Final demand by industry: Consumption (ml current yuan), \emph{Capital formation (ml current yuan)}, Export (ml current yuan), Import (ml current yuan), 37 industry-levels,1981-2010.

\item 1.7: Final demand by industry: Consumption (ml constant yuan, based on previous years prices), Capital formation (ml current yuan), Export (ml current yuan), Import (ml current yuan), 37 industry-levels,1981-2010.
\item 2.1: Investment Total, in "equipment" \& "non-residential structures" by industrial enterprises at or above "designated size" in ml current yuan, Total for 13 industry-levels,Equip and nonres struc at 24 industry-level.
\item 2.2: Investment Total,  in "equipment" \& "non-residential structures"by industrial enterprises at or above "designated size" in ml 1990 yuan, Total for 13 industry-levels,Equip and nonres struc at 24 industry-level.

\item Idem above capital stock
\item
\item

\end{itemize}


\paragraph{India}
Source : Asia KLEMS, \href{http://asiaklems.net/data/archive.asp#} No capital data.

\paragraph{Korea}

On Asia KLEMS no capital data, but capital data on the world KLEMS site. The available variables are:
\begin{itemize}
\item Iq, real investment (in millions of Korean Won) for 11 asset-levels and 72 industry-levels (local currency, millions Won), 1970 - 2012,
\item real net capital stock (in millions of Korean Won) for 11 asset levels and 72 industry-levels (local currency, millions Won),
\item Ip, investment deflator for 11 asset-levels,
\item I, nominal investment (in millions of Korean Won) for 72 industry-levels,
\item nominal net capital stock (in millions of Korean Won) for 72 industry-levels.
\end{itemize}


\paragraph{Russia}

No capital data available. 

\paragraph{Taiwan}




\subsection{Regions covered}

\renewcommand{\arraystretch}{1.2}
\setlength{\tabcolsep}{8pt}
\tiny
\begin{longtable}{| l | l | l | l |}
%\begin{tabular}{| l | p{360pt} | p{80pt} |}
\hline	
IMACLIM regions	&	Country	&	Basic (labour etc.)	&	Capital input	\\
\hline
Enlarged Europe	&	Austria	&	X	&	X	\\
	&	Belgium	&	X	&	confidential	\\
	&	Denmark	&	X	&	X	\\
	&	Finland	&	X	&	X	\\
	&	France	&	X	&	confidential	\\
	&	Germany	&	X	&	X	\\
	&	United Kingdom	&	X	&	X	\\
	&	Greece	&	X	&		\\
	&	Ireland	&	X	&	confidential	\\
	&	Italy	&	X	&	X	\\
	&	Luxembourg	&	X	&		\\
	&	Netherlands	&	X	&	X	\\
	&	Portugal	&	X	&		\\
	&	Spain	&	X	&	X	\\
	&	Sweden	&	X	&	X	\\
	&	Switzerland	&		&		\\
	&	Rest of EFTA	&		&		\\
	&	Rest of Europe	&		&		\\
	&	Albania	&		&		\\
	&	Bulgaria	&		&		\\
	&	Croatia	&		&		\\
	&	Cyprus	&	X	&		\\
	&	Czech Republic	&	X	&	X	\\
	&	Hungary	&	X	&	confidential	\\
	&	Malta	&	X	&		\\
	&	Poland	&	X	&		\\
	&	Romania	&		&		\\
	&	Slovakia	&	X	&		\\
	&	Slovenia	&	X	&	X	\\
	&	Estonia	&	X	&		\\
	&	Latvia	&	X	&		\\
	&	Lithuania	&	X	&		\\
\hline
Canada	&		&	X	&	X	\\
\hline
USA	&		&	X	&	X	\\
\hline
OCDE Pacifique	&	Australia	&	X	&	X	\\
	&	New-Zealand	&		&		\\
	&	Japan	&	X	&	X	\\
	&	Korea	&	X	&	X	\\
\hline
China	&	China	&		&	X (1980 - 2010)	\\
\hline
India	&	India	&	X	&	X (Cap stock \& services 1980 - 2008)	\\
\hline
Brazil	&	Brazil	&		&		\\
\hline
Africa	&		&		&		\\
\hline
Rest of Asia	&	Asia KLEMS under construction	&		&		\\
\hline
Rest of Latin America	&	Latin America KLEMS under construction.	&		&		\\
\hline
	&	Argentina	&	X	&	X Cap stock + ?(1990 - 2010)	\\
\hline
CEI	&	Russian Federation	&	X	&	X Cap compensation \& services 1995 - 2009	\\
	&	Rest of Former Soviet Union.	&		&		\\
\hline
%\end{tabular}
\caption{Regions covered by the KLEMS data versus IMACLIM regions.}
\label{RegionsKLEMS}
\end{longtable}

\normalsize

\section{KLEMS Capital input data}
The capital input data provided by the KLEMS database are structured in the form of a use table. More generally, all the variables in the database are provided according to the detailed framework of supply- and use tables (SUTs) \cite{timmer_eu_2007}. In 1995, SUTs where added to the European System of Accounts (ESA), and constitute, together with input-output tabels, ESA95 \cite{eurostat_eurostat_2008}. Unlike input-output tables, SUTs are not symmetric with respect to the column and row entries: the supply and use tables list products in the rows and industries in the columns. The supply table shows for each product by which industries it is produced and how much is imported. The use table shows per product type in what amounts the different industries consume the product as a production factor, as well as the final uses, i.e. final consumption, investment and exports.

The level of detail provided by the KLEMS database in distinguishing between product types and industrial sectors, is guided both by the imperative of consistency accross countries and by the level of detail needed for the database to be an effective tool to analyse productivity in the European Union at the industry level.

The classification used for industries is the \href{http://ec.europa.eu/eurostat/ramon/nomenclatures/index.cfm?TargetUrl=LST_NOM_DTL&StrNom=NACE_1_1}{‘General Industrial Classification of Economic Activities within the European Communities’ (NACE) revision 1.1}\footnote{Link consulted July 2016.} and the classification employed for products is the ‘Classification of Products by Activity’ (CPA).

\subsection{Industries}
The KLEMS database provides at its most detailed level data for 72 industries. These consist of the 60 industries used by almost all EU countries% in the constitution of national SUTs since 1995
, to which further detail was added when particular industries stood out with regard to investment in skills, R\&D, information and communication technologies (ICT) or ICT share in output. On this basis, the following 8 industries were added because beleived to carry information essential for productivity analysis: pharmaceuticals, insulated wire, electronic valves, telecommunication equipment, scientific instruments, manufacturing of ships, manufacturing of aircraft and
legal/technical/advertising services. Because in practice data availability varies, for growth accounting variables including capital formation variables, a minimum list of 31 sectors was used \cite{timmer_eu_2007}.Table \ref{IndustriesKLEMS} lists the 71 industries for which the KLEMS project collected information and shows at what level of detail capital input data are available.

For some countries, notably for the US, data is available at a more detailed industry level (is there?).

\tiny

\renewcommand{\arraystretch}{1.2}
\setlength{\tabcolsep}{8pt}
\begin{longtable}{|l|l|p{1cm}|}
%\begin{tabular}{| l | p{360pt} | p{60pt} |}
\hline	
 NACE\footnote{Nomenclature statistique des activités économiques dans la Communauté européenne} code	&	Industries	&	Growth accounting variables availability	\\
TOT	&	TOTAL ECONOMY	&	X	\\
AtB	&	AGRICULTURE, HUNTING, FORESTRY AND FISHING	&	X	\\
A	&	…AGRICULTURE, HUNTING AND FORESTRY	&		\\
1	&	……Agriculture	&		\\
2	&	……Forestry	&		\\
B	&	…FISHING	&		\\
C	&	MINING AND QUARRYING	&	X	\\
10t12	&	…MINING AND QUARRYING OF ENERGY PRODUCING MATERIALS	&		\\
10	&	……Mining of coal and lignite; extraction of peat	&		\\
11	&	……Extraction of crude petroleum and natural gas and services	&		\\
12	&	……Mining of uranium and thorium ores	&		\\
13t14	&	…MINING AND QUARRYING EXCEPT ENERGY PRODUCING MATERIALS	&		\\
13	&	……Mining of metal ores	&		\\
14	&	……Other mining and quarrying	&		\\
D	&	TOTAL MANUFACTURING	&	X	\\
15t16	&	…FOOD PRODUCTS, BEVERAGES AND TOBACCO	&	X	\\
15	&	……Food products and beverages	&		\\
16	&	……Tobacco products	&		\\
17t19	&	…TEXTILES, TEXTILE PRODUCTS, LEATHER AND FOOTWEAR	&	X	\\
17t18	&	……Textiles and textile products	&		\\
17	&	………Textiles	&		\\
18	&	………Wearing Apparel, Dressing And Dying Of Fur	&		\\
19	&	……Leather, leather products and footwear	&		\\
20	&	…WOOD AND PRODUCTS OF WOOD AND CORK	&	X	\\
21t22	&	…PULP, PAPER, PAPER PRODUCTS, PRINTING AND PUBLISHING	&	X	\\
21	&	……Pulp, paper and paper products	&		\\
22	&	……Printing, publishing and reproduction	&		\\
221	&	………Publishing	&		\\
22x	&	………Printing and reproduction	&		\\
23t25	&	…CHEMICAL, RUBBER, PLASTICS AND FUEL PRODUCTS	&	X	\\
23	&	……Coke, refined petroleum products and nuclear fuel	&	X	\\
24	&	……Chemicals and chemical products	&	X	\\
244	&	………Pharmaceuticals	&		\\
24x	&	………Chemicals excluding pharmaceuticals	&		\\
25	&	……Rubber and plastics products	&	X	\\
26	&	…OTHER NON-METALLIC MINERAL PRODUCTS	&	X	\\
27t28	&	…BASIC METALS AND FABRICATED METAL PRODUCTS	&	X	\\
27	&	……Basic metals	&		\\
28	&	……Fabricated metal products	&		\\
29	&	…MACHINERY, NEC	&	X	\\
30t33	&	…ELECTRICAL AND OPTICAL EQUIPMENT	&	X	\\
30	&	……Office, accounting and computing machinery	&		\\
31t32	&	……Electrical engineering	&		\\
31	&	………Electrical machinery and apparatus, nec	&		\\
313	&	…………Insulated wire	&		\\
31x	&	…………Other electrical machinery and apparatus nec	&		\\
32	&	………Radio, television and communication equipment	&		\\
321	&	…………Electronic valves and tubes	&		\\
322	&	…………Telecommunication equipment	&		\\
323	&	…………Radio and television receivers	&		\\
33	&	……Medical, precision and optical instruments	&		\\
331t3	&	………Scientific instruments	&		\\
334t5	&	………Other instruments	&		\\
34t35	&	…TRANSPORT EQUIPMENT	&	X	\\
34	&	……Motor vehicles, trailers and semi-trailers	&		\\
35	&	……Other transport equipment	&		\\
351	&	………Building and repairing of ships and boats	&		\\
353	&	………Aircraft and spacecraft	&		\\
35x	&	………Railroad equipment and transport equipment nec	&		\\
36t37	&	…MANUFACTURING NEC; RECYCLING	&	X	\\
36	&	……Manufacturing nec	&		\\
37	&	……Recycling	&		\\
E	&	ELECTRICITY, GAS AND WATER SUPPLY	&	X	\\
40	&	…ELECTRICITY AND GAS	&		\\
40x	&	……Electricity supply	&		\\
402	&	……Gas supply	&		\\
41	&	…WATER SUPPLY	&		\\
F	&	CONSTRUCTION	&	X	\\
G	&	WHOLESALE AND RETAIL TRADE	&	X	\\
50	&	……Sale, maintenance and repair of motor vehicles and motorcycles; retail sale of	&	X	\\
51	&	……Wholesale trade and commission trade, except of motor vehicles and motorcycles	&	X	\\
52	&	……Retail trade, except of motor vehicles and motorcycles; repair of household goods	&	X	\\
H	&	HOTELS AND RESTAURANTS	&	X	\\
I	&	TRANSPORT AND STORAGE AND COMMUNICATION	&	X	\\
60t63	&	…TRANSPORT AND STORAGE	&	X	\\
60	&	……Inland transport	&		\\
61	&	……Water transport	&		\\
62	&	……Air transport	&		\\
63	&	……Supporting and auxiliary transport activities; activities of travel agencies	&		\\
64	&	…POST AND TELECOMMUNICATIONS	&	X	\\
JtK	&	FINANCE, INSURANCE, REAL ESTATE AND BUSINESS SERVICES	&	X	\\
J	&	…FINANCIAL INTERMEDIATION	&	X	\\
65	&	……Financial intermediation, except insurance and pension funding	&		\\
66	&	……Insurance and pension funding, except compulsory social security	&		\\
67	&	……Activities related to financial intermediation	&		\\
K	&	…REAL ESTATE, RENTING AND BUSINESS ACTIVITIES	&	X	\\
70	&	……Real estate activities	&	X	\\
71t74	&	……Renting of m\&eq and other business activities	&	X	\\
71	&	………Renting of machinery and equipment	&		\\
72	&	………Computer and related activities	&		\\
73	&	………Research and development	&		\\
74	&	………Other business activities	&		\\
741t4	&	…………Legal, technical and advertising	&		\\
745t8	&	…………Other business activities, nec	&		\\
LtQ	&	COMMUNITY SOCIAL AND PERSONAL SERVICES	&	X	\\
L	&	…PUBLIC ADMIN AND DEFENCE; COMPULSORY SOCIAL SECURITY	&	X	\\
M	&	…EDUCATION	&	X	\\
N	&	…HEALTH AND SOCIAL WORK	&	X	\\
O	&	…OTHER COMMUNITY, SOCIAL AND PERSONAL SERVICES	&	X	\\
90	&	……Sewage and refuse disposal, sanitation and similar activities	&		\\
91	&	……Activities of membership organizations nec	&		\\
92	&	……Recreational, cultural and sporting activities	&		\\
921t2	&	………Media activities	&		\\
923t7	&	………Other recreational activities	&		\\
93	&	……Other service activities	&		\\
P	&	…PRIVATE HOUSEHOLDS WITH EMPLOYED PERSONS	&	X	\\
Q	&	…EXTRA-TERRITORIAL ORGANIZATIONS AND BODIES	&		\\
\hline
%\end{tabular}
\caption{Industry classification in KLEMS.}
\label{IndustriesKLEMS}
\end{longtable}
%\clearpage
\normalsize

\subsection{Product types}\label{ProductTypes}
At the lowest level of aggregation, the product types used in the KLEMS database are: residential structures, non-residential structures, infrastructure, transport equipment, computing equipment, communications equipment, software, other machinery and equipment, products of agriculture and forestry, other products, software, other intangibles. This minimum list with 11 asset types was defined because most European countries provide a limited amount of asset detail in the capital formation matrices. The capital investment data are available for a slightly more aggregated version of 8 asset types, listed in table \ref{CapAssets}.

\begin{table}
\setlength{\tabcolsep}{6pt}
\begin{tabular}{| l | l | p{160pt} |}
\hline			
IMACLIM	&	Asset types	&	  \\	
\hline	
Industries	&	Computing equipment	&	IT	\\
Industries	&	Communications equipment	&	CT	\\
Autres \& services	&	Software	&	Soft	\\
Industries	&	Transport Equipment	&	TraEq	\\
Industries	&	Other Machinery and Equipment	&	OMach	\\
Construction	&	Total Non-residential investment	&	OCon = NRStruc + Infra	\\
Construction	&	Residential structures	&	RStruc	\\
Autres \& services + Agri	&	Other assets	&	Other = Agri + Oth + OGFCFI	\\
\hline
& ICT assets & ICT = IT + CT + Soft \\
& non-ICT assets & Non-ICT = TraEq + Omach + OCon + Rstruc + Other \\
& All assets & ICT + Non-ICT \\
\hline
\end{tabular}
\caption{Aggregation level of asset types for which capital input data are available. NRStruc = non-residential structures, Infra = infrastructure, Agri = agriculture, Other = other products, OGFCFI = Other intangibles.}
\label{CapAssets}
\end{table}

A more detailed description of the contents of each of the asset-levels is obtained by referring to the 1993 System of National Accounts\footnote{
    Cf. Chapter XIII. The balance sheet, Annex: Definitions of assets.
} \cite{commission1993system}, and the OECD manual ``Measuring Capital''\cite{oecd_measuring_2001}\footnote{
    Cf. Chapter 3, ``Coverage and classification of stocks and flows, Classification by asset type, 3.8''.
}.
They are listed in \cref{table:CapAssetsDetailed}.


\begin{table}
    \setlength{\tabcolsep}{6pt}
    \footnotesize
    \vskip-0.0cm\hskip-2.0cm\begin{tabular}{| l | l | l | p{110pt} | p{260pt} |}
        \hline	
        \rot{Tangibles}	&	RStruc	&		&	Dwellings	&	Buildings that are used entirely or primarily as residences, including any associated structures, such as garages, and all permanent fixtures customarily installed in residences. Houseboats, barges, mobile homes and caravans used as principal residences of households are also included, as are historic monuments identified primarily as dwellings. Costs of site clearance and preparation are also included.	\\
        \cline{2-5}
        &	OCon	&	= NRStruc	&	Non-residential structures	&	Buildings other than dwellings, including fixtures, facilities and equipment that are integral parts of the structures and costs of site clearance and preparation. Historic monuments identified primarily as non-residential buildings are also included. Examples include products included in CPC class 5212, nonresidential buildings, such as warehouse and industrial buildings, commercial buildings, buildings for public entertainment, hotels, restaurants, educational buildings, health buildings, etc.	\\
        \cline{3-5}
        &		&+	Infra	&	Other structures \newline (=infrastructure)	&	Structures other than buildings, including the cost of the streets,
        sewers and site clearance and preparation other than for residential
        or non-residential buildings. Also included are historic
        monuments for which identification as dwellings or nonresidential
        buildings is not possible and shafts, tunnels and other
        structures associated with mining subsoil assets. (Major
        improvements to land, such as dams and dykes for flood control,
        are included in the value of land.)
        Examples include products included in CPC group 522, civil
        engineering works, such as highways, streets, roads, railways and
        airfield runways; bridges, elevated highways, tunnels and
        subways; waterways, harbours, dams and other waterworks; longdistance
        pipelines, communication and power lines; local pipelines
        and cables, ancillary works; constructions for mining and
        manufacture; and constructions for sport and recreation.	\\
        \cline{2-5}
        &	TraEq	&		&	Transport equipment	&	Equipment for moving people and objects: motor vehicles, trailers and semitrailers; ships;
        railway and tramway locomotives and rolling stock; aircraft and
        spacecraft; and motorcycles, bicycles, etc.	\\
        \cline{2-5}
        &	OMach	&		&	Machinery and equipment  \newline excluding communications  \newline equipment and
        computers \newline (office machinery?) 	&	Machinery and equipment not elsewhere classified. Examples
        include products other than parts included in CPC divisions 43,
        general purpose machinery; 44, special purpose machinery; 45,
        office, accounting and (computing equipment), 46, electrical
        machinery and apparatus, 47, radio, television and (communication
        equipment and apparatus); and 48, medical appliances, precision
        and optical instruments, watches and clocks.\\
        \cline{2-5}
        &	IT	&		&		&		\\
        \cline{2-5}
        &	CT	&		&		&		\\
        \cline{2-5}
        &	Other	&	=Agri	&	Cultivated assets	&		\\
        \cline{3-5}
        &		&	+Other	&		&	Other examples are
        products other than parts included in CPC groups 337, fuel
        elements (cartridges) for nuclear reactors; 381, furniture; 383,
        musical instruments; 384, sports goods; and 423, steam generators
        except central heating boilers.	\\
        \cline{1-1} \cline{3-5}
        \rot{Intangibles}	&		&	+OGFCFI	&	Mineral exploration, \newline Entertainment, literary and \newline artistic originals	&		\\
        \cline{2-5}
        &	Soft	&		&		&	Computer programs, program descriptions and supporting materials for both systems and applications software. Included are purchased software and software developed on own account, if the expenditure is large. Large expenditures on the purchase, development or extension of computer databases that are expected to be used for more than one year, whether marketed or not, are also included.	\\
        \hline
    \end{tabular}
    \caption{Detailed description of the KLEMS asset-levels, SNA 1993 \cite{commission1993system}.}
    \label{table:CapAssetsDetailed}
\end{table}




\subsection{From the KLEMS use table to the IMACLIM IO table}

For the purpose of using the capital formation data provided by the detailed use table as input for the IMACLIM model, the use table needs to be transformed to the input-output form with 12 specified sectors. To this end, the 11 product types specified in KLEMS to which investment is directed need to be mapped to the corresponding producing industries, and secondly, the 31 industries for which the KLEMS database specifies capital investment need to be mapped to the GTAP sectors. 

\subsubsection{Mapping industries}

Mapping industries is a straightforward procedure since information is available on the correspondance between the GTAP sectorial divisions and the International Standard Industrial Classification (ISIC) classification. NACE is the European adaptation of the ISIC. The correspondance is given in table \ref{IndKLEMSGTAP}.

\renewcommand{\arraystretch}{1.15}
\begin{table}[!p]
    \tiny
\vskip-0.0cm\hskip-2.0cm\begin{tabular}{| p{110 pt} | l | p{180 pt}| p{92 pt}|}
\hline
	&	code	&	EU KLEMS	&	GTAP	\\
\hline							
AtB & AtB & Agriculture, Hunting, Forestry And Fishing & pdr wht gro v\_f osd c\_b pfb ocr ctl oap rmk wol frs fsh \\
\hline 
C & C & Mining And Quarrying & coa oil gas omn \\
\hline 
D & 15t16 & Food , Beverages And Tobacco & cmt omt vol mil pcr sgr ofd b\_t \\
Total Manufacturing & 17t19 & Textiles, Textile , Leather And Footwear & tex lea \\
& 20 & Wood And Of Wood And Cork & lum \\
& 21t22 & Pulp, Paper, Paper , Printing And Publishing & ppp \\
& 23t25 & Chemical, Rubber, Plastics And Fuel & crp p\_c \\
& 23 & Coke, Refined Petroleum And Nuclear Fuel & p\_c \\
& 24 & Chemicals And Chemical Products & crp \\
& 25 & Rubber And Plastics & crp \\
& 26 & Other Non-Metallic Mineral & nmm \\
& 27t28 & Basic Metals And Fabricated Metal & is nfm fmp \\
& 29 & Machinery, nec & ome \\
& 30t33 & Electrical and Optical Equipment & ome \\
& 34t35 & Transport Equipment & mvh otn \\
& 36t37 & Manufacturing nec; Recycling & omf \\
\hline 
E & E & Electricity, Gas and Water Supply & ely gdt wtr \\
\hline 
F & F & Construction & cns \\
\hline 
G & 50 & Sale, Maintenance and Repair of Motor Vehicles and Motorcycles; Retail Sale of Fuel & trd \\
Wholesale and Retail Trade & 51 & Wholesale Trade and Commission Trade, Except of Motor Vehicles and Motorcycles & trd \\
& 52 & Retail Trade, Except of Motor Vehicles and Motorcycles; Repair of Household Goods & trd \\
\hline 
H & H & Hotels and Restaurants & trd \\
\hline 
I & 60t63 & transport and storage & otp wtp atp \\
Transport and Storage and Communication & 64 & Post and Telecommunications & cmn \\
\hline 
JtK & J & Financial Intermediation & ofi isr \\
\begin{flushleft}Finance, Insurance, Real Estate and Business Services \end{flushleft}& 70 & Real Estate Activities & obs \\
K & 71t74 & Renting of M\&Eq and Other Business Activities & obs \\
\hline 
LtQ & L & Public Admin and Defence; Compulsory Social Security & osg \\
Community Social & M & Education & osg \\
and Personal Services & N & Health and Social Work & osg \\
 & O & Other Community, Social and Personal Services & osg \\
 & P & Private Households With Employed Persons & ros \\
& Q & Extra-Territorial Organizations and Bodies & osg \\

\hline
\end{tabular}
\caption{Correspondance between the industrial levels used in KLEMS and GTAP.}
\label{IndKLEMSGTAP}
\end{table}

Although the IMACLIM model uses 12 sectors and the KLEMS database provides capital input data for a 32 sector division, in three cases IMACLIM uses a more detailed subdivision of the corresponding KLEMS aggregation level: 
\begin{itemize}
\item The IMACLIM sectors Coal, Crude oil and Natural Gas are lumped together with non-energy mining activities in the KLEMS database at the "Mining and quarrying" level. Non-energy mining activities correspond to the GTAP sector omn and are a part of Industry in IMACLIM. 
\item In the KLEMS database, Electricity supply data are aggregated with Gas and water supply. 
\item IMACLIM disaggregates Transport activities in Air, Sea and Other transport, which is not the case in the KLEMS database. 
\end{itemize} 

To disaggregate the KLEMS capital input data, capital revenue data have been used from GTAP. Denote $GTAPCap_{i}$ the capital compensation of GTAP sector i, then the weights used to disaggregate the three aforementioned KLEMS industrial sectors are: 
\begin{equation}
s^{cap}_{i} = \frac{GTAPCap_i}{ \sum_j GTAPCap_{j}}
\end{equation}
where j sums over the GTAP sectors that aggregated together constitute the KLEMS industrial sector in question. Explicitly, i,j respectively run over \{coa, oil, gas, omn\}, \{ely, gdt, wtr\} and \{otp, atp, wtp\} to disaggregate mining of energy producing materials from non-energy producing materials, electricity distribution from gas and water supply, and the different types of transport activities. 

Table \ref{IndKLEMSIMACLIM} shows how the KLEMS industries aggregate to the 12 IMACLIM industries. 


\renewcommand{\arraystretch}{1.2}
\begin{table}
\setlength{\tabcolsep}{2pt}
\begin{tabular}{| r p{340 pt}|}
\hline
Agriculture = & Agriculture, hunting, forestry and fishing  \\
\ & + Food , beverages and tobacco \\
\hline
Coal = & $s^{cap}_{coa}\ \times$ Mining and quarrying \\
\hline
Crude oil = & $s^{cap}_{oil}\ \times$ Mining and quarrying \\
\hline
Gas = & $s^{cap}_{gas}\ \times$ Mining and quarrying \\ \ & + $s^{cap}_{gdt}\ \times$ Electricity, gas and water supply \\
\hline
Industry = & $s^{cap}_{omn}\ \times$ Mining and quarrying \\ \ & + Textiles, textile , leather and footwear \\ \ & + Wood and of wood and cork \\ \ &  + Pulp, paper, paper , printing and publishing \\ \ &  + Chemicals and chemical products \\ \ &  + Rubber and plastics \\ \ &  + Other non-metallic mineral \\ \ &  + Basic metals and fabricated metal \\ \ &  + Machinery, nec \\ \ &  + Electrical and optical equipment \\ \ & + Transport equipment  \\
\hline
Other industries = & $s^{cap}_{wtr}\ \times$ Electricity, gas and water supply \\ \ & + Wholesale and retail trade \\ \ & + Hotels and restaurants \\ \ & + Post and telecommunications \\ \ & + Financial intermediation \\ \ & + Real estate activities \\ \ & + Renting of m\&eq and other business activities \\ \ & + Public admin and defence; compulsory social security \\ \ & + Education \\ \ & + Health and social work \\ \ & + Other community, social and personal services \\ \ & + Private households with employed persons \\ \ & + Extra-territorial organizations and bodies \\
\hline
Refined products = & Coke, refined petroleum and nuclear fuel \\
\hline
Electricity = & $s^{cap}_{ely}\ \times$ Electricity, gas and water supply \\
\hline
Construction = & Construction \\
\hline
Air Transport = & $s^{cap}_{atp}\ \times$ Transport and storage \\
\hline
Sea Transport = & $s^{cap}_{stp}\ \times$ Transport and storage \\
\hline
Other Transport = & $s^{cap}_{otp}\ \times$ Transport and storage \\
\hline
\end{tabular}
\caption{Mapping between the IMACLIM industrial sectors and the industries used in KLEMS}
\label{IndKLEMSIMACLIM}
\end{table}

\subsubsection{Mapping product types to industries}

Mapping the list of 11 asset types in the KLEMS database, aggregated to 8 asset types for which capital input data are provided (see table \ref{CapAssets}), is straightforward when making the assumption that every product type is produced by one industrial sector. 

The classification of products in KLEMS is based on the Statistical Classification of Products by Activity in the European Economic Community (CPA)\footnote{\href{http://ec.europa.eu/eurostat/ramon/nomenclatures/index.cfm?TargetUrl=LST_NOM_DTL&StrNom=CPA_2008&StrLanguageCode=EN&IntPcKey=&StrLayoutCode=HIERARCHIC}{CPA} consulted June 2106} and is closely related to NACE, the EU classification of industries. Table \ref{CapAssets} maps the 8 assets for which capital input data are provided to the IMACLIM sectors.

The publishing of \emph{Software} is, under CPA, classified as part of Publishing services, which falls under Information and Communication services. Following this classification, Software falls under the NACE industry classification 64 Post and Telecommunications, which corresponds to cmn in GTAP.




\begin{table}
\setlength{\tabcolsep}{2pt}
\begin{tabular}{| r p{200 pt}|}
\hline
Industry =& Computing equipment \\ \ & + Communications equipment \\ \ & + Transport equipment \\ \ &  + Other machinery and equipment  \\
\hline
Other industries =& Software \\ \ & + Other assets \\
\hline
Construction = & Total Non-residential investment \\ \ & + Residential structures \\

\hline
\end{tabular}
\caption{Mapping between the IMACLIM industrial sectors and the KLEMS asset types}
\label{IndKLEMSIMACLIM}
\end{table}

\section{Capital input data: missing values}

\renewcommand{\arraystretch}{1.3}
\begin{table}
\setlength{\tabcolsep}{6pt}
\hskip-2.2cm\begin{tabular}{| p{60pt} | l | p{70pt} | l| p{70pt} | p{70pt} | p{70pt} |}
\hline

	&		&	industry-levels	&    &		  		&	Variables	&	Additional sources	\\
IMACLIM regions	&	Country	&	P	&	Q	&	LtQ	&	Price index	&		\\
\hline
Enlarged Europe	&	Austria	&	ok	&	missing	&	ok	&	ful/part missing	&		\\
	&	Belgium	&		&		&		&		&		\\
	&	Denmark	&	ok	&	missing	&	ok	&	ful/part missing	&		\\
	&	Finland	&	missing for Ip and D, partially missing for Iq	&	missing	&	missing for D. Ip and Iq missing for all but IT	&	ful/part missing	&		\\
	&	France	&		&		&		&		&		\\
	&	Germany	&	ok	&	missing	&	ok	&	ful/part missing	&		\\
	&	United Kingdom	&	missing	&	missing	&	ok	&	ful/part missing	&		\\
													
	&	Ireland	&		&		&		&		&		\\
	&	Italy	&	ok	&	missing	&	ok	&	ful/part missing	&		\\
													
	&	Netherlands	&	missing	&	missing	&	ok	&	ful/part missing	&		\\
													
	&	Spain	&	missing	&	missing	&	ok	&	ok	&	for K, CAP, D, asset level Other missing	\\
	&	Sweden	&	missing for Ip, K, VAP, D	&	missing	&		&	price index RStruc	&	product-level Other for all variables missing	\\
																
	&	Czech Republic	&	ok	&	missing	&	ok	&	ful/part missing	&		\\
	&	Hungary	&		&		&		&		&		\\							
													
	&	Slovenia	&		&		&		&		&		\\
\hline
													
Canada	&		&		&		&		&		&		\\
\hline
USA	&		&	missing	&	missing	&	ok	&		&		\\
\hline
OCDE Pacifique	&	Australia	&	missing	&	missing	&	ok	&	ful/part missing	&		\\
	
\hline
\end{tabular}
\caption{Sources of missing values}
\label{missingvalues}
\end{table}

Generally the KLEMS data base provides capital input data from 1970 on. For some countries, data are available only from a later starting date \footnote{See appendix \ref{AppMV}, table \ref{nanCountries}.}:
\begin{itemize}
\item Austria - 1970 - 2007
\item Australia - 1976 - 2007 (no data 1970 - 1975)
\item Tcheque rep - 1995 - 2007 (no data 1970 - 1994)
\item Denmark - 1970 - 2007
\item Spain - 1970 - 2007
\item Finland - 1970 - 2007
\item Germany - 1991 - 2007 (no data 1970 - 1990)
\item Italy - 1970 - 2007
\item The Netherlands - 1970 - 2007
\item Sweden - 1993 - 2007 (no data 1970 - 1992)
\item UK - 1970 - 2007
\item USA - 1970 - 2007
\item France
\item Ireland
\item Belgium
\item Japan
\item Slovenia
\end{itemize}

Data for the industry-level P (Private households with employed persons) are entirely missing for the countries Australia, Spain, the Netherlands, UK, USA. For Austria, the Czeque Republic, Denmark, Germany and Italy, data are fully available for this industry-level. For Finland, data for the industry-level P are missing for the variables Ip and D, are available for CAP, I and K, and partially unavailable for Iq. For Sweden, data are entirely missing for the level P for the variables Ip, K, CAP and D, and available (and equal to 0) for the variables I and Iq, except for for the product-level Other for which they are missing. See tables \ref{nanIndustries_0_20} and \ref{nanIndustries_21} and \ref{nanCountries_noP}.

Data for the industry-level Q (Extra-territorial organizations and bodies) are entirely missing for all countries Austria, Australia, Tcheque rep, Denmark, Spain, Finland, Germany, Italy, The Netherlands, Sweden, UK, USA. See tables \ref{nanCountries_noP} and \ref{nanCountries_noPQ}.

KLEMS provides data directly on the aggregated level LtQ regrouping Community Social and Personal Services. Data on this level are consistently available for the countries Australia, Austria, Czech Republic, Denmark, Spain, Germany, Italy, the Netherlands, UK and USA. For Finland, LtQ-level data are available for K capital stock and I nominal capital investment but for the capital formation price index and real investment only available for investment in IT and missing for all other product-levels. LtQ-level data are entirely missing for D Cunsumption of capital. For Sweden, there are 7 counts of missing data for LtQ ; 6 of those stem from missing data for the product-level Other, the 7th comes from missing data for the price index of RStruc asset-types. So generally, LtQ data are also available for Sweden.  See tables \ref{nanCountries} and \ref{nanCountries_noLtQ}.

The capital formation price index is fully or partially missing for Australia, Austria, Czech rep., Denmark, Finland, Germany, Italy, the Netherlands, Sweden, UK and USA. See tables \ref{nanCountries_noPQ} and \ref{nanCountries_noPQ_noIp}. It is fully available for Spain.

In addition to previuously mentioned missing variables, for Spain, data are missing for capital stock, capital compensation and consumption of fixed capital for the asset-level "Other". 

For Finland, data are not available for capital formation and capital consumption, for the industry-level PtQ, which is also reflected in missing values at the aggregated total industry-level TOT. (Check Finland.)

For Sweden, any capital stock data (variables I, Iq, K, Cap, D) for the product-level "Other" are not available. (check other RStruc)

%\section{Investment in the USA, 1970 - 2007}
\subsection{Choosing variables for answering questions}
The data available for the USA over these 38 years are: for each year, what industry invests in what product. One of the sources the US capital inupt data are based on is the BEA capital flow table for 1997 \todo{insert reference}. Referring to this table provides a more concrete description of in what products precisely the different industry-levels are investing in.    

First questions considered:
\begin{enumerate}
	\item Over the years, has there been an evolution in the overall level of investment, was there an increase or decrease ? To answer this question, we need to define what an increase or decrease in investment would be. The most basic variable - basic in the sense of defined in a straightforward way, without making many hypothesis - in the KLEMS database is nominal investment $I$. But an increase in nominal investment does not mean that capital formation increased, it means more money in the current currency was spent, which might be due to a price increase. The standard solution to this problem is to correct for price increases by using price indices. The KLEMS database provides, albeit with some missing values, price indices specific for the 8 asset-levels and differentiated by industry-level. The price index used is a hedonic price index (see section \ref{hedonic}, \pageref{hedonic}), which specifies a relationship between the prices and the characteristics of products. A hedonic price index changes faster than a more traditional price index for products whose characteristics evolve fast. One of the features of KLEMS is that it distinguishes investment in information and communication technology (ICT) related assets (disaggregated in CT, IT and Software).\footnote{Indeed, the database is constructed amongst other reasons to evaluate the impact on productivity and growth of the new type of capital introduced in the information technology age. To attest of the sincerity of purpose, in addition to the 60 industry-levels for which supply- and use-tables are available for most EU countries, 8 extra industry-levels were added in the KLEMS database: 8 industries that ``either stand out in terms of skill and R\&D intensity, or in terms of ICT investment intensity or ICT share in output' \cite{timmer_eu_2007}}. Since ICT assets have known dramatic improvements in characteristics such as processing power and memory, this leads to a price index that decreases much more for IT assets than for other types of assets. Hence, using real investment $I_q$ (KLEMS notation), introduces (a lot of) information on the quality change of this asset type, which is strictly speaking true for all asset types, but which is faster changing for IT assets. So if one would want to evaluate scenarios of future investment against past patterns on the basis of the KLEMS real investment variable, one needs to make a hypothesis on the evolution of the price index of ICT assets and thus of their quality improvements. Possible alternatives are:
	\begin{itemize}
		%\item Use nominal investment corrected with the over-all-industries price index of non-ICT assets. The evolution of the price index on non-ICT assets is more what we expect, increasing steadily over time. The evolution of the ICT-price index increases up to 1984 after which it decreases faster up to 2007 than the nonICT index increases over the period 1970-2007. (One can decompose the evolution in the evolutions of IT, CT and Soft.) The interpretation of this measure is a bit tricky: real variables typically allow comparison in the sense of larger or smaller value of something (in price of a baseyear), from one year to another. For the case under consideration this would be: more value directed towards investment than the previous year. If we correct the price of computers with a general price index, 
		\item Use nominal investment corrected for value added $I/VA$. This variable addresses the question ``What share of national income has been allocated to capital formation by industry-level $x$ in product-level $y$''. Here, an increase in investment means that a larger of share of national income has been used for capital formation. The drawback of this variable is that it introduces, into the measure of investment evolution, variablity of VA. 
		\item A variable that in the spirit of  $I/VA$ conveys whether a larger or smaller portion of the national budget has been directed towards capital formation, but without adding new variablility, is nominal investment $I$ corrected for the best linear approximation of VA, instead of year-by-year comparing I with the actual measurement of $VA$. \todo{Explain how variable is constructed}%This variable does not introduce extra variability and loosely conveys what portion of national income has been directed towards investment.    
	\end{itemize} 
	\item For a specific, or a few specific years, what industries invest in what products? What industries have (dis)simular investment profiles, and what assets are the most similar in the sense of being formed by the same industrial sectors ? \label{question2}
	%\begin{itemize}
		%\item To perform CA for systemetic evaluation of what industries invest in what products, etc., ponder once more on what variable to compare with what metric.
	%\end{itemize} 

	\item How did what was described in question \ref{question2} evolve over the 38-year period under consideration?
	%\begin{itemize}
		\item Table with min and max growth rate of I and maybe other vars too. Min and Max portion of VA, min and max share.
		%\item To consider the evolution over the period, ponder once more on what variable to compare with what metric.
	%\end{itemize} 
\end{enumerate} 

% latex table generated in R 3.3.0 by xtable 1.8-2 package
% Tue Sep 06 11:50:36 2016
\begin{table}[ht]
\setlength\tabcolsep{6pt}
\centering
\begingroup\footnotesize
\begin{tabular}{lrrrrrrrrr}
  \hline
growthrates& nobs & NAs & Minimum & Maximum & Mean & Stdev & Skewness & Kurtosis & period\\ 
  \hline
I & 41 & 4 & -4.1 \% & 20.7 \% & 7.5 \% & 5.9 \% & 0.1 \% & -0.4 \% & 1970-2007 \\ 
  VA & 41 & 11 & 3.2 \% & 13.1 \% & 6.6 \% & 2.5 \% & 1.0 \% & 0.3 \% & 1977-2007 \\ 
 I/VA & 41 & 11 & -7.0 \% & 6.8 \% & -0.1 \% & 3.7 \% & -0.2 \% & -1.1 \% & 1977-2007 \\ 
   I/$\mathrm{VA_{trend}}$ & 41 & 11 & -13.7 \% & 7.4 \% & -1.2 \% & 5.6 \% & -0.4 \% & -0.8 \% & 1977-2007 \\ 
   \hline
\end{tabular}
\endgroup
\caption{Statistics on growthrates, total industry, 1970-2007, USA} 
\label{table:Summary, growthrates_USA}
\end{table}
 
To evaluate the variability introduced by deflating $I$ by $VA$, \Cref{table:Summary, growthrates_USA} shows statistics of the growthrates of both variables.  Both the spread between the maximum and the minimum growth rate and the standard deviation point towards nominal investment being the most variable variable ; deflating I for the trend of VA takes away some variability but is much more reduced by yearly deflating I by measured VA. The latter is due to the fact that both are influenced by common chocks. 


% Insert graph: evolution I, I/VA, I/trendVA, Iq, I/IpnonGFCF for USA, 1970 - 2007.  
% For I, I/VA and maybe also the variable chosen (I/trendVA?),  




\clearpage
\section{Investment in the USA, 1970 - 2007, reduced system analysis} 
\subsection{Reduced systems}
%\todo{Same graph for K. Is USA still on the lower end for K? Is GFCF/VA low in the USA because of high K already, low labour costs, accounting differences, less redistribution (residential structures, travel equipment) ? }
The variables in the KLEMS capital input data, amongst which nominal investment ($I$) and capital stock ($K$), are detailed according to 4 categorical variables: country, year, product, industry. For each quantitative variable, 6 types of contingency tables can be considered: product x year, industry x product, industry x year, country x product, country x year, country x industry. In the following sections, the first 3 types of contingency tables will be considered for the USA. 

\begin{figure}[!h]
    \centering
    \includegraphics[scale=0.88]{img/plotted_I_K_VA/I_ShareVA-Prod-yScaleFree.pdf}
    \caption{\label{fig:I_ShareVA-gridCountries-TOT}Capital formation as share of value added (\%), source: KLEMS.}
\end{figure} 
\clearpage

\subsection{Choosing variables for answering questions}
The questions considered, which will by addressed by analysing the different types of reduced systems, are:
\begin{enumerate}
	\item Over the years, has there been an evolution in the overall, per asset and/or per industry, level of investment, was there an increase or decrease? \label{question1}
	\item For a specific, or a few specific years, what industries invest in what products? What industries have (dis)simular investment profiles, and what assets are the most similar in the sense of being formed by the same industrial sectors? \label{question2}
	%\begin{itemize}
		%\item To perform CA for systemetic evaluation of what industries invest in what products, etc., ponder once more on what variable to compare with what metric.
	%\end{itemize} 
	\item How did what was described in question \ref{question2} evolve over the 38-year period under consideration?
	%\begin{itemize}
	\todo{Table with min and max growth rate of I and maybe other vars too. Min and Max portion of VA, min and max share.}
		%\item To consider the evolution over the period, ponder once more on what variable to compare with what metric.
	%\end{itemize} 
\end{enumerate} 

To answer question \ref{question1} and compare investment levels over years, using the nominal variable for investment $I$ doesn't distinguish between increasing levels originating in price-increases due to inflation, or in actual increasing amounts of durable assets.

The standard solution to the problem of correcting for price increases, is by using price indices. The price index provided by the KLEMS database  is a hedonic price index (see section \ref{hedonic}, \pageref{hedonic}), which specifies a relationship between the prices and the characteristics of products. Since ICT assets are fast changing in quality, using real investment $I_q$ introduces (a lot of) information on the quality change of this asset type%, which is strictly speaking true for all asset types, but which is faster changing for IT assets
. To evaluate scenarios of future investment against past patterns on the basis of the KLEMS real investment variable, hypothesis are needed on the future evolution of the price index of ICT assets and thus of their quality improvements. (An objection or not? ICT investment is only a few percentages of GFCF.) Possible alternatives are:
%One of the sources the US capital inupt data are based on is the BEA capital flow table for 1997 \todo{insert reference}. Referring to this table provides a more concrete description of in what products precisely the different industry-levels are investing in.    

	%\item The standard solution to the problem is to correct for price increases by using price indices. The KLEMS database provides, albeit with some missing values, price indices specific for the 8 asset-levels and differentiated by industry-level. The price index used is a hedonic price index (see section \ref{hedonic}, \pageref{hedonic}), which specifies a relationship between the prices and the characteristics of products. A hedonic price index changes faster than a more traditional price index for products whose characteristics evolve fast. One of the features of KLEMS is that it distinguishes investment in information and communication technology (ICT) related assets (disaggregated in CT, IT and Software).\footnote{Indeed, the database is constructed amongst other reasons to evaluate the impact on productivity and growth of the new type of capital introduced in the information technology age. To attest of the sincerity of purpose, in addition to the 60 industry-levels for which supply- and use-tables are available for most EU countries, 8 extra industry-levels were added in the KLEMS database: 8 industries that ``either stand out in terms of skill and R\&D intensity, or in terms of ICT investment intensity or ICT share in output' \cite{timmer_eu_2007}}. Since ICT assets have known dramatic improvements in characteristics such as processing power and memory, this leads to a price index that decreases much more for IT assets than for other types of assets. Hence, using real investment $I_q$ (KLEMS notation), introduces (a lot of) information on the quality change of this asset type, which is strictly speaking true for all asset types, but which is faster changing for IT assets. So if one would want to evaluate scenarios of future investment against past patterns on the basis of the KLEMS real investment variable, one needs to make a hypothesis on the evolution of the price index of ICT assets and thus of their quality improvements. Possible alternatives are:
	\begin{itemize}
		%\item Use nominal investment corrected with the over-all-industries price index of non-ICT assets. The evolution of the price index on non-ICT assets is more what we expect, increasing steadily over time. The evolution of the ICT-price index increases up to 1984 after which it decreases faster up to 2007 than the nonICT index increases over the period 1970-2007. (One can decompose the evolution in the evolutions of IT, CT and Soft.) The interpretation of this measure is a bit tricky: real variables typically allow comparison in the sense of larger or smaller value of something (in price of a baseyear), from one year to another. For the case under consideration this would be: more value directed towards investment than the previous year. If we correct the price of computers with a general price index, 
		\item Use nominal investment corrected for value added $I/V\!\!A$. This variable addresses the question ``What share of national income has been allocated to capital formation by industry-level $x$ in product-level $y$''. Here, an increase in investment means that a larger of share of national income has been used for capital formation. The drawback of this variable is that it introduces, into the measure of investment evolution, variablity of VA. 
		\item A variable that in the spirit of  $I/V\!\!A$ conveys whether a larger or smaller portion of the national budget has been directed towards capital formation, but without adding new variablility, is nominal investment $I$ corrected for the best linear approximation of VA, instead of year-by-year comparing I with the actual measurement of $V\!\!A$. \todo{Explain how variable is constructed}%This variable does not introduce extra variability and loosely conveys what portion of national income has been directed towards investment.    
	\end{itemize} 


% latex table generated in R 3.3.0 by xtable 1.8-2 package
% Tue Sep 06 11:50:36 2016
\begin{table}[ht]
\setlength\tabcolsep{6pt}
\centering
\begingroup\footnotesize
\begin{tabular}{lrrrrrrrrr}
  \hline
growthrates& nobs & NAs & Minimum & Maximum & Mean & Stdev & Skewness & Kurtosis & period\\ 
  \hline
I & 41 & 4 & -4.1 \% & 20.7 \% & 7.5 \% & 5.9 \% & 0.1 \% & -0.4 \% & 1970-2007 \\ 
  VA & 41 & 11 & 3.2 \% & 13.1 \% & 6.6 \% & 2.5 \% & 1.0 \% & 0.3 \% & 1977-2007 \\ 
 I/VA & 41 & 11 & -7.0 \% & 6.8 \% & -0.1 \% & 3.7 \% & -0.2 \% & -1.1 \% & 1977-2007 \\ 
   I/$\mathrm{VA_{trend}}$ & 41 & 11 & -13.7 \% & 7.4 \% & -1.2 \% & 5.6 \% & -0.4 \% & -0.8 \% & 1977-2007 \\ 
   \hline
\end{tabular}
\endgroup
\caption{Statistics on growthrates, total industry, 1970-2007, USA} 
\label{table:Summary_growthrates_USA}
\end{table}
 
To evaluate the variability introduced by deflating $I$ by $VA$, \cref{table:Summary_growthrates_USA} shows statistics of the growthrates of both variables.  Both the spread between the maximum and the minimum growth rate and the standard deviation point towards nominal investment being the most variable variable ; deflating I for the trend of $V\!\!A$ takes away some variability but is much more reduced by yearly deflating I by measured $V\!\!A$. The latter is due to the fact that both are influenced by common chocks. 


When comparing investment data across countries, using the variables $\frac{I}{GFCF}$ and $\frac{I}{V\!\!A}$, has the added advantage of also correcting for the sizes of the economies and for exchange rates compared to the nominal investment series $I$.
%In the following sections, these reduced systems will be considered for the variables $\frac{I}{GFCF}$ and $\frac{I}{VA}$, which correct for the size of an economy and for exchange rates compared to the nominal series.


\subsection{VA, I, K shares and levels per industry-level}
\input{img/summarizeIndustries/VA-I-K_indSummary_USA_2007.tex}
%\Cref{table:VA-I-K_indSummary_USA_2007} shows for the USA, for 2007, several variables related to the investment profile for the 11 industry-levels.% specified in the rows according to which the total industry is disaggregated. 
%\begin{itemize}
%\item The first 3 columns are respectively the shares per industry in total value added (shareVA), capital formation (shareI) and capital stock (shareK) (\SI{}{\percent}).
%\item The last 3 columns are the levels per industry of value added (VA) and capital formation (I) in billions current dollars, and capital stock (K) in billions constant 1995 dollars. 
%\end{itemize}

\subsection{Product $\times$ industry}


\todo{Retail and hotels lumped together in G+H: similar investment strucuture?}
\Cref{table:I_USA_2007} constitutes the reduced system product $\times$ industry for nominal investment in the USA in 2007. The table, which forms a use-table, details investment allocated by 11 industry-levels\footnote{
    The 11 industry-levels considered correspond to aggregated levels of the original 32 levels of KLEMS : \cref{IndustriesKLEMS}: Agr is AtB Agriculture, Mining is C Mining and quarrying, Manufact is D Total Manufacturing, including for example food, refined petroleum products, telecommunication equipment and transport equipment, Elec Gas is level E and includes water supply, Construc is F Construction, Sales is G+H, where G is Wholesale and retail trade and H Hotels and restaurant, Trans Comm is I Transport, storage, post and communication, Finance is J Financial intermediation, Real estate is 70 real estate activities and RE Business 71t74 Renting of machinery and business activities like R\&D and advertising, Community LtQ represents service activities, both social, provided by the state such as public administration, defense and education, and personal serivces, such as personnel employed by private households.
} to the asset-levels specified in the columns. The left half of the table considers the 8 KLEMS asset-levels, the right part of the table aggregates these 8 levels in 2: assets related to information and communication technologies (ICT) and other types of capital (nonICT) \footnote{A detailed description of the 8 asset-levels can be found in section \ref{ProductTypes} and \cref{table:CapAssetsDetailed} therein}.%in each of the 8 KLEMS product-types. On the right-hand side of the table, the columns NonICT and ICT are an aggregation of the same data\footnote{ICT=CT+IT+Soft, NonICT=RStruc + OCon + TraEq + OMach + Other. See section \ref{ProductTypes}, page \pageref{ProductTypes}}. 
In what follows the question ``Who invests in what'' will be answered by analysing the contingency table with correspondance analysis.

\input{img/summarizeIndustries/I_USA_2007.tex}

Concentrating on the left part of \cref{table:I_USA_2007}, several transformations of this $11 \times 8$ contingency table clarify which sectors invest mostly in what products and which sectors have (dis)similar investment profiles:

\begin{itemize}
\item \cref{table:I-rowProfiles_USA_2007} shows the rowprofiles, which detail for every industry what share of its total investment level is allocated to a specific product, $I_{ip}/I_i$, where $I_i = \sum_{p} I_{ip}/n_p$, 
\item \cref{table:I-colProfiles_USA_2007} shows the columns profiles, which detail for every capital asset what share of its formation is done by what industry, $I_{ip}/I_p$, where $I_p = \sum_i I_{ip}/n_i$,
\item table \ref{table:distancesInd_USA_2007} shows the distances $d_{ij}$ between the rowprofiles, i.e. between the investment profiles of the different industries, \begin{equation}
    d_{ij}=\sum_p \frac{ (I_{ip}/I_i - I_{jp}/I_j)^2}{\frac{1}{n_i} \sum_k I_{kp} / I_k},
\end{equation}  
\item \cref{table:I-independence_USA_2007} shows what the investment levels per industry and product would be in case both categorical variables are independent, $$
I^{dep}_{ip}= \frac{I_i I_p}{GFCF},
$$
\item \cref{table:I-deviations_USA_2007} shows how actual investment levels differ from what the levels would be under the independency hypothesis, 
$$
I_{ip}- I^{dep}_{ip},
$$
\item \cref{table:I-chi2_USA_2007} shows the contributions of each data point to the chi square of independance statistic, $\chi^2_{ip}$ which are easurements of to what extent the data differ with the linear dependency hypothesis: 
$$
\chi^2_{ip} = \frac{(I_{ip}- I^{dep}_{ip})^2}{I^{dep}_{ip}}.
$$
\end{itemize}

\input{img/summarizeIndustries/I-rowProfiles_USA_2007.tex}
\input{img/summarizeIndustries/I-colProfiles_USA_2007.tex}
\input{img/summarizeIndustries/distancesInd_USA_2007.tex}
\input{img/summarizeIndustries/I-independence_USA_2007.tex}
\input{img/summarizeIndustries/I-deviations_USA_2007.tex}
\input{img/summarizeIndustries/I-chi2_USA_2007.tex}

Referring to \cref{table:I-rowProfiles_USA_2007}, the real estate sector stands out in that \SI{94}{\percent} of its investments are allocated to residential structures. Conversely, it is virtually the only sector investing in residential structures: \SI{98}{\percent} of capital formation of residential structures is done by the real estate sector (\cref{table:I-colProfiles_USA_2007}).

Continuing to look at industries which allocate a lot of investment in a specific asset class, the mining sector spends close to \SI{80}{\percent} on non-residential investment, which includes both non-residential structures and infrastructure. A few other industries allocate a large share of investment on non-residential structures: Community (\SI{63}{\percent}), Elec Gas(\SI{52}{\percent}), and Finance, Trans Comm and Sales around (\SI{30}{\percent}).

The other asset that receives large shares of investment of some industries is Other machinery and equipment: the agricultural sector's share in investment in this asset is \SI{59}{\percent}, Construction sector's \SI{53}{\percent}, the manufacturing sector's \SI{51}{\percent} and the retail sector's  \SI{30}{\percent}.

Apart from the transport and communications sector which allocates \SI{28}{\percent} of its capital investment to communication technology-related assets and the Real estate business industry which allocates \SI{34}{\percent} to software, there is no other case where an industry spends more than \SI{25}{\percent} of its fixed capital  investment on one of the asset-levels under consideration.


\subsubsection{(Dis)similarity between industries' investment profiles}
%It was mentioned above that the Real estate sector has a very different investment profile than the other sectors, since there is an almost 1 to 1 correspondancy between this sector and the capital formed by residential structures.

%To evaluate to what extent the different industries have similar or different investment profiles, the distance $d_{ij}$ between the different rowprofiles is used:
%\begin{equation}
%    d_{ij}=\sum_p \frac{ (I_i^p - s_j^p)^2}{\sum_k I_k^p / n_i}
%\end{equation}
%where $I_i^p$ are the shares presented in \cref{table:I-rowProfiles_USA_2007}: $p$ runs over asset-levels, i, j, k run over industries and $n_i$ is the number of industrial sectors. The results for the USA, 2007 are represented in table \ref{table:distancesInd_USA_2007}. 

The distances between the investment profiles confirm that the real estate sector stands apart in its profile from the other sectors: the $d_{Real\ estate ,j}$ where $j$ is any of the other sectors, fluctuates around 100, which is three times higher than the next highest distance between investment profiles: the mining sector has a quite different profile from the Construction sector ($d=28$) and the Real estate business sector ($d=34$) (\cref{table:distancesInd_USA_2007}). 
\begin{itemize}
\item The most similar investment profiles are between Community and both Elec Gas ($d= 1.3$) and Mining ($d=1.6$). Both Community and Mining invest larger than average levels in Other construction, which contributes 4-\SI{5}{\percent} to $\chi^2$, and they overall deviate or not in a similar manner form the average profile, the only exception being Software investment: the Mining sector investing $3 \times$ less than average, which places it just before the Agricultural and Real estate sector (\cref{table:I-deviations_USA_2007}). Together with Elec and Gas, they overall do less than average investment in ICT-assets, again after Agri and RE. Community and Elec and Gas are similar overall, except for Elec and Gas doing proportianally twice as much investment in Other machinery. 
\item The Manufacturing and Construction sectors both invest three times the average in Other machinery, and also in ICT assets, specifically in Software, in which they invest double share of the average.
\end{itemize}

To establish whether the investment profiles are overall significantly linearly independant, the chi square of independence statistic ($\chi^2$) compares actual data with data under the hypothesis of linear dependency (\cref{table:I-independence_USA_2007}), i.e. under the hypothesis that the amount of investment in a given asset-level is independent of industry and vice versa.

%Investment data under the hypothesis of linear dependency, as presented in (\cref{table:I-independence_USA_2007}, are:
%$$ I^{dep}_{ij}= \frac{I_p I_i}{GFCF},$$
%where $I_p$ is investment per product by all industrial sectors, $I_i$ is investment in all products per industry and $GFCF$ is total investment.
Summing the deviations of the actual data from the linear dependency hypothesis gives $\chi^2$:
$$
\chi^2 = \sum_{i,p} \chi^2_{ip} \ .
$$

\Cref{table:I-chi2_USA_2007} represents the individual contributions of each data point to total $\chi^2$ which equals 3624186. Under the independency hypothesis the $\chi^2$ statistic follows a $\chi^2$ distribution of ($n_i-1$, $n_p-1$)\footnote{$n_i$, $n_p$ are respectively the number of industry- and asset-levels.} degrees of freedom. The hypothesis of linear dependancy is rejected with almost certainty (p-value=0).

\subsubsection{CA}
Correspondance analysis (CA) allows for a systematic evaluation of which industries invest mostly in what products, what industries have more or less similar investment profiles, and what fixed assets are the most similar in the sense of being formed by the same industrial sectors.

The method seeks to explain overall variance in the data with new variables which are linear combinations of the original categorical variables. The new variables are referred to as dimensions. One can consider an amount of new dimensions up to the number of categories in the categorical variables, at which point \SI{100}{\percent} of original variance is explained. Albeit, the last dimensions might not explain much nor offer significant inisght. The point of CA is to reduce the amount of dimensions and identify the main sources of variance.

The amount of dimensions to consider can be chosen with respect to the eigenvalues of each axis\footnote{\href{http://www.sthda.com/english/wiki/correspondence-analysis-in-r-the-ultimate-guide-for-the-analysis-the-visualization-and-the-interpretation-r-software-and-data-mining\#summary-of-ca-outputs}{Correspondance analysis interpretation, handy online source}}. If the data were randomly distributed, the expected value of the eigenvalues of each axis would equal $1/(1-n_i)$ and $1/(1-n_p)$ in terms of the rows/columns. One can use the rule of thumb that any contribution larger than the maximum of those eigenvalues is important. In our case that is max (1/10,1/7)= \SI{14.2}{\percent}. \todo{Insert dim2-dim3 plot.} Following this rule, 3 dimensions are retained for the investment matrix under consideration (\cref{table:I_CA_ProdInd_contributions}). 


\begin{table}[ht]
	\setlength{\tabcolsep}{4pt}
	\centering
	\footnotesize
  	 \vskip-0.0cm\hskip-2.0cm\begin{tabular}{p{30pt} |l|c| c  c  c  c  c  c  c  c  c  c  c | c  c  c  c  c  c  c  c }
  \hline
	&		&	\rot{eigenvalue}	&	\rot{Agri}	&	\rot{Mining}	&	\rot{Manufacturing}	&	\rot{Elec Gas Wtr}	&	\rot{Construction}	&	\rot{Sale}	&	\rot{Transport Communication}	&	\rot{Finance}	&	\rot{Real estate}	&	\rot{Real Estate Business}	&	\rot{Community}	&	\rot{RStruc}	&	\rot{OCon}	&	\rot{TraEq}	&	\rot{OMach}	&	\rot{Other}	&	\rot{CT}	&	\rot{IT}	&	\rot{Soft}	\\
   \hline
Total	&	dim1\ 	&	90.3	&		&		&	3.4	&		&		&	3.3	&		&		&	73.4	&	3.5	&	6.7	&	74.4	&	8.1	&		&	6.6	&		&		&		&		\\
	&	dim2	&	27.3	&		&	22.7	&		&		&		&		&		&	6.0	&	0.1	&	32.0	&	25.8	&		&	51.6	&	18.0	&		&		&	7.8	&	9.4	&	11.9	\\
	&	dim3	&	15.1	&	8.8	&		&	32.0	&		&		&		&	49.9	&		&		&		&		&		&		&		&	45.1	&		&	43.8	&		&		\\

   \hline
\end{tabular}
%\endgroup
\caption{\label{table:I_CA_ProdInd_contributions}Contributions by industry and asset to the first two dimensions after CA.} 
\end{table}

To interpret the new axes after orthogonalization, the contributions to the definition of the dimensions of each industry and product - which are the projections of the dimensions on the original categories - are evaluated again with respect to what the contributions would be if the data were randomly distributed. Under random distribution, the contribution of each industry would equal $1/n_i =  \SI{9.1}{\percent}$ ; for the asset-levels, the contributions with randomly dsitributed data would be  \SI{12.5}{\percent}. \Cref{table:I_CA_ProdInd_contributions} mentions the eigenvalues for the most important dimensions and the most important contributions to the dimensions. For the total system the dimensions are:

\begin{alignat}{2}
&\mathrm{dim}1 = 73.4 * \mathrm{Real\ estate} && (+\ 6.7 * \mathrm{Community}) \nonumber \\
&\mathrm{dim}2 = 32.0 * \mathrm{RE\ Business} &&+ 25.8 * \mathrm{Community} + 22.7 * \mathrm{Mining} \label{eq:dim_industry} \\
&\mathrm{dim}3 = 49.9 * \mathrm{TransComm} &&+ 32.0 * \mathrm{Manufact} \nonumber \\ \nonumber \\
&\mathrm{dim}1 = 74.4 * \mathrm{RStruc} &&(+ \ 8.1 * \mathrm{OCon}) \nonumber \\
&\mathrm{dim}2 = 51.6 * \mathrm{OCon} &&+ 18.0 * \mathrm{TraEq} + (11.9 * \mathrm{Soft})  \label{eq:dim_product} \\
&\mathrm{dim}3 = 45.1 * \mathrm{OMach} &&+ 43.8 * \mathrm{CT} \nonumber
\end{alignat}


The dimensions analysis makes more precise the remarks made earlier based on inspecting the distances between the investment profiles: 
\begin{itemize}
\item The first dimension captures the previously mentioned correspondance between the Real estate sector and Residential structures. 
\item The second dimension groups 3 industrial sectors that seperate themselves from the other sectors by their investment behaviour in the asset-levels Other construction, Travel equipment and Software. Community and Mining both invest a large share in Other construction, whereas RE Business invests a much smaller share than average in Other construction, but a lot in Travel equipment, contrary to the first 2.\todo{Use detailed tables from BEA to find out what is invested in by whom?}
\item The third dimension opposes the Transport and communication sector which does a lot of investing in communication technology, and the Manufacturing sector which invest is Machinery and equipment other than transport equipment.  
\end{itemize}

One might summarize that the first dimension captures variations in investment according to whether a sector in question invests in residential structures, the second dimension seperates investment in infrastructure from travel equipment, and the the third dimension seperates machinery from CT. 

To get a more detailed picture of what ``Other structures (OCon)'' the industry-levels Community and Mining are investing in, one can, for the USA, refer to the Bureau of Economic Analysis' (BEA) 1997 capital flow table which decomposes capital formation in 123 industries and 180 products (appendix \ref{appendix:classifications}, page \pageref{BEA_capital_flow}). This capital flow table is one of the sources on which the USA KLEMS capital input data are based. It reveals that:
\begin{itemize}
\item investment in ``New highways, bridges and other horizontal construction' is entirely attributed to the Real estate sector,
\item the KLEMS inustry-level Community LtQ's investment in OCon is mainly in public service buildings: hospitals, academic facilities, office buildings, recreation facilities, residential, institutional, and other health facilities, waste treatment plants, 
\item railroad construction is attributed to the Transport and Storage and Communication sector (I, level 60),
\item the Mining industries' investment in OCon mainly represents the wells themselves,
\item the RE Business investment in travel equipment comes from the NAICS industry 5321 consisting of Automotive equipment rental and leasing investing in Automobiles and light trucks.
\end{itemize}

The asset-level Other, which groups both specific types of tangible and intangible assets, finds itself in between Soft and Other machinery.
Elec Gas and Wtr invests both more than average in Other construction and Other machinery. ``Sales'' has the most average profile.


\subsubsection{Contributions to the dimensions}

\emph{France}
\input{img/reducedSystemProdInd/I-dimensions_FRA_1970.tex}
\input{img/reducedSystemProdInd/I-dimensions_FRA_1980.tex}
\input{img/reducedSystemProdInd/I-dimensions_FRA_1990.tex}
\input{img/reducedSystemProdInd/I-dimensions_FRA_2001.tex}
\input{img/reducedSystemProdInd/I-dimensions_FRA_2007.tex}
\clearpage
\emph{USA}

\begin{figure}[ht]
    \centering
    \includegraphics[scale=0.8]{img/reducedSystemProdInd/USA_1970_ProdInd.pdf}
    \caption{\label{fig:CA-I-USA-1970}Correspondance analysis on 8 asset and 11 industry-levels, USA, 1970 source: KLEMS, dimensions 1 and 2.}
\end{figure} 
\begin{figure}[ht]
    \centering
    \includegraphics[scale=0.8]{img/reducedSystemProdInd/USA_1970_ProdInd_dim2-3.pdf}
    \caption{\label{fig:CA-I-USA-1970-dim23}Correspondance analysis on 8 asset and 11 industry-levels, USA, 1970 source: KLEMS, dimensions 2 and 3.}
\end{figure}
\input{img/reducedSystemProdInd/I-dimensions_USA_1970.tex}
\clearpage
\begin{figure}[ht]
    \centering
    \includegraphics[scale=0.8]{img/reducedSystemProdInd/USA_1980_ProdInd.pdf}
    \caption{\label{fig:CA-I-USA-1980}Correspondance analysis on 8 asset and 11 industry-levels, USA, 1980 source: KLEMS, dimensions 1 and 2.}
\end{figure} 
\begin{figure}[ht]
    \centering
    \includegraphics[scale=0.8]{img/reducedSystemProdInd/USA_1980_ProdInd_dim2-3.pdf}
    \caption{\label{fig:CA-I-USA-1980-dim23}Correspondance analysis on 8 asset and 11 industry-levels, USA, 1980 source: KLEMS, dimensions 2 and 3.}
\end{figure}
\input{img/reducedSystemProdInd/I-dimensions_USA_1980.tex}
\clearpage
\begin{figure}[h]
    \centering
    \includegraphics[scale=0.8]{img/reducedSystemProdInd/USA_1990_ProdInd.pdf}
    \caption{\label{fig:CA-I-USA-1990}Correspondance analysis on 8 asset and 11 industry-levels, USA, 1990 source: KLEMS, dimensions 1 and 2.}
\end{figure} 
\begin{figure}[h]
    \centering
    \includegraphics[scale=0.8]{img/reducedSystemProdInd/USA_1990_ProdInd_dim2-3.pdf}
    \caption{\label{fig:CA-I-USA-1990-dim23}Correspondance analysis on 8 asset and 11 industry-levels, USA, 1990 source: KLEMS, dimensions 2 and 3.}
\end{figure}
\input{img/reducedSystemProdInd/I-dimensions_USA_1990.tex}
\clearpage
\begin{figure}[h]
    \centering
    \includegraphics[scale=0.8]{img/reducedSystemProdInd/USA_2001_ProdInd.pdf}
    \caption{\label{fig:CA-I-USA-2001}Correspondance analysis on 8 asset and 11 industry-levels, USA, 2001 source: KLEMS, dimensions 1 and 2.}
\end{figure} 
\begin{figure}[h]
    \centering
    \includegraphics[scale=0.8]{img/reducedSystemProdInd/USA_2001_ProdInd_dim2-3.pdf}
    \caption{\label{fig:CA-I-USA-2001-dim23}Correspondance analysis on 8 asset and 11 industry-levels, USA, 2001 source: KLEMS, dimensions 2 and 3.}
\end{figure}
\input{img/reducedSystemProdInd/I-dimensions_USA_2001.tex}
\clearpage
\begin{figure}[h]
    \centering
    \includegraphics[scale=0.8]{img/reducedSystemProdInd/USA_2007_ProdInd.pdf}
    \caption{\label{fig:CA-I-USA-2007}Correspondance analysis on 8 asset and 11 industry-levels, USA, 2007 source: KLEMS, dimensions 1 and 2.}
\end{figure} 
\begin{figure}[h]
    \centering
    \includegraphics[scale=0.8]{img/reducedSystemProdInd/USA_2007_ProdInd_dim2-3.pdf}
    \caption{\label{fig:CA-I-USA-2007-dim23}Correspondance analysis on 8 asset and 11 industry-levels, USA, 2007 source: KLEMS, dimensions 2 and 3.}
\end{figure}
\input{img/reducedSystemProdInd/I-dimensions_USA_2007.tex}
\clearpage

%\Cref{fig:CA-I-USA-2007} shows, again, the close correspondance between the real estate sector and residential structures.To get a better view on the pattern amongst the other sectors and products, another figure \cref{fig:CA-USA-2007-noResCon} shows the CA representation omitting the aforementioned industrial and asset level.


\fbox{
    \parbox{\textwidth}{
        \textbf{CA vs. PCA:}\newline
            \begin{itemize}
                \item[ANSWER Frank]
                    CA is a form of PCA where distance is not euclidian norm but $\chi_2$.
                    What that means is that data is weighted by row totals and column totals (everything is to be understood as barycenters).
        \item[ANSWER 1]
            PCA works on the values where as CA works on the relative values.
            Both are fine for relative abundance data of the sort you mention (with one major caveat, see later).
            With \% data you already have a relative measure, but there will still be differences.
            Ask yourself do you want to emphasise the pattern in the abundant species/taxa (i.e. the ones with large \%cover), or do you want to focus on the patterns of relative composition?
            If the former, use PCA.
            If the latter use CA.
            What I mean by the two questions is would you want (50, 20, 10) and (5,  2,  1) to be considered different or the the same?
            PCA would consider these very different because of the Euclidean distance used, but CA would consider these two samples as being very similar because the have the same       relative profile.

            The big caveat here is the closed compositional nature of the data.
            If you have a few groups (Sand, Silt, Clay, for example) that sum to 1 (100\%) then neither approach is correct and you could move to a more appropriate analysis via Aitchison's Log-ratio   PCA which was designed for closed compositional data.
            (IIRC to do this you need to centre by rows and columns, and log transform the data.) There are other approaches too.
            If you use R, then one book that would be useful is
            Analyzing Compositional Data with R.
            \end{itemize}
    }
}



\ruben{CA vs. PCA pourquoi? For ruben}


%\clearpage
%\input{img/summarizeIndustries/I_USA_2007_noResCon.tex}
%\input{img/summarizeIndustries/I-rowProfiles_USA_2007_noResCon.tex}
%\input{img/summarizeIndustries/I-colProfiles_USA_2007_noResCon.tex}
%\input{img/summarizeIndustries/I-independence_USA_2007_noResCon.tex}
%\input{img/summarizeIndustries/I-deviations_USA_2007_noResCon.tex}
%\input{img/summarizeIndustries/I-chi2_USA_2007_noResCon.tex}
%\clearpage

%$\chi^22$ =1011616 For the system excluding the real estate sector and residential structures.

%\todo{work with the residuals of investment after acounting for capital stock structure, K}
%\todo{K/VA, the smaller, the less "capital intensive" the economy (or lower energy cost, labour cost ?)}
%\todo{multiply individuals ? create individuals, 1 column, productxindustryxcountry and consider that against years ?} 

\clearpage

\subsection{Year $\times$ product}
The product $\times$ year reduced system is summarized, for a given industry-level and (group of) country(ies), by the matrix detailing - for some measure of investment ($I$, $I/VA$, $I/GFCF$) - how much went to what product per year. It's analysis addresses the questions: ``Is the investment pattern similar over the years for all or some products?'' and ``Are there years for which the product-profile of capital formation is similar or significantly different than for other years ?''. 

\subsubsection{Overall investment trends per product}

\Cref{fig:I_ShareVA-gridCountries-TOT} gives a first impression of the evolution of investment per value added $\frac{I_p}{VA}$ for several product-levels $p$, for all countries, over the period 1970-2007. The visual impression of the upper row of the figure suggests that overall investment as a share of the total national budgets has been going down, on average, over all countries; the same trend in nonICT assets seems slightly clearer. The share of national budgets allocated to ICT investment increases. Beween all the asset-types, the evolution of Software stands out with its evident consistent rise over the period considered. 

The impressions above are put into numbers by performing a simple linear regression on $I_p/V\!\!A$, and thus evaluating whether there is for each asset-level a significant global trend $t_p$over all countries grouped together, over the 38 years under consideration. 
\begin{equation}
\frac{I_{yp}}{VA_y}= i_p + t_p y + u_yp
\end{equation}
The results are in the last 2 columns of \cref{table:I_ShareVA-GFCF-global}. The first 8 columns give statistics of $I/V\!\!A$ per asset-level: the minima and maxima for example state the lowest and highest shares of VA that have been allocated to the formation of the capital asset in question, over all countries, in between 1970 and 2007. In the following sections (appendix...), similar tables per asset-level and statistics per country are presented. 

\Cref{table:I_VA-trends-assets} summarizes for the USA and France whether the share of the national budgets allocated to investment in each asset-level shows a significant trend or not.

\begin{table}[ht]
\setlength\tabcolsep{6pt}
\centering
\begingroup\footnotesize
\begin{tabular}{rrrrrrrrrrr}
  \hline
 & nobs & NAs & Minimum & Maximum & Mean & Stdev & Skewness & Kurtosis & trend & t.pvalue \\ 
  \hline
  GFCF & 615 & 157 & 0.165 & 0.356 & 0.242 & 0.037 & 0.418 & -0.229 & -0.00097 & *** \\ 
\hline
  NonICT & 615 & 157 & 0.137 & 0.338 & 0.218 & 0.040 & 0.246 & -0.520 &  -0.00162 & *** \\ 
  ICT & 615 & 157 & 0.009 & 0.050 & 0.024 & 0.009 & 0.283 & -0.660 & 0.00064 & *** \\ 
\hline
  Con & 615 & 157 & 0.066 & 0.241 & 0.137 & 0.030 & 0.313 & 0.242 & -0.00112 & *** \\ 
  RStruc & 615 & 157 & 0.016 & 0.130 & 0.061 & 0.018 & 0.671 & 1.581 & -0.00062 & *** \\ 
 OCon & 615 & 157 & 0.042 & 0.159 & 0.076 & 0.019 & 0.680 & 0.351 & -0.00050 & *** \\ 
 TraEq & 615 & 157 & 0.009 & 0.045 & 0.022 & 0.006 & 0.870 & 1.044 & -0.00005 & * \\
 OMach & 615 & 157 & 0.022 & 0.103 & 0.052 & 0.014 & 0.622 & 0.050 & -0.00033 & *** \\ 
 Other & 615 & 172 & 0.000 & 0.056 & 0.007 & 0.008 & 2.258 & 5.977 & -0.00010 & ** \\ 
 CT & 615 & 157 & 0.000 & 0.015 & 0.006 & 0.003 & 0.154 & -0.092 & 0.00005 & *** \\ 
 IT & 615 & 157 & 0.002 & 0.029 & 0.009 & 0.005 & 1.147 & 1.423 & 0.00014 & *** \\ 
 Soft & 615 & 157 & 0.000 & 0.029 & 0.009 & 0.006 & 0.800 & 0.205 & 0.00046 & *** \\ 
   \hline
\end{tabular}
\endgroup
\caption{\label{table:I_ShareVA-GFCF-global}I / VA, 1970 - 2010, all industries} 
\end{table}



\begin{table}[t]
\setlength\tabcolsep{6pt}
\centering
\begingroup\footnotesize
\begin{tabular}{l|c|cc|c|ccccccccc}
  \hline
 &  GFCF & NonICT  & ICT & Con\footnote{Con = RStruc +OCon} & RStruc & OCon & TraEq & OMach & Other & CT & IT & Soft \\ 
  \hline
 All countries &  $\downarrow^{***}$ & $\downarrow^{***}$  & $\uparrow^{***}$ & $\downarrow^{***}$ & $\downarrow^{***}$ & $\downarrow^{***}$ & $\downarrow^{*}$ & $\downarrow^{***}$ & $\downarrow^{**}$ & $\uparrow^{**}$ & $\uparrow^{***}$ & $\uparrow^{***}$ \\ 
 USA 		& $\_$ & $\downarrow^{***}$  & $\uparrow^{***}$ & $\downarrow^{*}$ & $\_$  & $\downarrow^{***}$ & $\downarrow^{**}$ & $\downarrow^{***}$ & $\uparrow^{***}$  & $\_$ & $\uparrow^{***}$ & $\uparrow^{***}$ \\ 
 France 	&  $\downarrow^{***}$ & $\downarrow^{***}$  & $\uparrow^{***}$ & $\downarrow^{***}$ & $\downarrow^{***}$           & $\downarrow^{***}$ & $\_$ & $\downarrow^{***}$ & $\downarrow^{*}$ & $\_$ & $\uparrow^{***}$ & $\uparrow^{***}$ \\ 
   \hline
\end{tabular}
\endgroup
\caption[\label{table:I_VA-trends-assets}Trends of $I_p$ / VA, all industries, France: 1970 - 2010, USA: 1977-2007.]{\label{table:I_VA-trends-assets}Trends of $I_p$ / VA, all industries, France: 1970 - 2010, USA: 1977-2007. \protect\footnotemark}

\end{table}
\footnotetext{The asterisks indicate the p-value: *** = 0 - 0.001, ** = 0.001 - 0.01, * = 0.01 - 0.05, $\_$ = not significant}


\subsubsection{(Dis)similarities between assets' investment-evolution}

To account for more complex similarities in evolutions than a linear trend, the contingency matrix is analysed with CA.

To evaluate similarities in evolution in investment in the different products, the distance between the investment evolutions over the years is determined using the variable $I/V\!\!A$:
\begin{equation}  \label{eq:distanceCol_year-prod}
d_{pq} = \sum_y \frac{( \frac{I_{yp}/V\!\!A_y}{\sum_z I_{zp}/V\!\!A_z}  -\frac{I_{yq}/V\!\!A_y}{\sum_z I_{zq}/V\!\!A_z} )^2}{    \frac{1}{n_p}  \sum_r      \frac{I_{yr}/V\!\!A_y}{\sum_z I_{zr}/V\!\!A_z}       }.
\end{equation}
where $p$, $q$ and $r$ run over product-levels and $y$ and $z$ over years. The choice of the variable $I_p/V\!\!A$ instead of $I_p$ is made because calculating (distances between) asset-profiles involves summing, or averaging, the measure for investment over years. For nominal investment $I_yp$ this is not meaningful. The measures that allow for comparibility over years are $I_{yp}/GFCF_y$ and $I_{yp}/V\!\!A_y$. The latter is here chosen for the following reason: if the overall level of investment goes up or down\footnote{Whatever going up or down means. Here to be taken as what part of ... hence the choice of I/V\!\!A. See section ...}, say hypothetically due to a marked evolution in only one product-level, then using shares of GFCF will show a change in all asset-levels. $I_{yp}/V\!\!A_y$ will show no evolution in the product-levels for all but one. 

%Using $I_{py}/GFCF_y$ comes down to doing a standard contingency table analysis, where the year-profiles would be calculated by weighting the data by the sum of the year-profile. Using $I_{py}/V\!\!A_y$ thus diverges slightly from the standard contingency-table method.
 
%The product x year system can be analysed by considering the contingency matrix products x years, for the variables $\frac{I_p}{I_{GFCF}}$ (what share of total investment is directed towards a particular product) and/or $\frac{I_p}{VA}$ (what share of total national income is directed towards the investment in a particular product).

\input{img/summarizeYears/distancesYears_USA_I_VA_prod.tex}

\begin{figure}[!h]
    \centering
    \includegraphics[scale=0.8]{img/summarizeYears/I_ShareVA_evol-USA-TOT.pdf}
    \caption{\label{fig:I_VA-I_evol-USA-TOT}Relative evolution of nominal investment I in 8 asset-levels, USA, source: KLEMS.}
\end{figure} 

\Cref{table:distancesYears_USA_I_VA_prod} shows what was remarked by looking at already mentioned \cref{fig:I_ShareVA-gridCountries-TOT}: the investment evolution of software is distinctly different from that of the other assets. It is most different from OCon, TraEq and OMach, which are the three assets that know a downward trend (\cref{table:I_VA-trends-assets}), in contrast to Software which knows a clear upward trend. The next ``most different'' from Soft assets are RStruc and CT which both do not show a clear trend for the USA, and the closest profiles to Soft are IT and Other which are together with Software the asset-levels that all show an upward trend. \Cref{fig:I_VA-I_evol-USA-TOT}.

The most similar industry-levels are OCon, TraEq and OMach which are the ones that show a significant downward trend. Inspecting \cref{fig:I_VA-I_evol-USA-TOT} shows that OCon and OMach follow a similar evolution ; their profiles are the most similar (d=0.04).

The two asset-levels that show no overall trend, CT and RStruc, are more different from one another (d=0.21) than from Other \& the three that know an overall downward trend. Thus it is not the overall trend that explains their modest dissimilarity. From \cref{fig:I_VA-I_evol-USA-TOT} we learn that they follow opposite tendencies during the first and last third of the period considered; they only converge from the mid-eighties to the mid-nineties.
\clearpage

\subsubsection{(Dis)similarities between year-profiles}
\input{img/summarizeYears/distancesYears_USA_I_VA_part1.tex}
\input{img/summarizeYears/distancesYears_USA_I_VA_part2.tex}
\input{img/summarizeYears/distancesYears_USA_noSoft_I_VA_part1.tex}
\input{img/summarizeYears/distancesYears_USA_noSoft_I_VA_part2.tex}

Tables \ref{table:distancesYears_USA_I_VA_part1} to \ref{table:distancesYears_USA_nonSoft_I_VA_part2} compare investment profiles in the different products per year, as measured by the metric $d_{yy'}$\footnote{This distance measure is slightly different from the one typically proposed by contingency table analysis. For more details   see appendix \ref{metrics}}:
\begin{equation}  \label{eq:distanceRow_year-prod_alt}
d_{yy'}= \sum_p \frac{( I_{yp}/VA_y - I_{y'p}/VA_{y'})^2 }{\frac{1}{n_y}\sum_z I_{zp}/VA_z}
\end{equation}

By comparing the case where all 8 asset-levels are included (tables \ref{table:distancesYears_USA_I_VA_part1} and \ref{table:distancesYears_USA_I_VA_part2}) with the case where Software was excluded (tables \ref{table:distancesYears_USA_nonSoft_I_VA_part1} and \ref{table:distancesYears_USA_nonSoft_I_VA_part2}), it is clear that a large part of the variability between year-profiles comes from Software.

Unsurprisingly, following years differ on average less than years further apart, although there are exceptions to this rule, mostly so in the distances calculated over 7 assets, excluding software \footnote{The metric used to calculate the distances puts all asset-levels on equal footing, even though over the period considered, in between 1970 and 2000, investment in software goes up from $0 \%$ to levels around $1 - 2.5\%$ as share of added value, whereas TraEq + OCon + OMach varies around levels of $15 \%$ of VA. Thus, the fact that Soft has a different profile over time, even though its relative weight is low, has a big effect. Correspondancy analysis picks this effect up in the first dimension ; the efect is not present anymore in dimensions 2 and 3. Excluding Soft is an alternative way to look at the data beyond the evolution of Soft.}. For example, the distance between the 2007-investment profile is quite close to the ones of 1987-1990 and the distance increases before and after. Inspecting the graphical representation of the CA without Soft (\cref{fig:CA_I_VA_NoSoft_TOT_USA}) one can see that this is due to a decrease in 2007 in investing in residential structures. 

%The stacked bar graphs of product x year show for many countries decreasing total investment as a percentage of value added. Since growth has been slow for many countries, this indicates an even slower investment behaviour.

%On the basis of the available graphs, a product per product discussion can be made whether investment in that product has been going up or not, whether there are outliers (example Spain doing a lot of residential construction at some point, Australia doing a lot of Other investing in the 1970s and going down since (creation of agricultural land, deforestation)).

\subsubsection{Correspondance analysis}

Figures \ref{fig:CA_I_VA_TOT_USA}, \ref{fig:CA_I_VA_TOT_USA_dim2-3} and \ref{fig:CA_I_VA_NoSoft_TOT_USA} are a graphical representation of the correspondance analysis of the contingency-matrix products $\times$ years, for the USA, 1977 - 2007. The contributions to the dimensions are :
 \begin{alignat}{7}
 \mathrm{dim}1 =\ & 8.1 * \mathrm{1981} &&+\  7.9 * \mathrm{1980} &&+\  7.2 * \mathrm{1979} &&+\  6.8 * \mathrm{1982} &&+\  6.8 * \mathrm{1978}  \nonumber \\
+ \ & 6.1 * \mathrm{1977} &&+\  5.7 * \mathrm{2004} &&+\  5.5 * \mathrm{2003} &&+\  5.4 * \mathrm{2000} &&+\  5.2 * \mathrm{2002} \nonumber \\
+\ & 5.1 * \mathrm{2001} &&+\  4.9 * \mathrm{2005} &&+\  4.1 * \mathrm{1999} &&+\  3.7 * \mathrm{2006} &&+\  3.3 * \mathrm{2007}  \nonumber \\ 
 \mathrm{dim}2 =\ & 14.5 * \mathrm{2005} &&+\  12.2 * \mathrm{1977} &&+\  11 * \mathrm{1978} &&+\  9.3 * \mathrm{1982} &&+\  9.3 * \mathrm{2004} \\
+\ & 6.3 * \mathrm{2006} &&+\  5.2 * \mathrm{2000} &&+\  4.9 * \mathrm{1979} &&+\  3.4 * \mathrm{1997}  \nonumber \\ 
 \mathrm{dim}3 =\ & 16.2 * \mathrm{2007} &&+\  10.4 * \mathrm{1982} &&+\  7.9 * \mathrm{1995} &&+\  7.9 * \mathrm{1994} &&+\  6.1 * \mathrm{1999} \nonumber \\
+\ & 5.4 * \mathrm{1996} &&+\  5.2 * \mathrm{1977} &&+\  4.7 * \mathrm{1978} &&+\  4.7 * \mathrm{1997} &&+\  4.3 * \mathrm{1998} +\ & 3.9 * \mathrm{1981}  \nonumber
 \end{alignat} 
 \begin{alignat}{4} 
 &\mathrm{dim}1 = 68.8 * \mathrm{Soft} &&+\  11.6 * \mathrm{OCon} &&+\  9.4 * \mathrm{OMach} &&+\  5.5 * \mathrm{IT}  \nonumber \\ 
 &\mathrm{dim}2 = 60.5 * \mathrm{RStruc} &&+\  19.9 * \mathrm{IT} &&+\  10.7 * \mathrm{CT} &&+\  4.9 * \mathrm{OCon}   \\ 
 &\mathrm{dim}3 = 45.9 * \mathrm{TraEq} &&+\  40.7 * \mathrm{OCon} &&+\  4.5 * \mathrm{OMach} &&+\  4.2 * \mathrm{IT}  \nonumber
 \end{alignat}

The first dimension picks up the difference, between assets, in linear trends mentioned before: it opposes the evolution of Soft (contributes \SI{69}{\percent} to the dimension) to that of OCon ($\downarrow$) and OMach ($\downarrow$). There is also a small contribution from IT ($\uparrow$). It is the years 1999-2007, at the end of the period considered, that are associated with the upwards trend in Soft and where investment in OCon and OMach is lower than before. During the beginning of the period, years 1977- 1982, investment in Soft is lowest and in OCon and OMach highest. 

The second dimension opposes investment in residential structures to CT and IT. As mentioned before, RStruc and CT do not show a significant linear trend over the period, so the dimension picks up another type of dynamic: investment in RStruc is associated with 1977-1979 and 2004-2006, which correspponds to two booms (see \cref{fig:I_VA-I_evol-USA-TOT}). It is negatively associated with 1982, year at which investment in RStruc as a share of VA reaches its lowest point. The years 1997 and 2000 correpond to booms in IT and CT investment. The second dimension describes marked highs and lows in RStruc on the one hand, and IT and CT investment on the other hand. 

The third dimension 


\begin{figure}[!h]
    \centering
    \includegraphics[scale=0.8]{img/reducedSystem_Prod-Year/I_VA_TOT_USA.pdf}
    \caption{\label{fig:CA_I_VA_TOT_USA}CA analysis of the reduced system asset x years, USA, 2007, source: KLEMS.}
\end{figure} 

\begin{figure}[!h]
    \centering
    \includegraphics[scale=0.8]{img/reducedSystem_Prod-Year/I_VA_TOT_USA_dim2-3.pdf}
    \caption{\label{fig:CA_I_VA_TOT_USA_dim2-3}CA analysis of the reduced system asset x years, USA, 2007, source: KLEMS.}
\end{figure} 

\begin{figure}[!t]
    \centering
    \includegraphics[scale=0.8]{img/reducedSystem_Prod-Year/I_VA_NoSoft_TOT_USA.pdf}
    \caption{\label{fig:CA_I_VA_NoSoft_TOT_USA}CA analysis of the reduced system asset x years, without considering software assets, USA, 2007, source: KLEMS.}
\end{figure} 

%The CA factor maps show graphically which row and column profiles are (dis)similar. The measure used to evaluate ``distances'' between rows and columns puts all industries and all products on an equal footing, even though some take up a larger share of capital formation than others. This has as effect that even though over the period considered, in between 1970 and 2000, investment in software goes up from $0 \%$ to levels around $1 - 2.5\%$ as share of added value, whereas TraEq + OCon + OMach varies around levels of $15 \%$ of VA, the fact that Soft has a different profile over time, even though its relative weight is low, has a big effect. Differently put, even though NonICT investment has an average level of say $22 \%$ of VA, and ICT investment an average level of $2 \%$, using centered variables eliminates this level effect (its as if growth rates are used as variables). So the assessment is really one of variations, not on whether the bulk of the investment profile is similar or different

%To answer the question whether 2 profiles are similar, it might seem more appropriate to give row profiles their respective weight and not center the variables. This is not evident though, since even though a small portion of a total budget is allocated to a specific product, this product might play an important role in the economy and should not be ``washed away'' by bigger players when evaluating the similarity between profiles. \footnote{One might consider the analogy with doping materials.}



%France, 1970
%eigenvalues: dim1 - 0.84, dim2 - 0.18, dim3 - 0.14
%\begin{alignat}{3}
%&\mathrm{dim}1 = 56.2 * \mathrm{Real\ estate} && +\ 12.5 * \mathrm{Manufacturing} \nonumber \\
%&\mathrm{dim}2 = 52.3 * \mathrm{TransComm} &&+\ 21.7 * \mathrm{Community} &&+\ 11.6 * \mathrm{Agri} \label{eq:dim_industry_FRA1970} \\
%&\mathrm{dim}3 = 37.9 * \mathrm{Community} &&+\ 33.9 * \mathrm{Manufact} &&+\ 13.0 * \mathrm{Agri}\nonumber \\ \nonumber \\
%&\mathrm{dim}1 = 60.1 * \mathrm{RStruc} &&+ \ 17.3 * \mathrm{OMach} &&+ (13.2 * \mathrm{OCon}) \nonumber \\
%&\mathrm{dim}2 = 66.5 * \mathrm{TraEq} &&+\ 22.7 * \mathrm{OCon}   \label{eq:dim_product_FRA1970} \\
%&\mathrm{dim}3 = 50.0 * \mathrm{OMach} &&+\ 30.3 * \mathrm{OCon} \nonumber
%\end{alignat}

%France, 1980
%eigenvalues: dim1 - 0.86, dim2 - 0.18, dim3 - 0.15
%\begin{alignat}{3}
%&\mathrm{dim}1 = 58.8 * \mathrm{Real\ estate} && + (9.8 * \mathrm{Manufacturing}) \nonumber \\
%&\mathrm{dim}2 = 47.8 * \mathrm{Community} &&+\ 12.3 * \mathrm{RE Business} &&+\ 11.6 * \mathrm{Manufact} \label{eq:dim_industry_FRA1970} \\
%&\mathrm{dim}3 = 38.6 * \mathrm{Agri} &&+\ 15.2 * \mathrm{RE Business} &&+\ 15.1 * \mathrm{TransComm}\nonumber \\ \nonumber \\
%&\mathrm{dim}1 = 62.0 * \mathrm{RStruc} &&+ \ 15.5 * \mathrm{OMach} &&+ (13.3 * \mathrm{OCon}) \nonumber \\
%&\mathrm{dim}2 = 45.2 * \mathrm{OCon} &&+\ 20.5 * \mathrm{OMach} &&+ (11.9 * \mathrm{TraEq})   \label{eq:dim_product_FRA1970} \\
%&\mathrm{dim}3 = 32.6 * \mathrm{Other} &&+\ 31.6 * \mathrm{TraEq} &&+\ 25.5 * \mathrm{OMach} \nonumber
%\end{alignat}




\clearpage

\subsection{Industry $\times$ year}
\todo{graphs equivalent of prod graph, GFCF for every industry on 1 graph, divide variables by mean over the period } 
Analysing the reduced system Industry $\times$ year shows during which years the way total investment is spread over the different industry-levels is (dis)similar, and whether different industry-levels displayed a more or less similar investment evolution over the period considered. 

The variable used is $I/VA$. The choice of this vairable is explained in... The metric used in the CA analysis below measures, for the distance between:
\begin{itemize}
\item industry-profiles, invilves comparing for 2 industry-olevels, for eac
\item year-profiles, comes down to comparing shares. The distance between year-profiles is the same as when te varriable $I$ would have been used. The effect of corercting for the total growth of the economy as measured by $VA$ disappears when comparing 2 years. Thus the distance between year-profiles used in the CA analysis are different than the ones in tables ....
\end{itemize}
The first 3 dimensions of the CA analysis are graphically represented in figures \ref{fig:CA-I_VA_USA-GFCF} and \ref{fig:CA-I_VA_USA-GFCF-dim23}. The industries lying close to the origin of the factor map of dimensions 2 and 3 are the ones showing some variability but no clear trend. The first dimensions opposes industries that show either high/low investment levels during the period ... with industries whowing the reverse phenomenon.

The industries close 


\begin{figure}[!ht]
    \centering
    \includegraphics[scale=0.7]{img/reducedSystem_Ind-Year/I_VA_USA_GFCF.pdf}
    \caption{\label{fig:CA-I_VA_USA-GFCF}Correspondance analysis on 8 asset and 11 industry-levels, 1977-2007, USA, dimensions 1 and 2 (source: KLEMS).}
\end{figure} 
\begin{figure}[ht]
    \centering
    \includegraphics[scale=0.8]{img/reducedSystem_Ind-Year/I_VA_USA_GFCF_dim2-3.pdf}
    \caption{\label{fig:CA-I_VA_USA-GFCF-dim23}Correspondance analysis on 11 industry-levels, 1977-2007, USA, dimensions 2 and 3 (source: KLEMS).}
\end{figure}

Contributions to the dimensions, USA, 2010 
\begin{alignat}{7}
 \mathrm{dim}1 =\ & 18.3 * \mathrm{1981} &&+\  14.4 * \mathrm{1982} &&+\  10.3 * \mathrm{1980} &&+\  6.9 * \mathrm{1979} &&+\  5.7 * \mathrm{1999} \nonumber \\
+\ & 5.4 * \mathrm{1978} &&+\  4.5 * \mathrm{2000} &&+\  4.1 * \mathrm{1977} &&+\  3.7 * \mathrm{2004} &&+\  3.5 * \mathrm{2003} &&+\  3.4 * \mathrm{2002}  \nonumber \\ 
 \mathrm{dim}2 =\ & 18.2 * \mathrm{2006} &&+\  14 * \mathrm{2007} &&+\  13.7 * \mathrm{2005} &&+\  8.5 * \mathrm{2004} &&+\  5.5 * \mathrm{2003} &&  \\
+\ & 4.5 * \mathrm{1992} &&+\  3.7 * \mathrm{1991} &&+\  3.6 * \mathrm{1990}  \nonumber \\ 
 \mathrm{dim}3 =\ & 12.5 * \mathrm{2000} &&+\  10.9 * \mathrm{1999} &&+\  10 * \mathrm{1986} &&+\  9.9 * \mathrm{1987} &&+\  8.6 * \mathrm{1997} \nonumber \\
+\ & 6.8 * \mathrm{1998} &&+\  5.3 * \mathrm{1988} &&+\  4.3 * \mathrm{2001} &&+\  3.9 * \mathrm{1985} &&+\  3.9 * \mathrm{1981} \nonumber 
\end{alignat} 
%\begin{flalign}
 \begin{alignat}{4}
 \mathrm{dim}1 =\ & 38.9 * \mathrm{Mining} &&+\  31.5 * \mathrm{RE\ Business} &&+\  9 * \mathrm{Agr} & \nonumber \\ 
+& \  6.6 * \mathrm{Manufact} &&+\  4 * \mathrm{Elec\ Gas} &&+\  3.9 * \mathrm{Finance} & \nonumber \\ 
 \mathrm{dim}2 =\ & 39.2 * \mathrm{Mining} &&+\  19.7 * \mathrm{Manufact} &&+\  17.6 * \mathrm{RE\ Business} & \ \ \  \ \ \  \ \ \  \\ 
+&\  7.9 * \mathrm{Finance} &&+\  5.2 * \mathrm{Trans\ Comm} &&+\  5.1 * \mathrm{Real\ estate}&  \nonumber \\ 
 \mathrm{dim}3 =\ & 31.7 * \mathrm{Real\ estate} &&+\  21.9 * \mathrm{Trans\ Comm} &&+\  20 * \mathrm{RE\ Business}&  \nonumber \\ 
+&\  7.4 * \mathrm{Elec\ Gas} &&+\  7.1 * \mathrm{Construc} &&+\  6.7 * \mathrm{Manufact} &  \nonumber
%\end{flalign}
\end{alignat}
\begin{itemize}
\item The first dimension opposes 4 sectors that peak or know higher investment levels in the beginning of the period (1977 - 1982) - Mining, Agriculture, Manufatcturing and Elec, Gas Water - to RE Business which peaks at the end of the period (2003 - 2005). 
\item The second dimension identifies a dip in investment in the beginning of the nineties. The second dimension involves again the Miining sector. It opposes total investment levels in 2003-2007 to levels in 1990-1992. The increased investment by the mining sector after the year 2000 is due to the shale gas boom. The Mining sector is opposed the the Finance and Manufacturing sector 
\end{itemize}
Figures \ref{fig:I-VA-Product-Mining-USA} to .. show what underlies - to what assets investment was directed more or less - the variations in total investment for the different industries.
\begin{figure}[!ht]
    \centering
    \includegraphics[scale=0.5]{img/plotted_perCountry/I-VA-Product-Mining-USA.pdf}
    \caption{\label{fig:I-VA-Product-Mining-USA}Investment as share of VA by the Mining sector in 8 asset-levels, 1977-2007, USA (source: KLEMS).}
\end{figure} 

%From the distances between industries and comparing with CA analysis of the reduced system prod $\times$ industry over the years 1970, 1980, 1990, 2001 up to 2007, one observes :
%\begin{itemize}
%\item The Real estate sector is over the whole period the main investor in residential strucutures. 
%\item The Transport and communication sector is the main contributer to dimension 2, together with a negative contribution from Manfacturing. TransComm is the main investor in CT, up untill 1990. During the 90's, the previous pattern of dim 2 is overtaken.
%\item
%\item
%\end{itemize}
\input{img/summarizeIndustries/distancesInd_USA_1970.tex}
\input{img/summarizeIndustries/distancesInd_USA_1980.tex}
\input{img/summarizeIndustries/distancesInd_USA_1990.tex}
\input{img/summarizeIndustries/distancesInd_USA_2001.tex}
\input{img/summarizeIndustries/distancesInd_USA_2007.tex}



%Possible industry aggregations :
%\begin{itemize}
%\item Agriculture (AtB), Manufacturing (D), Mining \& Quarrying (C), Electricity, Gas \& Water (E), Construction \& Services (FtQ) (better not lump)
%\item Agriculture (AtB), Mining (C), Manufacturing (D), Electricity, Gas \& Water (E), Construction (F), Sales (GtH), Transport \& Communication (I), Finance (J), Real estate (70), Real estate Business activities (71t74), Community (LtQ)
%\end{itemize}

%Are there years where industries do different investing then in other years ? Do every year all industries do the same investing relative to one another ? 
%Look at product GFCF, ICT, NonICT, for every country.

%\subsection{Investment by industry}

%The content of NACE 1 industry-level K, Real estate activities:

%K	 	Real estate, renting and business activities	   
%KA	 	Real estate, renting and business activities	   
%70	 	Real estate activities
%70.1 = 70.11 + 70.12	 	Real estate activities with own property = Development and selling of real estate + Buying and selling of own real estate
   
%70.2	=70.20 	Letting of own property 

%70.3	 = 70.31	 + 70.32	Real estate activities on a fee or contract basis = Real estate agencies + Management of real estate on a fee or contract basis
	
%71	 	Renting of machinery and equipment without operator and of personal and household goods 
   
%72	 	Computer and related activities 
   
%73	 	Research and development 
   
%74	 	Other business activities 
   

  



%	Real estate, renting and business activities	   
%   		KA	 	Real estate, renting and business activities	   
%      		70	 	Real estate activities	   
%         		70.1	 	Real estate activities with own property	   
%            		70.11	 	Development and selling of real estate 
   
%            		70.12	 	Buying and selling of own real estate 
   


%\subsection{Country x year}

%Is country correlated with year ? Are there countries whos investment profile over the years, GFCF or for a particular product, is different from the average, is for every country the evolution of investment over the years the same ? 

%\subsection{Country x product}

%The stacked bar charts of country x GFCF$/$VA 

%Is there a difference between countries as to in what assets they invest ?

%First answer : no.

%\subsection{Country x industry}

%Industrial structure (as measured by I/K; I/VA)


\clearpage
\subsubsection{Variables and units}

$K$, real capital stock is measured in:
\begin{itemize}
\item 1995 local currency for the KLEMS data 2009 release (update 2011), ISIC 3.1,
\item volume indices with $2005= 100$, for the KLEMS 2012 release, ISIC 4.
\end{itemize}


VA: the following value added related variables are available:
\begin{itemize}
\item $VA$, gross value added at current basic prices, in local currency
\item $VA\_P$, price index for gross value added, with $1995 = 100$
\item $VA\_QI$, volume index for gross value added, with $1995=100$ (does not mean valuation at 1995 prices\footnote{For more information see \cite{timmer_eu_2007}, section 4, page 18.})
\end{itemize}


I, capital formation variables:
\begin{itemize}
\item $I$, nominal gross capital formation in current currency
\item $Iq$, real gross capital formation:
    \begin{itemize}
    \item in 1995 prices, for the 2009 release,
    \item volume indices were $2005 = 100$, for the 2012 release
    \end{itemize}
\end{itemize}

For comparison accross countries,the following measures have units:
\begin{itemize}
\item $I/VA$ : current local currency/current local currency.
\item $(K/VA)*(VA\_P/100)$ : (1995 local currency$/$current local currency) $*$ (current currency$/$ 1995 currency), for the 2009 release.

\end{itemize}








\section{TODO}

\begin{itemize}
    \item KLEMS data what is in the additional aggregations
    \item use depreciation rates for determining shares of investment going to EMA, OMA, etc. 
    \item input IMACLIM: couper les capital input data en 3 morceaux
    \item check Annual macro-economic database, which includes gross fixed capital formation by type of goods at current prices, constant prices and price deflators: \url{http://ec.europa.eu/economy_finance/db_indicators/ameco/zipped_en.htm}
\end{itemize}

\section{Definitions}

\paragraph{Basic price} is the amount a seller receives. It does not include any taxes the buyer pays (those go to the state not to the seller). It includes any subsidy the seller gets for selling (the seller gets the subsidy from the state not from the buyer). Point of view of the seller.
\paragraph{Producer's price} is the amount a producer receives from a buyer. It excludes transportation costs and subsidies (the producer gets those from the state not from the buyer). It includes non-deductible VAT. Point of view of the producer.

\paragraph{Purchaser's price} is the amount payed by the buyer. It includes transportation costs and non-deductible VAT. Point of view of the buyer.
\paragraph{Total investment} = GFCF + financial assets + changes in stocks + education of workers ? + ?
\paragraph{GFCF} is acquisition of fixed assets - disposal of fixed assets, where disposal does not include capital consumption nor losses due to natural disasters.

\paragraph{CFC - Consumptoin of fixed capital} represents the decline in the future benefits of the assets due to their use in the production process. Teh declien in the value of fixed assets of governments and firms, and of dwelling of households. Unforseen obsolescene is not included. How is it measured ? 
\paragraph{Gross operating surplus} is gross output minus the cost of intermediate goods and services and labour costs. It is the capital available to pay debt, invest and pay taxes. 

\appendix

\section{Classifications} \label{appendix:classifications}

Several national and international industry and product classifications are used in the KLEMS base and its construction. Below are listed all encountered systems and correspopndances.

\subsection{Industrial classifications}

\subsubsection{International Standard Industrial Classification (ISIC)}

ISIC is a United Nations industry classification system, first published in 1958. KLEMS uses revisions: 
\begin{itemize}
\item ISIC 3.1 (2002) for the 2009 release with data up to 2007,
\item ISIC 4 (2008) for the 2012 release with data often up to 2014. 
\end{itemize}
Sources:
\begin{itemize}
\item \href{http://unstats.un.org/unsd/cr/registry/regcst.asp?Cl=17}{ISIC rev 3.1, UN}
\item \href{http://unstats.un.org/unsd/cr/registry/isic-4.asp}{ISIC rev 4, UN}
%\item \href{http://unstats.un.org/unsd/cr/registry/isic-4.asp}{ISIC rev 4}
\end{itemize}

\subsubsection{Nomenclature Statistique des Activités Economiques dans la Communauté Européenne (NACE)}

NACE finds its historical origins in NICE (Nomenclature des industries établies dans les Communautés européennes), first used in 1961. Current NACE is derived from and more detailed than ISIC. They have exactly the same items at the highest levels, and NACE is more detailed at lower levels \footnote{\href{http://ec.europa.eu/eurostat/statistics-explained/index.php/NACE_background}{NACE background, eurostat}}.

Sources:
\begin{itemize}
\item \href{http://ec.europa.eu/eurostat/ramon/nomenclatures/index.cfm?TargetUrl=LST_NOM_DTL&StrNom=NACE_1_1&StrLanguageCode=EN&IntPcKey=&StrLayoutCode=EN}{NACE rev 1.1 (2002)}
\item \href{http://ec.europa.eu/eurostat/statistics-explained/index.php/NACE_background}{NACE background, Eurostat, Reference and Management of Nomenclatures (RAMON)}
\item \href{http://unstats.un.org/unsd/cr/registry/regso.asp?Ci=26&Lg=1}{Correspondence between ISIC Rev.3.1 and NACE Rev.1.1}
\end{itemize}  

%\paragraph{Detailing the KLEMS industry-levels} The KLEMS database uses the NACE classification of industries. To obtain more detail on the KLEMS industry-levels, a detailed description of each of them can be obtained

\paragraph{Using the detailed USA BEA capital flow tables to detail investment flows} \label{BEA_capital_flow}To learn more precisely what products the different industries are investing in, one can refer to the capital flow data from the Bureau of Economic Analysis (BEA) on which the construction of the USA KLEMS capital data is based \cite{timmer_eu_2007-1}. The most recent capital flow table dates from 1997. It shows investment in 180 products - new structures, equipment and software -, by 123 private sector industries \footnote{\href{http://www.bea.gov/industry/iedguide.htm\#cfe}{BEA, Industry Economic Accounts Information Guide, Capital Flow (consulted September 2016)}}. The industries are classified according to NAICS 1997 and the products according to the classification in the Unites States' national income and product accounts (NIPA). The concordance between the latter and NAICS 1997 is directly indicated in the table. 

Mapping the NAICS industry-levels of the capital flow table to the KLEMS NACE industry-levels provides %, via the 180 product levels of the capital flow table, 
a more detailed representation of what kind of products the different KLEMS industry-levels invest in. To map from NAICS to NACE, since NACE is basically the same as ISIC, it is easiest to refer to the \href{http://unstats.un.org/unsd/cr/registry/regso.asp?Ci=29&Lg=1&Co=&T=0&p=42}{concordance between NAICS 2007 and ISIC 3.1}\footnote{Correspondancy tables between NAICS 1997 and 2002 or 2007 and between NAICS 2002 and ISIC 3.1 are provided by the US Census Bureau: \href{https://www.census.gov/eos/www/naics/concordances/concordances.html}{Concordances, NAICS}}.

LtQ groups maps approximately to the capital flow table industry codes 5615, 561A, 5620, 6100, 6210, 6220, 6230, 6240, 71A0, 7130, 8120, 813A,	813B. The mapping is only approximate because there is often only a partial correspondance between categories. The level 71t74, Real estate business, correponds approximately to NAICS 5112, 5321, 5322, 5411, 5324, 5412, 5413, 5418, 5614, 5616. 


\subsubsection{North American Industry Classification System (NAICS)}

NAICS classifies business entities by type of economic activity (process of production) in Canada, Mexico, and the United States of America. The first version was released in 1997.

The KLEMS NAICS data are based on the 1997 revision.

Sources:
\begin{itemize}
\item \href{http://www.census.gov/eos/www/naics/}{NAICS, US Census Bureau}
\item \href{http://www.census.gov/eos/www/naics/concordances/concordances.html}{Concordances NAICS with other classifications, US Census Bureau}

\end{itemize}

\subsection{Product classifications}

\subsubsection{CPA}

Sources:

\begin{itemize}
\item \href{http://ec.europa.eu/eurostat/ramon/nomenclatures/index.cfm?TargetUrl=LST_NOM_DTL&StrNom=CPA_2008&StrLanguageCode=EN&IntPcKey=&StrLayoutCode=HIERARCHIC}{CPA, Eurostat, Reference And Management Of Nomenclatures (RAMON)}
\end{itemize}

\subsubsection{}


\section{Missing values} \label{AppMV}
To get an overview of the missing values amongst the set of capital input data, encompassing the variables \{Nominal gross fixed capital formation, Real gross fixed capital formation, Gross fixed capital formation price index, Real fixed capital stock, Capital compensation, Consumption of fixed capital \} and all product and industry levels provided, a strategy of counting missing variables was used. Firstly, the amount of missing values was determined per year and country (summing over variables, products and industries), per year and variable (summing over countries, products and industries) and per year and industry-level (summing over countries, variables and products). The results can be found respectively in tables \ref{nanCountries}, \ref{nanVars}, \ref{nanIndustries_0_20} and \ref{nanIndustries_21}.

% latex table generated in R 3.3.0 by xtable 1.8-2 package
% Fri Jun 10 14:20:50 2016
\begin{table}[ht]
\centering
\begingroup\tiny
\begin{tabular}{rrrrrrrrrrrrr}
  \hline
 & AUS & AUT & CZE & DNK & ESP & FIN & GER & ITA & NLD & SWE & UK & USA \\ 
  \hline
X1970 & 180 & 2640 & 2640 & 144 & 246 & 287 & 2640 & 110 & 168 & 2640 & 200 & 164 \\ 
  X1971 & 180 & 2640 & 2640 & 144 & 246 & 282 & 2640 & 110 & 168 & 2640 & 200 & 164 \\ 
  X1972 & 178 & 2640 & 2640 & 144 & 246 & 272 & 2640 & 110 & 162 & 2640 & 200 & 164 \\ 
  X1973 & 178 & 2640 & 2640 & 144 & 246 & 263 & 2640 & 110 & 154 & 2640 & 200 & 164 \\ 
  X1974 & 178 & 2640 & 2640 & 144 & 246 & 259 & 2640 & 110 & 153 & 2640 & 200 & 164 \\ 
  X1975 & 174 & 2640 & 2640 & 144 & 246 & 254 & 2640 & 110 & 153 & 2640 & 200 & 164 \\ 
  X1976 & 172 & 147 & 2640 & 144 & 246 & 252 & 2640 & 110 & 146 & 2640 & 200 & 164 \\ 
  X1977 & 170 & 147 & 2640 & 144 & 246 & 250 & 2640 & 110 & 139 & 2640 & 200 & 164 \\ 
  X1978 & 169 & 145 & 2640 & 144 & 246 & 243 & 2640 & 110 & 139 & 2640 & 200 & 164 \\ 
  X1979 & 169 & 145 & 2640 & 144 & 246 & 239 & 2640 & 110 & 138 & 2640 & 200 & 164 \\ 
  X1980 & 169 & 145 & 2640 & 144 & 246 & 235 & 2640 & 110 & 137 & 2640 & 200 & 164 \\ 
  X1981 & 169 & 145 & 2640 & 144 & 246 & 231 & 2640 & 110 & 136 & 2640 & 200 & 164 \\ 
  X1982 & 169 & 145 & 2640 & 144 & 246 & 231 & 2640 & 110 & 136 & 2640 & 200 & 164 \\ 
  X1983 & 169 & 145 & 2640 & 144 & 246 & 231 & 2640 & 110 & 135 & 2640 & 200 & 164 \\ 
  X1984 & 169 & 145 & 2640 & 144 & 246 & 229 & 2640 & 110 & 134 & 2640 & 200 & 164 \\ 
  X1985 & 169 & 145 & 2640 & 144 & 246 & 227 & 2640 & 110 & 134 & 2640 & 200 & 164 \\ 
  X1986 & 169 & 145 & 2640 & 144 & 246 & 225 & 2640 & 110 & 135 & 2640 & 200 & 164 \\ 
  X1987 & 169 & 145 & 2640 & 144 & 246 & 223 & 2640 & 110 & 146 & 2640 & 200 & 164 \\ 
  X1988 & 169 & 145 & 2640 & 144 & 246 & 223 & 2640 & 110 & 139 & 2640 & 200 & 164 \\ 
  X1989 & 169 & 145 & 2640 & 144 & 246 & 223 & 2640 & 110 & 142 & 2640 & 200 & 164 \\ 
  X1990 & 169 & 145 & 2640 & 144 & 246 & 221 & 2640 & 110 & 141 & 2640 & 200 & 164 \\ 
  X1991 & 169 & 145 & 2640 & 144 & 246 & 221 & 111 & 110 & 139 & 2640 & 200 & 164 \\ 
  X1992 & 169 & 145 & 2640 & 144 & 246 & 221 & 111 & 110 & 146 & 2640 & 200 & 164 \\ 
  X1993 & 169 & 145 & 2640 & 144 & 246 & 221 & 111 & 110 & 146 & 381 & 200 & 164 \\ 
  X1994 & 169 & 145 & 2640 & 144 & 246 & 223 & 111 & 110 & 145 & 381 & 200 & 164 \\ 
  X1995 & 169 & 145 & 102 & 144 & 246 & 226 & 111 & 110 & 146 & 380 & 200 & 164 \\ 
  X1996 & 169 & 145 & 98 & 144 & 246 & 226 & 111 & 110 & 149 & 381 & 200 & 164 \\ 
  X1997 & 169 & 145 & 87 & 144 & 246 & 226 & 111 & 110 & 141 & 380 & 200 & 164 \\ 
  X1998 & 169 & 145 & 83 & 144 & 246 & 225 & 111 & 110 & 144 & 380 & 200 & 164 \\ 
  X1999 & 169 & 145 & 83 & 144 & 246 & 225 & 111 & 110 & 138 & 380 & 200 & 164 \\ 
  X2000 & 169 & 145 & 86 & 144 & 246 & 225 & 111 & 110 & 143 & 380 & 200 & 164 \\ 
  X2001 & 169 & 145 & 82 & 144 & 246 & 223 & 111 & 110 & 144 & 380 & 201 & 164 \\ 
  X2002 & 169 & 145 & 85 & 144 & 246 & 226 & 111 & 110 & 143 & 376 & 201 & 164 \\ 
  X2003 & 169 & 145 & 88 & 144 & 246 & 223 & 111 & 110 & 147 & 376 & 201 & 164 \\ 
  X2004 & 169 & 145 & 85 & 144 & 246 & 226 & 111 & 110 & 142 & 376 & 200 & 164 \\ 
  X2005 & 169 & 145 & 89 & 144 & 246 & 227 & 111 & 110 & 149 & 376 & 201 & 164 \\ 
  X2006 & 169 & 145 & 93 & 144 & 246 & 226 & 111 & 110 & 151 & 376 & 200 & 164 \\ 
  X2007 & 169 & 145 & 87 & 144 & 246 & 227 & 111 & 110 & 151 & 376 & 200 & 164 \\ 
   \hline
\end{tabular}
\endgroup
\caption{Missing variables by country}
\label{nanCountries}
\end{table}

% latex table generated in R 3.3.0 by xtable 1.8-2 package
% Thu Jun 23 10:19:00 2016
\begin{table}[ht]
\centering
\begingroup\tiny
\begin{tabular}{rrrrrrrrrrrrr}
  \hline
 & AUS & AUT & CZE & DNK & ESP & FIN & GER & ITA & NLD & SWE & UK & USA \\ 
  \hline
X1970 & 114 & 2574 & 2574 & 133 & 180 & 255 & 2574 & 99 & 102 & 2574 & 134 & 98 \\ 
  X1971 & 114 & 2574 & 2574 & 133 & 180 & 250 & 2574 & 99 & 102 & 2574 & 134 & 98 \\ 
  X1972 & 112 & 2574 & 2574 & 133 & 180 & 240 & 2574 & 99 & 96 & 2574 & 134 & 98 \\ 
  X1973 & 112 & 2574 & 2574 & 133 & 180 & 231 & 2574 & 99 & 88 & 2574 & 134 & 98 \\ 
  X1974 & 112 & 2574 & 2574 & 133 & 180 & 227 & 2574 & 99 & 87 & 2574 & 134 & 98 \\ 
  X1975 & 108 & 2574 & 2574 & 133 & 180 & 222 & 2574 & 99 & 87 & 2574 & 134 & 98 \\ 
  X1976 & 106 & 136 & 2574 & 133 & 180 & 220 & 2574 & 99 & 80 & 2574 & 134 & 98 \\ 
  X1977 & 104 & 136 & 2574 & 133 & 180 & 218 & 2574 & 99 & 73 & 2574 & 134 & 98 \\ 
  X1978 & 103 & 134 & 2574 & 133 & 180 & 211 & 2574 & 99 & 73 & 2574 & 134 & 98 \\ 
  X1979 & 103 & 134 & 2574 & 133 & 180 & 207 & 2574 & 99 & 72 & 2574 & 134 & 98 \\ 
  X1980 & 103 & 134 & 2574 & 133 & 180 & 203 & 2574 & 99 & 71 & 2574 & 134 & 98 \\ 
  X1981 & 103 & 134 & 2574 & 133 & 180 & 199 & 2574 & 99 & 70 & 2574 & 134 & 98 \\ 
  X1982 & 103 & 134 & 2574 & 133 & 180 & 199 & 2574 & 99 & 70 & 2574 & 134 & 98 \\ 
  X1983 & 103 & 134 & 2574 & 133 & 180 & 199 & 2574 & 99 & 69 & 2574 & 134 & 98 \\ 
  X1984 & 103 & 134 & 2574 & 133 & 180 & 197 & 2574 & 99 & 68 & 2574 & 134 & 98 \\ 
  X1985 & 103 & 134 & 2574 & 133 & 180 & 195 & 2574 & 99 & 68 & 2574 & 134 & 98 \\ 
  X1986 & 103 & 134 & 2574 & 133 & 180 & 193 & 2574 & 99 & 69 & 2574 & 134 & 98 \\ 
  X1987 & 103 & 134 & 2574 & 133 & 180 & 191 & 2574 & 99 & 80 & 2574 & 134 & 98 \\ 
  X1988 & 103 & 134 & 2574 & 133 & 180 & 191 & 2574 & 99 & 73 & 2574 & 134 & 98 \\ 
  X1989 & 103 & 134 & 2574 & 133 & 180 & 191 & 2574 & 99 & 76 & 2574 & 134 & 98 \\ 
  X1990 & 103 & 134 & 2574 & 133 & 180 & 189 & 2574 & 99 & 75 & 2574 & 134 & 98 \\ 
  X1991 & 103 & 134 & 2574 & 133 & 180 & 189 & 100 & 99 & 73 & 2574 & 134 & 98 \\ 
  X1992 & 103 & 134 & 2574 & 133 & 180 & 189 & 100 & 99 & 80 & 2574 & 134 & 98 \\ 
  X1993 & 103 & 134 & 2574 & 133 & 180 & 189 & 100 & 99 & 80 & 335 & 134 & 98 \\ 
  X1994 & 103 & 134 & 2574 & 133 & 180 & 191 & 100 & 99 & 79 & 335 & 134 & 98 \\ 
  X1995 & 103 & 134 & 91 & 133 & 180 & 194 & 100 & 99 & 80 & 334 & 134 & 98 \\ 
  X1996 & 103 & 134 & 87 & 133 & 180 & 194 & 100 & 99 & 83 & 335 & 134 & 98 \\ 
  X1997 & 103 & 134 & 76 & 133 & 180 & 194 & 100 & 99 & 75 & 334 & 134 & 98 \\ 
  X1998 & 103 & 134 & 72 & 133 & 180 & 193 & 100 & 99 & 78 & 334 & 134 & 98 \\ 
  X1999 & 103 & 134 & 72 & 133 & 180 & 193 & 100 & 99 & 72 & 334 & 134 & 98 \\ 
  X2000 & 103 & 134 & 75 & 133 & 180 & 193 & 100 & 99 & 77 & 334 & 134 & 98 \\ 
  X2001 & 103 & 134 & 71 & 133 & 180 & 191 & 100 & 99 & 78 & 334 & 135 & 98 \\ 
  X2002 & 103 & 134 & 74 & 133 & 180 & 194 & 100 & 99 & 77 & 330 & 135 & 98 \\ 
  X2003 & 103 & 134 & 77 & 133 & 180 & 191 & 100 & 99 & 81 & 330 & 135 & 98 \\ 
  X2004 & 103 & 134 & 74 & 133 & 180 & 194 & 100 & 99 & 76 & 330 & 134 & 98 \\ 
  X2005 & 103 & 134 & 78 & 133 & 180 & 195 & 100 & 99 & 83 & 330 & 135 & 98 \\ 
  X2006 & 103 & 134 & 82 & 133 & 180 & 194 & 100 & 99 & 85 & 330 & 134 & 98 \\ 
  X2007 & 103 & 134 & 76 & 133 & 180 & 195 & 100 & 99 & 85 & 330 & 134 & 98 \\ 
   \hline
\end{tabular}
\endgroup
\caption{Missing values by country,not including industry levels P}
\label{nanCountries_noP}
\end{table}

% latex table generated in R 3.3.0 by xtable 1.8-2 package
% Thu Jun 23 10:04:50 2016
\begin{table}[ht]
\centering
\begingroup\tiny
\begin{tabular}{rrrrrrrrrrrrr}
  \hline
 & AUS & AUT & CZE & DNK & ESP & FIN & GER & ITA & NLD & SWE & UK & USA \\ 
  \hline
X1970 & 48 & 2508 & 2508 & 67 & 114 & 189 & 2508 & 33 & 36 & 2508 & 68 & 32 \\ 
  X1971 & 48 & 2508 & 2508 & 67 & 114 & 184 & 2508 & 33 & 36 & 2508 & 68 & 32 \\ 
  X1972 & 46 & 2508 & 2508 & 67 & 114 & 174 & 2508 & 33 & 30 & 2508 & 68 & 32 \\ 
  X1973 & 46 & 2508 & 2508 & 67 & 114 & 165 & 2508 & 33 & 22 & 2508 & 68 & 32 \\ 
  X1974 & 46 & 2508 & 2508 & 67 & 114 & 161 & 2508 & 33 & 21 & 2508 & 68 & 32 \\ 
  X1975 & 42 & 2508 & 2508 & 67 & 114 & 156 & 2508 & 33 & 21 & 2508 & 68 & 32 \\ 
  X1976 & 40 & 70 & 2508 & 67 & 114 & 154 & 2508 & 33 & 14 & 2508 & 68 & 32 \\ 
  X1977 & 38 & 70 & 2508 & 67 & 114 & 152 & 2508 & 33 & 7 & 2508 & 68 & 32 \\ 
  X1978 & 37 & 68 & 2508 & 67 & 114 & 145 & 2508 & 33 & 7 & 2508 & 68 & 32 \\ 
  X1979 & 37 & 68 & 2508 & 67 & 114 & 141 & 2508 & 33 & 6 & 2508 & 68 & 32 \\ 
  X1980 & 37 & 68 & 2508 & 67 & 114 & 137 & 2508 & 33 & 5 & 2508 & 68 & 32 \\ 
  X1981 & 37 & 68 & 2508 & 67 & 114 & 133 & 2508 & 33 & 4 & 2508 & 68 & 32 \\ 
  X1982 & 37 & 68 & 2508 & 67 & 114 & 133 & 2508 & 33 & 4 & 2508 & 68 & 32 \\ 
  X1983 & 37 & 68 & 2508 & 67 & 114 & 133 & 2508 & 33 & 3 & 2508 & 68 & 32 \\ 
  X1984 & 37 & 68 & 2508 & 67 & 114 & 131 & 2508 & 33 & 2 & 2508 & 68 & 32 \\ 
  X1985 & 37 & 68 & 2508 & 67 & 114 & 129 & 2508 & 33 & 2 & 2508 & 68 & 32 \\ 
  X1986 & 37 & 68 & 2508 & 67 & 114 & 127 & 2508 & 33 & 3 & 2508 & 68 & 32 \\ 
  X1987 & 37 & 68 & 2508 & 67 & 114 & 125 & 2508 & 33 & 14 & 2508 & 68 & 32 \\ 
  X1988 & 37 & 68 & 2508 & 67 & 114 & 125 & 2508 & 33 & 7 & 2508 & 68 & 32 \\ 
  X1989 & 37 & 68 & 2508 & 67 & 114 & 125 & 2508 & 33 & 10 & 2508 & 68 & 32 \\ 
  X1990 & 37 & 68 & 2508 & 67 & 114 & 123 & 2508 & 33 & 9 & 2508 & 68 & 32 \\ 
  X1991 & 37 & 68 & 2508 & 67 & 114 & 123 & 34 & 33 & 7 & 2508 & 68 & 32 \\ 
  X1992 & 37 & 68 & 2508 & 67 & 114 & 123 & 34 & 33 & 14 & 2508 & 68 & 32 \\ 
  X1993 & 37 & 68 & 2508 & 67 & 114 & 123 & 34 & 33 & 14 & 269 & 68 & 32 \\ 
  X1994 & 37 & 68 & 2508 & 67 & 114 & 125 & 34 & 33 & 13 & 269 & 68 & 32 \\ 
  X1995 & 37 & 68 & 25 & 67 & 114 & 128 & 34 & 33 & 14 & 268 & 68 & 32 \\ 
  X1996 & 37 & 68 & 21 & 67 & 114 & 128 & 34 & 33 & 17 & 269 & 68 & 32 \\ 
  X1997 & 37 & 68 & 10 & 67 & 114 & 128 & 34 & 33 & 9 & 268 & 68 & 32 \\ 
  X1998 & 37 & 68 & 6 & 67 & 114 & 127 & 34 & 33 & 12 & 268 & 68 & 32 \\ 
  X1999 & 37 & 68 & 6 & 67 & 114 & 127 & 34 & 33 & 6 & 268 & 68 & 32 \\ 
  X2000 & 37 & 68 & 9 & 67 & 114 & 127 & 34 & 33 & 11 & 268 & 68 & 32 \\ 
  X2001 & 37 & 68 & 5 & 67 & 114 & 125 & 34 & 33 & 12 & 268 & 69 & 32 \\ 
  X2002 & 37 & 68 & 8 & 67 & 114 & 128 & 34 & 33 & 11 & 264 & 69 & 32 \\ 
  X2003 & 37 & 68 & 11 & 67 & 114 & 125 & 34 & 33 & 15 & 264 & 69 & 32 \\ 
  X2004 & 37 & 68 & 8 & 67 & 114 & 128 & 34 & 33 & 10 & 264 & 68 & 32 \\ 
  X2005 & 37 & 68 & 12 & 67 & 114 & 129 & 34 & 33 & 17 & 264 & 69 & 32 \\ 
  X2006 & 37 & 68 & 16 & 67 & 114 & 128 & 34 & 33 & 19 & 264 & 68 & 32 \\ 
  X2007 & 37 & 68 & 10 & 67 & 114 & 129 & 34 & 33 & 19 & 264 & 68 & 32 \\ 
   \hline
\end{tabular}
\endgroup
\caption{Missing values by country,not including industry levels P Q}
\label{nanCountries_noPQ} 
\end{table}

% latex table generated in R 3.3.0 by xtable 1.8-2 package
% Thu Jun 23 11:56:45 2016
\begin{table}[ht]
\centering
\begingroup\tiny
\begin{tabular}{rrrrrrrrrrrrr}
  \hline
 & AUS & AUT & CZE & DNK & ESP & FIN & GER & ITA & NLD & SWE & UK & USA \\ 
  \hline
X1970 & 179 & 2574 & 2574 & 143 & 243 & 256 & 2574 & 109 & 168 & 2574 & 199 & 164 \\ 
  X1971 & 179 & 2574 & 2574 & 143 & 243 & 251 & 2574 & 109 & 168 & 2574 & 199 & 164 \\ 
  X1972 & 177 & 2574 & 2574 & 143 & 243 & 241 & 2574 & 109 & 162 & 2574 & 199 & 164 \\ 
  X1973 & 177 & 2574 & 2574 & 143 & 243 & 232 & 2574 & 109 & 154 & 2574 & 199 & 164 \\ 
  X1974 & 177 & 2574 & 2574 & 143 & 243 & 228 & 2574 & 109 & 153 & 2574 & 199 & 164 \\ 
  X1975 & 173 & 2574 & 2574 & 143 & 243 & 223 & 2574 & 109 & 153 & 2574 & 199 & 164 \\ 
  X1976 & 171 & 146 & 2574 & 143 & 243 & 221 & 2574 & 109 & 146 & 2574 & 199 & 164 \\ 
  X1977 & 169 & 146 & 2574 & 143 & 243 & 219 & 2574 & 109 & 139 & 2574 & 199 & 164 \\ 
  X1978 & 168 & 144 & 2574 & 143 & 243 & 212 & 2574 & 109 & 139 & 2574 & 199 & 164 \\ 
  X1979 & 168 & 144 & 2574 & 143 & 243 & 208 & 2574 & 109 & 138 & 2574 & 199 & 164 \\ 
  X1980 & 168 & 144 & 2574 & 143 & 243 & 204 & 2574 & 109 & 137 & 2574 & 199 & 164 \\ 
  X1981 & 168 & 144 & 2574 & 143 & 243 & 200 & 2574 & 109 & 136 & 2574 & 199 & 164 \\ 
  X1982 & 168 & 144 & 2574 & 143 & 243 & 200 & 2574 & 109 & 136 & 2574 & 199 & 164 \\ 
  X1983 & 168 & 144 & 2574 & 143 & 243 & 200 & 2574 & 109 & 135 & 2574 & 199 & 164 \\ 
  X1984 & 168 & 144 & 2574 & 143 & 243 & 198 & 2574 & 109 & 134 & 2574 & 199 & 164 \\ 
  X1985 & 168 & 144 & 2574 & 143 & 243 & 196 & 2574 & 109 & 134 & 2574 & 199 & 164 \\ 
  X1986 & 168 & 144 & 2574 & 143 & 243 & 194 & 2574 & 109 & 135 & 2574 & 199 & 164 \\ 
  X1987 & 168 & 144 & 2574 & 143 & 243 & 192 & 2574 & 109 & 146 & 2574 & 199 & 164 \\ 
  X1988 & 168 & 144 & 2574 & 143 & 243 & 192 & 2574 & 109 & 139 & 2574 & 199 & 164 \\ 
  X1989 & 168 & 144 & 2574 & 143 & 243 & 192 & 2574 & 109 & 142 & 2574 & 199 & 164 \\ 
  X1990 & 168 & 144 & 2574 & 143 & 243 & 190 & 2574 & 109 & 141 & 2574 & 199 & 164 \\ 
  X1991 & 168 & 144 & 2574 & 143 & 243 & 190 & 110 & 109 & 139 & 2574 & 199 & 164 \\ 
  X1992 & 168 & 144 & 2574 & 143 & 243 & 190 & 110 & 109 & 146 & 2574 & 199 & 164 \\ 
  X1993 & 168 & 144 & 2574 & 143 & 243 & 190 & 110 & 109 & 146 & 374 & 199 & 164 \\ 
  X1994 & 168 & 144 & 2574 & 143 & 243 & 192 & 110 & 109 & 145 & 374 & 199 & 164 \\ 
  X1995 & 168 & 144 & 102 & 143 & 243 & 195 & 110 & 109 & 146 & 373 & 199 & 164 \\ 
  X1996 & 168 & 144 & 98 & 143 & 243 & 195 & 110 & 109 & 149 & 374 & 199 & 164 \\ 
  X1997 & 168 & 144 & 87 & 143 & 243 & 195 & 110 & 109 & 141 & 373 & 199 & 164 \\ 
  X1998 & 168 & 144 & 83 & 143 & 243 & 194 & 110 & 109 & 144 & 373 & 199 & 164 \\ 
  X1999 & 168 & 144 & 83 & 143 & 243 & 194 & 110 & 109 & 138 & 373 & 199 & 164 \\ 
  X2000 & 168 & 144 & 86 & 143 & 243 & 194 & 110 & 109 & 143 & 373 & 199 & 164 \\ 
  X2001 & 168 & 144 & 82 & 143 & 243 & 192 & 110 & 109 & 144 & 373 & 200 & 164 \\ 
  X2002 & 168 & 144 & 85 & 143 & 243 & 195 & 110 & 109 & 143 & 369 & 200 & 164 \\ 
  X2003 & 168 & 144 & 88 & 143 & 243 & 192 & 110 & 109 & 147 & 369 & 200 & 164 \\ 
  X2004 & 168 & 144 & 85 & 143 & 243 & 195 & 110 & 109 & 142 & 369 & 199 & 164 \\ 
  X2005 & 168 & 144 & 89 & 143 & 243 & 196 & 110 & 109 & 149 & 369 & 200 & 164 \\ 
  X2006 & 168 & 144 & 93 & 143 & 243 & 195 & 110 & 109 & 151 & 369 & 199 & 164 \\ 
  X2007 & 168 & 144 & 87 & 143 & 243 & 196 & 110 & 109 & 151 & 369 & 199 & 164 \\ 
   \hline
\end{tabular}
\endgroup
\caption{Missing values by country,not including aggregated industry level LtQ} 
\label{nanCountries_noLtQ}
\end{table}


\begin{landscape}
% latex table generated in R 3.3.0 by xtable 1.8-2 package
% Mon Jun 13 10:48:07 2016
\begin{table}[ht]
\centering
\begingroup\tiny
\begin{tabular}{rrrrrrrrrrrrrrrrrrrrr}
  \hline
 & TOT & AtB & C & D & 15t16 & 17t19 & 20 & 21t22 & 23t25 & 23 & 24 & 25 & 26 & 27t28 & 29 & 30t33 & 34t35 & 36t37 & E & F \\ 
  \hline
X1970 & 298 & 280 & 281 & 275 & 276 & 281 & 281 & 275 & 276 & 282 & 276 & 282 & 281 & 276 & 275 & 276 & 276 & 281 & 280 & 284 \\ 
  X1971 & 298 & 280 & 280 & 275 & 276 & 282 & 278 & 275 & 276 & 282 & 276 & 282 & 278 & 276 & 275 & 276 & 277 & 281 & 280 & 284 \\ 
  X1972 & 298 & 280 & 280 & 275 & 276 & 281 & 278 & 275 & 276 & 281 & 276 & 282 & 278 & 276 & 275 & 275 & 276 & 281 & 279 & 280 \\ 
  X1973 & 298 & 280 & 280 & 275 & 276 & 279 & 278 & 275 & 276 & 277 & 276 & 279 & 278 & 276 & 275 & 275 & 276 & 278 & 279 & 279 \\ 
  X1974 & 298 & 280 & 280 & 275 & 276 & 279 & 278 & 275 & 276 & 277 & 276 & 278 & 278 & 276 & 275 & 275 & 276 & 278 & 279 & 279 \\ 
  X1975 & 298 & 277 & 278 & 275 & 275 & 277 & 279 & 275 & 275 & 278 & 275 & 281 & 279 & 275 & 275 & 275 & 277 & 279 & 279 & 280 \\ 
  X1976 & 232 & 211 & 214 & 211 & 211 & 213 & 213 & 211 & 211 & 216 & 211 & 217 & 213 & 211 & 211 & 211 & 213 & 215 & 215 & 215 \\ 
  X1977 & 232 & 210 & 212 & 211 & 211 & 213 & 213 & 211 & 211 & 216 & 211 & 215 & 211 & 211 & 211 & 211 & 213 & 213 & 215 & 215 \\ 
  X1978 & 232 & 210 & 211 & 211 & 211 & 213 & 212 & 211 & 211 & 215 & 211 & 215 & 211 & 211 & 211 & 211 & 211 & 211 & 215 & 215 \\ 
  X1979 & 232 & 208 & 211 & 211 & 211 & 213 & 212 & 211 & 211 & 214 & 211 & 213 & 211 & 211 & 211 & 211 & 211 & 213 & 215 & 213 \\ 
  X1980 & 232 & 207 & 211 & 211 & 211 & 213 & 211 & 211 & 211 & 213 & 211 & 211 & 211 & 211 & 211 & 211 & 211 & 213 & 215 & 213 \\ 
  X1981 & 232 & 207 & 211 & 211 & 211 & 213 & 211 & 211 & 211 & 212 & 211 & 211 & 211 & 211 & 211 & 211 & 211 & 213 & 215 & 213 \\ 
  X1982 & 232 & 207 & 211 & 211 & 211 & 213 & 211 & 211 & 211 & 211 & 211 & 211 & 211 & 211 & 211 & 211 & 211 & 213 & 215 & 213 \\ 
  X1983 & 232 & 207 & 211 & 211 & 211 & 213 & 211 & 211 & 211 & 211 & 211 & 211 & 211 & 211 & 211 & 211 & 211 & 213 & 214 & 213 \\ 
  X1984 & 232 & 207 & 211 & 211 & 211 & 213 & 211 & 211 & 211 & 211 & 211 & 211 & 211 & 211 & 211 & 211 & 211 & 211 & 213 & 213 \\ 
  X1985 & 232 & 207 & 211 & 211 & 211 & 213 & 211 & 211 & 211 & 211 & 211 & 211 & 211 & 211 & 211 & 211 & 211 & 211 & 213 & 213 \\ 
  X1986 & 232 & 207 & 211 & 211 & 211 & 211 & 211 & 211 & 211 & 211 & 211 & 211 & 211 & 211 & 211 & 211 & 212 & 211 & 213 & 213 \\ 
  X1987 & 232 & 207 & 210 & 211 & 211 & 212 & 212 & 212 & 211 & 212 & 211 & 212 & 211 & 211 & 212 & 211 & 212 & 212 & 213 & 213 \\ 
  X1988 & 232 & 207 & 210 & 211 & 211 & 212 & 212 & 211 & 211 & 212 & 211 & 211 & 211 & 211 & 211 & 212 & 212 & 211 & 213 & 213 \\ 
  X1989 & 232 & 207 & 210 & 211 & 211 & 212 & 212 & 211 & 211 & 212 & 211 & 212 & 211 & 211 & 211 & 212 & 212 & 212 & 213 & 213 \\ 
  X1990 & 232 & 208 & 210 & 211 & 211 & 212 & 212 & 211 & 211 & 211 & 211 & 211 & 211 & 211 & 211 & 212 & 212 & 211 & 213 & 213 \\ 
  X1991 & 166 & 143 & 144 & 146 & 146 & 147 & 147 & 146 & 146 & 147 & 146 & 146 & 146 & 146 & 146 & 146 & 147 & 146 & 148 & 148 \\ 
  X1992 & 166 & 142 & 145 & 146 & 146 & 147 & 147 & 147 & 147 & 147 & 147 & 147 & 146 & 146 & 146 & 146 & 147 & 146 & 148 & 148 \\ 
  X1993 & 106 & 84 & 87 & 87 & 87 & 88 & 88 & 88 & 87 & 89 & 87 & 87 & 87 & 88 & 88 & 88 & 88 & 88 & 89 & 89 \\ 
  X1994 & 106 & 84 & 86 & 87 & 87 & 88 & 88 & 87 & 87 & 88 & 87 & 88 & 87 & 87 & 88 & 88 & 88 & 88 & 89 & 89 \\ 
  X1995 & 40 & 19 & 20 & 21 & 22 & 23 & 23 & 22 & 22 & 25 & 22 & 23 & 22 & 22 & 23 & 23 & 23 & 23 & 23 & 23 \\ 
  X1996 & 40 & 18 & 20 & 21 & 23 & 23 & 23 & 23 & 23 & 24 & 23 & 23 & 21 & 23 & 23 & 23 & 23 & 23 & 23 & 23 \\ 
  X1997 & 40 & 18 & 20 & 21 & 21 & 23 & 23 & 22 & 21 & 23 & 21 & 23 & 21 & 21 & 21 & 22 & 21 & 22 & 23 & 23 \\ 
  X1998 & 40 & 18 & 20 & 21 & 21 & 22 & 23 & 22 & 21 & 24 & 22 & 22 & 21 & 21 & 21 & 22 & 22 & 22 & 24 & 23 \\ 
  X1999 & 40 & 18 & 19 & 21 & 21 & 22 & 22 & 22 & 21 & 24 & 21 & 22 & 21 & 21 & 21 & 22 & 22 & 21 & 24 & 23 \\ 
  X2000 & 40 & 18 & 19 & 21 & 21 & 22 & 22 & 22 & 23 & 24 & 23 & 23 & 22 & 21 & 21 & 22 & 23 & 21 & 23 & 23 \\ 
  X2001 & 40 & 18 & 19 & 21 & 21 & 22 & 23 & 22 & 21 & 25 & 21 & 22 & 22 & 21 & 22 & 22 & 22 & 22 & 24 & 23 \\ 
  X2002 & 40 & 18 & 19 & 21 & 21 & 22 & 23 & 23 & 22 & 26 & 23 & 22 & 21 & 21 & 22 & 22 & 22 & 21 & 23 & 23 \\ 
  X2003 & 40 & 18 & 20 & 21 & 21 & 22 & 23 & 23 & 22 & 26 & 23 & 22 & 23 & 21 & 22 & 23 & 22 & 21 & 23 & 23 \\ 
  X2004 & 40 & 18 & 21 & 21 & 21 & 23 & 22 & 21 & 22 & 24 & 23 & 22 & 21 & 21 & 21 & 23 & 22 & 21 & 24 & 23 \\ 
  X2005 & 40 & 18 & 21 & 21 & 21 & 22 & 22 & 23 & 23 & 26 & 23 & 23 & 22 & 22 & 22 & 23 & 23 & 22 & 25 & 23 \\ 
  X2006 & 40 & 18 & 22 & 21 & 21 & 23 & 22 & 23 & 22 & 24 & 22 & 22 & 23 & 23 & 23 & 23 & 22 & 24 & 24 & 23 \\ 
  X2007 & 40 & 18 & 22 & 21 & 21 & 22 & 22 & 23 & 23 & 23 & 23 & 23 & 22 & 22 & 22 & 23 & 23 & 22 & 25 & 23 \\ 
   \hline
\end{tabular}
\endgroup
\caption{Missing values by industry-levels A-F}
\label{nanIndustries_0_20}
\end{table}

% latex table generated in R 3.3.0 by xtable 1.8-2 package
% Mon Jun 13 11:09:53 2016
\begin{table}[ht]
\centering
\begingroup\tiny
\begin{tabular}{rrrrrrrrrrrrrrrrrrrrr}
  \hline
 & G & 50 & 51 & 52 & H & I & 60t63 & 64 & JtK & J & K & 70 & 71t74 & LtQ & L & M & N & O & P & Q \\ 
  \hline
X1970 & 279 & 281 & 279 & 281 & 285 & 277 & 279 & 278 & 271 & 280 & 271 & 279 & 276 & 302 & 276 & 278 & 280 & 275 & 648 & 792 \\ 
  X1971 & 279 & 281 & 279 & 281 & 285 & 277 & 279 & 278 & 271 & 280 & 271 & 279 & 276 & 302 & 276 & 278 & 280 & 275 & 648 & 792 \\ 
  X1972 & 279 & 281 & 279 & 280 & 282 & 277 & 279 & 278 & 271 & 279 & 271 & 278 & 276 & 302 & 276 & 277 & 278 & 275 & 648 & 792 \\ 
  X1973 & 279 & 281 & 279 & 279 & 282 & 277 & 277 & 278 & 271 & 279 & 271 & 278 & 276 & 302 & 276 & 276 & 278 & 275 & 648 & 792 \\ 
  X1974 & 279 & 281 & 279 & 279 & 282 & 276 & 276 & 278 & 271 & 278 & 271 & 278 & 276 & 302 & 276 & 276 & 278 & 274 & 648 & 792 \\ 
  X1975 & 277 & 279 & 277 & 278 & 282 & 277 & 277 & 278 & 271 & 278 & 271 & 275 & 276 & 302 & 276 & 276 & 278 & 275 & 648 & 792 \\ 
  X1976 & 213 & 213 & 213 & 213 & 218 & 213 & 213 & 214 & 206 & 214 & 206 & 208 & 212 & 237 & 214 & 212 & 214 & 208 & 593 & 792 \\ 
  X1977 & 213 & 213 & 213 & 213 & 217 & 213 & 213 & 213 & 206 & 214 & 206 & 208 & 212 & 237 & 214 & 212 & 214 & 208 & 593 & 792 \\ 
  X1978 & 213 & 213 & 213 & 213 & 216 & 213 & 213 & 213 & 206 & 214 & 206 & 208 & 212 & 237 & 212 & 212 & 214 & 208 & 593 & 792 \\ 
  X1979 & 213 & 213 & 213 & 213 & 216 & 213 & 213 & 213 & 206 & 214 & 206 & 208 & 212 & 237 & 212 & 212 & 214 & 208 & 593 & 792 \\ 
  X1980 & 213 & 213 & 213 & 213 & 216 & 213 & 213 & 214 & 206 & 214 & 206 & 208 & 212 & 237 & 212 & 212 & 213 & 208 & 593 & 792 \\ 
  X1981 & 213 & 213 & 213 & 213 & 213 & 213 & 213 & 213 & 206 & 214 & 206 & 208 & 212 & 237 & 212 & 212 & 213 & 208 & 593 & 792 \\ 
  X1982 & 213 & 213 & 213 & 213 & 213 & 213 & 213 & 214 & 206 & 214 & 206 & 208 & 212 & 237 & 212 & 212 & 213 & 208 & 593 & 792 \\ 
  X1983 & 213 & 213 & 213 & 213 & 213 & 213 & 213 & 214 & 206 & 214 & 206 & 208 & 212 & 237 & 212 & 212 & 213 & 208 & 593 & 792 \\ 
  X1984 & 213 & 213 & 213 & 213 & 213 & 213 & 213 & 214 & 206 & 214 & 206 & 208 & 212 & 237 & 212 & 212 & 213 & 208 & 593 & 792 \\ 
  X1985 & 213 & 213 & 213 & 213 & 213 & 213 & 213 & 214 & 206 & 215 & 206 & 206 & 212 & 237 & 212 & 212 & 212 & 208 & 593 & 792 \\ 
  X1986 & 213 & 213 & 213 & 213 & 213 & 213 & 213 & 214 & 206 & 215 & 206 & 206 & 212 & 237 & 212 & 212 & 212 & 208 & 593 & 792 \\ 
  X1987 & 213 & 214 & 213 & 213 & 213 & 213 & 213 & 215 & 206 & 215 & 206 & 206 & 212 & 237 & 212 & 212 & 212 & 209 & 593 & 792 \\ 
  X1988 & 213 & 213 & 213 & 213 & 213 & 213 & 213 & 214 & 206 & 214 & 206 & 206 & 212 & 237 & 212 & 212 & 212 & 208 & 593 & 792 \\ 
  X1989 & 213 & 214 & 213 & 213 & 213 & 213 & 213 & 213 & 206 & 214 & 206 & 206 & 212 & 237 & 212 & 212 & 212 & 209 & 593 & 792 \\ 
  X1990 & 213 & 214 & 213 & 213 & 213 & 212 & 212 & 214 & 206 & 215 & 206 & 206 & 212 & 237 & 212 & 212 & 212 & 208 & 593 & 792 \\ 
  X1991 & 148 & 148 & 148 & 148 & 148 & 147 & 147 & 148 & 140 & 150 & 140 & 140 & 148 & 172 & 147 & 147 & 147 & 143 & 538 & 792 \\ 
  X1992 & 148 & 149 & 148 & 148 & 148 & 148 & 148 & 149 & 140 & 150 & 140 & 140 & 147 & 172 & 147 & 147 & 147 & 143 & 538 & 792 \\ 
  X1993 & 89 & 89 & 89 & 89 & 89 & 88 & 88 & 91 & 80 & 90 & 80 & 82 & 88 & 113 & 90 & 90 & 89 & 83 & 518 & 792 \\ 
  X1994 & 89 & 89 & 89 & 89 & 89 & 89 & 89 & 91 & 80 & 91 & 80 & 82 & 88 & 113 & 90 & 90 & 89 & 84 & 518 & 792 \\ 
  X1995 & 24 & 24 & 24 & 24 & 24 & 24 & 24 & 26 & 14 & 25 & 14 & 18 & 22 & 47 & 25 & 24 & 23 & 18 & 463 & 792 \\ 
  X1996 & 24 & 24 & 24 & 24 & 24 & 23 & 23 & 25 & 14 & 25 & 14 & 19 & 22 & 47 & 25 & 24 & 23 & 18 & 463 & 792 \\ 
  X1997 & 23 & 23 & 23 & 23 & 24 & 24 & 24 & 25 & 14 & 25 & 14 & 17 & 22 & 47 & 25 & 24 & 23 & 18 & 463 & 792 \\ 
  X1998 & 23 & 23 & 23 & 23 & 23 & 23 & 23 & 24 & 14 & 25 & 14 & 17 & 22 & 47 & 25 & 25 & 23 & 17 & 463 & 792 \\ 
  X1999 & 23 & 23 & 23 & 24 & 23 & 23 & 23 & 23 & 14 & 25 & 14 & 17 & 22 & 47 & 24 & 24 & 23 & 17 & 463 & 792 \\ 
  X2000 & 23 & 23 & 24 & 23 & 24 & 23 & 23 & 24 & 14 & 25 & 14 & 17 & 22 & 47 & 24 & 24 & 23 & 17 & 463 & 792 \\ 
  X2001 & 23 & 23 & 23 & 23 & 23 & 22 & 22 & 25 & 14 & 25 & 14 & 17 & 22 & 47 & 24 & 24 & 23 & 17 & 463 & 792 \\ 
  X2002 & 23 & 24 & 23 & 23 & 23 & 23 & 23 & 25 & 14 & 24 & 14 & 17 & 22 & 47 & 22 & 23 & 22 & 18 & 463 & 792 \\ 
  X2003 & 23 & 23 & 23 & 23 & 24 & 22 & 23 & 25 & 14 & 25 & 14 & 17 & 22 & 47 & 23 & 22 & 23 & 17 & 463 & 792 \\ 
  X2004 & 23 & 23 & 23 & 23 & 23 & 22 & 22 & 25 & 14 & 26 & 14 & 17 & 22 & 47 & 22 & 22 & 23 & 18 & 463 & 792 \\ 
  X2005 & 23 & 23 & 23 & 23 & 23 & 23 & 23 & 25 & 14 & 26 & 14 & 17 & 22 & 47 & 22 & 23 & 22 & 18 & 463 & 792 \\ 
  X2006 & 23 & 23 & 23 & 23 & 25 & 23 & 23 & 25 & 14 & 27 & 14 & 18 & 22 & 47 & 22 & 22 & 23 & 18 & 463 & 792 \\ 
  X2007 & 23 & 23 & 23 & 23 & 24 & 23 & 23 & 25 & 14 & 25 & 14 & 18 & 22 & 47 & 22 & 23 & 22 & 18 & 463 & 792 \\ 
   \hline
\end{tabular}
\endgroup
\caption{Missing values by industry-levels G-Q}
\label{nanIndustries_21}
\end{table}
 
\end{landscape}
% latex table generated in R 3.3.0 by xtable 1.8-2 package
% Fri Jun 10 14:38:26 2016
\begin{table}[ht]
    \centering
\begingroup\scriptsize
\begin{tabular}{rrrrrrr}
  \hline
 & I & Iq & Ip & K & CAP & D \\ 
  \hline
X1970 & 1903 & 1933 & 2367 & 1941 & 1941 & 1974 \\ 
  X1971 & 1903 & 1933 & 2362 & 1941 & 1941 & 1974 \\ 
  X1972 & 1903 & 1933 & 2344 & 1941 & 1941 & 1974 \\ 
  X1973 & 1903 & 1933 & 2327 & 1941 & 1941 & 1974 \\ 
  X1974 & 1903 & 1933 & 2322 & 1941 & 1941 & 1974 \\ 
  X1975 & 1903 & 1933 & 2313 & 1941 & 1941 & 1974 \\ 
  X1976 & 1474 & 1504 & 1954 & 1512 & 1512 & 1545 \\ 
  X1977 & 1474 & 1504 & 1943 & 1512 & 1512 & 1545 \\ 
  X1978 & 1474 & 1504 & 1933 & 1512 & 1512 & 1545 \\ 
  X1979 & 1474 & 1504 & 1928 & 1512 & 1512 & 1545 \\ 
  X1980 & 1474 & 1504 & 1923 & 1512 & 1512 & 1545 \\ 
  X1981 & 1474 & 1504 & 1918 & 1512 & 1512 & 1545 \\ 
  X1982 & 1474 & 1504 & 1918 & 1512 & 1512 & 1545 \\ 
  X1983 & 1474 & 1504 & 1917 & 1512 & 1512 & 1545 \\ 
  X1984 & 1474 & 1504 & 1914 & 1512 & 1512 & 1545 \\ 
  X1985 & 1474 & 1504 & 1912 & 1512 & 1512 & 1545 \\ 
  X1986 & 1474 & 1504 & 1911 & 1512 & 1512 & 1545 \\ 
  X1987 & 1474 & 1504 & 1920 & 1512 & 1512 & 1545 \\ 
  X1988 & 1474 & 1504 & 1913 & 1512 & 1512 & 1545 \\ 
  X1989 & 1474 & 1504 & 1916 & 1512 & 1512 & 1545 \\ 
  X1990 & 1474 & 1504 & 1913 & 1512 & 1512 & 1545 \\ 
  X1991 & 1045 & 1075 & 1527 & 1083 & 1083 & 1116 \\ 
  X1992 & 1045 & 1075 & 1534 & 1083 & 1083 & 1116 \\ 
  X1993 & 655 & 685 & 1195 & 703 & 703 & 736 \\ 
  X1994 & 655 & 685 & 1196 & 703 & 703 & 736 \\ 
  X1995 & 226 & 256 & 806 & 274 & 274 & 307 \\ 
  X1996 & 226 & 256 & 806 & 274 & 274 & 307 \\ 
  X1997 & 226 & 256 & 786 & 274 & 274 & 307 \\ 
  X1998 & 226 & 256 & 784 & 274 & 274 & 307 \\ 
  X1999 & 226 & 256 & 778 & 274 & 274 & 307 \\ 
  X2000 & 226 & 256 & 786 & 274 & 274 & 307 \\ 
  X2001 & 226 & 256 & 782 & 274 & 274 & 307 \\ 
  X2002 & 226 & 256 & 783 & 274 & 274 & 307 \\ 
  X2003 & 226 & 256 & 787 & 274 & 274 & 307 \\ 
  X2004 & 226 & 256 & 781 & 274 & 274 & 307 \\ 
  X2005 & 226 & 256 & 794 & 274 & 274 & 307 \\ 
  X2006 & 226 & 256 & 798 & 274 & 274 & 307 \\ 
  X2007 & 226 & 256 & 793 & 274 & 274 & 307 \\ 
   \hline
\end{tabular}
\endgroup
\caption{Missing values by variable}
\label{nanVars}
\end{table}


%From the missing values per industry-level, tables \ref{nanIndustries_0_20} and \ref{nanIndustries_21}, it appears data are often missing for the industry levels: P - Private Households With Employed Persons and Q - Extra-Territorial Organizations and Bodies. For this reason, for further analysis of missing data, the levels P and Q were excluded from here on. Industry-levels P and Q are aggragated in LtQ - Community Social and Personal Services, at which level data are usually available. 

\begin{itemize}
\item UK : UK,Ip,TraEq,23 qqs valeurs manquantes (~1990?). Price index investment in residential structures is based only on 70 - real estate activities. Price index products "Other" is based on 
\item USA : price index investment in residential structures is missing for 28 sectors (! KLEMS transforms NA in 0's, see for example line USA,Ip,RStruc,JtK and levels below.)
\end{itemize}

% latex table generated in R 3.3.0 by xtable 1.8-2 package
% Mon Jun 13 11:49:00 2016
\begin{table}[ht]
\centering
\begingroup\tiny
\begin{tabular}{rrrrrrr}
  \hline
 & I & Iq & Ip & K & CAP & D \\ 
  \hline
X1970 & 1672 & 1692 & 2103 & 1710 & 1710 & 1732 \\ 
  X1971 & 1672 & 1692 & 2098 & 1710 & 1710 & 1732 \\ 
  X1972 & 1672 & 1692 & 2080 & 1710 & 1710 & 1732 \\ 
  X1973 & 1672 & 1692 & 2063 & 1710 & 1710 & 1732 \\ 
  X1974 & 1672 & 1692 & 2058 & 1710 & 1710 & 1732 \\ 
  X1975 & 1672 & 1692 & 2049 & 1710 & 1710 & 1732 \\ 
  X1976 & 1254 & 1274 & 1690 & 1292 & 1292 & 1314 \\ 
  X1977 & 1254 & 1274 & 1679 & 1292 & 1292 & 1314 \\ 
  X1978 & 1254 & 1274 & 1669 & 1292 & 1292 & 1314 \\ 
  X1979 & 1254 & 1274 & 1664 & 1292 & 1292 & 1314 \\ 
  X1980 & 1254 & 1274 & 1659 & 1292 & 1292 & 1314 \\ 
  X1981 & 1254 & 1274 & 1654 & 1292 & 1292 & 1314 \\ 
  X1982 & 1254 & 1274 & 1654 & 1292 & 1292 & 1314 \\ 
  X1983 & 1254 & 1274 & 1653 & 1292 & 1292 & 1314 \\ 
  X1984 & 1254 & 1274 & 1650 & 1292 & 1292 & 1314 \\ 
  X1985 & 1254 & 1274 & 1648 & 1292 & 1292 & 1314 \\ 
  X1986 & 1254 & 1274 & 1647 & 1292 & 1292 & 1314 \\ 
  X1987 & 1254 & 1274 & 1656 & 1292 & 1292 & 1314 \\ 
  X1988 & 1254 & 1274 & 1649 & 1292 & 1292 & 1314 \\ 
  X1989 & 1254 & 1274 & 1652 & 1292 & 1292 & 1314 \\ 
  X1990 & 1254 & 1274 & 1649 & 1292 & 1292 & 1314 \\ 
  X1991 & 836 & 856 & 1263 & 874 & 874 & 896 \\ 
  X1992 & 836 & 856 & 1270 & 874 & 874 & 896 \\ 
  X1993 & 456 & 476 & 931 & 494 & 494 & 516 \\ 
  X1994 & 456 & 476 & 932 & 494 & 494 & 516 \\ 
  X1995 & 38 & 58 & 542 & 76 & 76 & 98 \\ 
  X1996 & 38 & 58 & 542 & 76 & 76 & 98 \\ 
  X1997 & 38 & 58 & 522 & 76 & 76 & 98 \\ 
  X1998 & 38 & 58 & 520 & 76 & 76 & 98 \\ 
  X1999 & 38 & 58 & 514 & 76 & 76 & 98 \\ 
  X2000 & 38 & 58 & 522 & 76 & 76 & 98 \\ 
  X2001 & 38 & 58 & 518 & 76 & 76 & 98 \\ 
  X2002 & 38 & 58 & 519 & 76 & 76 & 98 \\ 
  X2003 & 38 & 58 & 523 & 76 & 76 & 98 \\ 
  X2004 & 38 & 58 & 517 & 76 & 76 & 98 \\ 
  X2005 & 38 & 58 & 530 & 76 & 76 & 98 \\ 
  X2006 & 38 & 58 & 534 & 76 & 76 & 98 \\ 
  X2007 & 38 & 58 & 529 & 76 & 76 & 98 \\ 
   \hline
\end{tabular}
\endgroup
\caption{Missing values by variable, excluding the industry-levels P and Q.} 
\label{nanVars_noPQ}
\end{table}


Table \ref{nanVars_noPQ} shows missing values per variable and year, summing over countries, product- and industry-levels (where industry-levels P and Q were excluded). One can see that there are many more missing values for the price index of capital formation (Ip) then for the other variables.  

%To confirm that the values are missing mostly for price indices of capital formation and for the industry levels P - PRIVATE HOUSEHOLDS WITH EMPLOYED PERSONS and Q - EXTRA-TERRITORIAL ORGANIZATIONS AND BODIES, three more counting exercises were performed: one by counting missing values per country and year, summing as previously over all product and industry-levels and over all variables except over the price index; secondly, missing values were counted per country and per year, summing over all variables, products and over all industry-levels except over levels P and Q\footnote{To use only the level LtQ and delete the levels L,M,N,O,P,Q, the data were processed with getData\_noLMNPQ.sh.}. The results are summerized in tables \ref{nanCountries_noIp} and \ref{nanCountries_noLMNOPQ}.

Finally, a count of missing values per year and country was done combining both previous observations: the price index of capital formation was excluded from the count and the industry-levels P and Q. Results can be found in tables \ref{nanCountries_noPQ_noIp}.

% latex table generated in R 3.3.0 by xtable 1.8-2 package
% Mon Jun 13 11:56:39 2016
\begin{table}[ht]
\centering
\begingroup\tiny
\begin{tabular}{rrrrrrrrrrrrr}
  \hline
 & AUS & AUT & CZE & DNK & ESP & FIN & GER & ITA & NLD & SWE & UK & USA \\ 
  \hline
X1970 & 0 & 2090 & 2090 & 0 & 114 & 42 & 2090 & 0 & 0 & 2090 & 0 & 0 \\ 
  X1971 & 0 & 2090 & 2090 & 0 & 114 & 42 & 2090 & 0 & 0 & 2090 & 0 & 0 \\ 
  X1972 & 0 & 2090 & 2090 & 0 & 114 & 42 & 2090 & 0 & 0 & 2090 & 0 & 0 \\ 
  X1973 & 0 & 2090 & 2090 & 0 & 114 & 42 & 2090 & 0 & 0 & 2090 & 0 & 0 \\ 
  X1974 & 0 & 2090 & 2090 & 0 & 114 & 42 & 2090 & 0 & 0 & 2090 & 0 & 0 \\ 
  X1975 & 0 & 2090 & 2090 & 0 & 114 & 42 & 2090 & 0 & 0 & 2090 & 0 & 0 \\ 
  X1976 & 0 & 0 & 2090 & 0 & 114 & 42 & 2090 & 0 & 0 & 2090 & 0 & 0 \\ 
  X1977 & 0 & 0 & 2090 & 0 & 114 & 42 & 2090 & 0 & 0 & 2090 & 0 & 0 \\ 
  X1978 & 0 & 0 & 2090 & 0 & 114 & 42 & 2090 & 0 & 0 & 2090 & 0 & 0 \\ 
  X1979 & 0 & 0 & 2090 & 0 & 114 & 42 & 2090 & 0 & 0 & 2090 & 0 & 0 \\ 
  X1980 & 0 & 0 & 2090 & 0 & 114 & 42 & 2090 & 0 & 0 & 2090 & 0 & 0 \\ 
  X1981 & 0 & 0 & 2090 & 0 & 114 & 42 & 2090 & 0 & 0 & 2090 & 0 & 0 \\ 
  X1982 & 0 & 0 & 2090 & 0 & 114 & 42 & 2090 & 0 & 0 & 2090 & 0 & 0 \\ 
  X1983 & 0 & 0 & 2090 & 0 & 114 & 42 & 2090 & 0 & 0 & 2090 & 0 & 0 \\ 
  X1984 & 0 & 0 & 2090 & 0 & 114 & 42 & 2090 & 0 & 0 & 2090 & 0 & 0 \\ 
  X1985 & 0 & 0 & 2090 & 0 & 114 & 42 & 2090 & 0 & 0 & 2090 & 0 & 0 \\ 
  X1986 & 0 & 0 & 2090 & 0 & 114 & 42 & 2090 & 0 & 0 & 2090 & 0 & 0 \\ 
  X1987 & 0 & 0 & 2090 & 0 & 114 & 42 & 2090 & 0 & 0 & 2090 & 0 & 0 \\ 
  X1988 & 0 & 0 & 2090 & 0 & 114 & 42 & 2090 & 0 & 0 & 2090 & 0 & 0 \\ 
  X1989 & 0 & 0 & 2090 & 0 & 114 & 42 & 2090 & 0 & 0 & 2090 & 0 & 0 \\ 
  X1990 & 0 & 0 & 2090 & 0 & 114 & 42 & 2090 & 0 & 0 & 2090 & 0 & 0 \\ 
  X1991 & 0 & 0 & 2090 & 0 & 114 & 42 & 0 & 0 & 0 & 2090 & 0 & 0 \\ 
  X1992 & 0 & 0 & 2090 & 0 & 114 & 42 & 0 & 0 & 0 & 2090 & 0 & 0 \\ 
  X1993 & 0 & 0 & 2090 & 0 & 114 & 42 & 0 & 0 & 0 & 190 & 0 & 0 \\ 
  X1994 & 0 & 0 & 2090 & 0 & 114 & 42 & 0 & 0 & 0 & 190 & 0 & 0 \\ 
  X1995 & 0 & 0 & 0 & 0 & 114 & 42 & 0 & 0 & 0 & 190 & 0 & 0 \\ 
  X1996 & 0 & 0 & 0 & 0 & 114 & 42 & 0 & 0 & 0 & 190 & 0 & 0 \\ 
  X1997 & 0 & 0 & 0 & 0 & 114 & 42 & 0 & 0 & 0 & 190 & 0 & 0 \\ 
  X1998 & 0 & 0 & 0 & 0 & 114 & 42 & 0 & 0 & 0 & 190 & 0 & 0 \\ 
  X1999 & 0 & 0 & 0 & 0 & 114 & 42 & 0 & 0 & 0 & 190 & 0 & 0 \\ 
  X2000 & 0 & 0 & 0 & 0 & 114 & 42 & 0 & 0 & 0 & 190 & 0 & 0 \\ 
  X2001 & 0 & 0 & 0 & 0 & 114 & 42 & 0 & 0 & 0 & 190 & 0 & 0 \\ 
  X2002 & 0 & 0 & 0 & 0 & 114 & 42 & 0 & 0 & 0 & 190 & 0 & 0 \\ 
  X2003 & 0 & 0 & 0 & 0 & 114 & 42 & 0 & 0 & 0 & 190 & 0 & 0 \\ 
  X2004 & 0 & 0 & 0 & 0 & 114 & 42 & 0 & 0 & 0 & 190 & 0 & 0 \\ 
  X2005 & 0 & 0 & 0 & 0 & 114 & 42 & 0 & 0 & 0 & 190 & 0 & 0 \\ 
  X2006 & 0 & 0 & 0 & 0 & 114 & 42 & 0 & 0 & 0 & 190 & 0 & 0 \\ 
  X2007 & 0 & 0 & 0 & 0 & 114 & 42 & 0 & 0 & 0 & 190 & 0 & 0 \\ 
   \hline
\end{tabular}
\endgroup
\caption{Missing values by country, not including industry levels P Q, nor the capital formation price index}
\label{nanCountries_noPQ_noIp}
\end{table}

%\input{nanVars_noLMNOPQ.tex}
%\input{nanCountries_noLMNOPQ.tex}
% latex table generated in R 3.3.0 by xtable 1.8-2 package
% Fri Jun 10 14:46:45 2016
\begin{table}[ht]
\centering
\begingroup\tiny
\begin{tabular}{rrrrrrrrrrrrr}
  \hline
 & AUS & AUT & CZE & DNK & ESP & FIN & GER & ITA & NLD & SWE & UK & USA \\ 
  \hline
X1970 & 110 & 2200 & 2200 & 55 & 224 & 118 & 2200 & 55 & 110 & 2200 & 110 & 110 \\ 
  X1971 & 110 & 2200 & 2200 & 55 & 224 & 118 & 2200 & 55 & 110 & 2200 & 110 & 110 \\ 
  X1972 & 110 & 2200 & 2200 & 55 & 224 & 118 & 2200 & 55 & 110 & 2200 & 110 & 110 \\ 
  X1973 & 110 & 2200 & 2200 & 55 & 224 & 118 & 2200 & 55 & 110 & 2200 & 110 & 110 \\ 
  X1974 & 110 & 2200 & 2200 & 55 & 224 & 118 & 2200 & 55 & 110 & 2200 & 110 & 110 \\ 
  X1975 & 110 & 2200 & 2200 & 55 & 224 & 118 & 2200 & 55 & 110 & 2200 & 110 & 110 \\ 
  X1976 & 110 & 55 & 2200 & 55 & 224 & 118 & 2200 & 55 & 110 & 2200 & 110 & 110 \\ 
  X1977 & 110 & 55 & 2200 & 55 & 224 & 118 & 2200 & 55 & 110 & 2200 & 110 & 110 \\ 
  X1978 & 110 & 55 & 2200 & 55 & 224 & 118 & 2200 & 55 & 110 & 2200 & 110 & 110 \\ 
  X1979 & 110 & 55 & 2200 & 55 & 224 & 118 & 2200 & 55 & 110 & 2200 & 110 & 110 \\ 
  X1980 & 110 & 55 & 2200 & 55 & 224 & 118 & 2200 & 55 & 110 & 2200 & 110 & 110 \\ 
  X1981 & 110 & 55 & 2200 & 55 & 224 & 118 & 2200 & 55 & 110 & 2200 & 110 & 110 \\ 
  X1982 & 110 & 55 & 2200 & 55 & 224 & 118 & 2200 & 55 & 110 & 2200 & 110 & 110 \\ 
  X1983 & 110 & 55 & 2200 & 55 & 224 & 118 & 2200 & 55 & 110 & 2200 & 110 & 110 \\ 
  X1984 & 110 & 55 & 2200 & 55 & 224 & 118 & 2200 & 55 & 110 & 2200 & 110 & 110 \\ 
  X1985 & 110 & 55 & 2200 & 55 & 224 & 118 & 2200 & 55 & 110 & 2200 & 110 & 110 \\ 
  X1986 & 110 & 55 & 2200 & 55 & 224 & 118 & 2200 & 55 & 110 & 2200 & 110 & 110 \\ 
  X1987 & 110 & 55 & 2200 & 55 & 224 & 118 & 2200 & 55 & 110 & 2200 & 110 & 110 \\ 
  X1988 & 110 & 55 & 2200 & 55 & 224 & 118 & 2200 & 55 & 110 & 2200 & 110 & 110 \\ 
  X1989 & 110 & 55 & 2200 & 55 & 224 & 118 & 2200 & 55 & 110 & 2200 & 110 & 110 \\ 
  X1990 & 110 & 55 & 2200 & 55 & 224 & 118 & 2200 & 55 & 110 & 2200 & 110 & 110 \\ 
  X1991 & 110 & 55 & 2200 & 55 & 224 & 118 & 55 & 55 & 110 & 2200 & 110 & 110 \\ 
  X1992 & 110 & 55 & 2200 & 55 & 224 & 118 & 55 & 55 & 110 & 2200 & 110 & 110 \\ 
  X1993 & 110 & 55 & 2200 & 55 & 224 & 118 & 55 & 55 & 110 & 280 & 110 & 110 \\ 
  X1994 & 110 & 55 & 2200 & 55 & 224 & 118 & 55 & 55 & 110 & 280 & 110 & 110 \\ 
  X1995 & 110 & 55 & 55 & 55 & 224 & 118 & 55 & 55 & 110 & 280 & 110 & 110 \\ 
  X1996 & 110 & 55 & 55 & 55 & 224 & 118 & 55 & 55 & 110 & 280 & 110 & 110 \\ 
  X1997 & 110 & 55 & 55 & 55 & 224 & 118 & 55 & 55 & 110 & 280 & 110 & 110 \\ 
  X1998 & 110 & 55 & 55 & 55 & 224 & 118 & 55 & 55 & 110 & 280 & 110 & 110 \\ 
  X1999 & 110 & 55 & 55 & 55 & 224 & 118 & 55 & 55 & 110 & 280 & 110 & 110 \\ 
  X2000 & 110 & 55 & 55 & 55 & 224 & 118 & 55 & 55 & 110 & 280 & 110 & 110 \\ 
  X2001 & 110 & 55 & 55 & 55 & 224 & 118 & 55 & 55 & 110 & 280 & 110 & 110 \\ 
  X2002 & 110 & 55 & 55 & 55 & 224 & 118 & 55 & 55 & 110 & 280 & 110 & 110 \\ 
  X2003 & 110 & 55 & 55 & 55 & 224 & 118 & 55 & 55 & 110 & 280 & 110 & 110 \\ 
  X2004 & 110 & 55 & 55 & 55 & 224 & 118 & 55 & 55 & 110 & 280 & 110 & 110 \\ 
  X2005 & 110 & 55 & 55 & 55 & 224 & 118 & 55 & 55 & 110 & 280 & 110 & 110 \\ 
  X2006 & 110 & 55 & 55 & 55 & 224 & 118 & 55 & 55 & 110 & 280 & 110 & 110 \\ 
  X2007 & 110 & 55 & 55 & 55 & 224 & 118 & 55 & 55 & 110 & 280 & 110 & 110 \\ 
   \hline
\end{tabular}
\endgroup
\caption{Missing values by country, capital information price index not included.}
\label{nanCountries_noIp}
\end{table}

%\input{nanCountries_noLMNOPQ_noIp.tex}
\clearpage

\clearpage
\section{Metrics for the system year $\times$ product/industry} \label{metrics}

If a contingency-matrix years $\times$ product/industry is considered, with years in the rows and products/industries indexed by $p$ and $r$ in the columns, with the variable $I_{yp}/V\!\!A_y$, then the rowprofiles, indexed by $y$, look like:
\begin{equation}  \label{row_year-prod}
\frac{I_{yp}/V\!\!A_y}{\sum_r I_{yr}/V\!\!A_y} = \frac{I_{yp}}{GFCF_y} , 
\end{equation}

with as the average rowprofile :
\begin{equation}  \label{eq:rowAvrg_year-prod}
\frac{1}{n_y}\sum_y \frac{I_{yp}}{GFCF_y} . 
\end{equation}

The distance between the rowprofiles is then :
\begin{equation}  \label{eq:distanceRow_year-prod}
d_{yy'}= \sum_p \frac{( I_{yp}/GFCF_y - I_{y'p}/GFCF_{y'})^2 }{\frac{1}{n_y}\sum_z I_{zp}/GFCF_z}.
\end{equation}

The column profiles, indexed by $p$, look like :
\begin{equation}  \label{eq:col_year-prod}
\frac{I_{yp}/V\!\!A_y}{\sum_z I_{zp}/V\!\!A_z},
\end{equation}

with as average column profile :
\begin{equation}  \label{eq:colAvrg_year-prod}
\frac{1}{n_p}  \sum_p      \frac{I_{yp}/V\!\!A_y}{\sum_z I_{zp}/V\!\!A_z}.
\end{equation}

The distance between the column profiles :
\begin{equation}  \label{eq:distanceCol_year-prod}
d_{pq} = \sum_y \frac{( \frac{I_{yp}/V\!\!A_y}{\sum_z I_{zp}/V\!\!A_z}  -\frac{I_{yq}/V\!\!A_y}{\sum_z I_{zq}/V\!\!A_z}              )^2}{    \frac{1}{n_p}  \sum_r      \frac{I_{yr}/V\!\!A_y}{\sum_z I_{zr}/V\!\!A_z}       }.
\end{equation}

From \ref{eq:distanceRow_year-prod} one can see that the correction for value added does not play any role in the standard calculation of the distances between row profiles. This is normal since correcting for value added per year comes down to weighting every row with a factor. The method is precisely designed to give equal total weights to all rows by constructing rowprofiles of which the elements sum to 1. In the calculation of the distance between the column profiles, weighting by VA does play a role.

Equation \ref{eq:distanceRow_year-prod} can be replaced by another measure of distance that compares I to VA in stead of to GFCF :
\begin{equation}  \label{eq:distanceRow_year-prod_alt}
d_{yy'}= \sum_p \frac{( I_{yp}/VA_y - I_{y'p}/VA_{y'})^2 }{\frac{1}{n_y}\sum_z I_{zp}/VA_z}.
\end{equation}
In the case of the system year $\times$ product, this measure will pick up whether a larger share of the national budget is allocated to formation of some fixed asset, without the overall level of investment affecting the individual levels. It is the overall level of VA which will affect individual levels. VA is less variable though than I, and I/VA is less variable than I (see ...). One could also correct I with the approximations of VA by the best linear regression of VA (see...). 










\section{Charts}

\subsection{Per product I/VA stats and charts}

\clearpage
\subsubsection{GFCF}
\input{img/summarizeData/Summary_SLR_GFCF_TOT.tex}
\vfill
\begin{figure}[!h]
\hfill\begin{minipage}{.85\textwidth}\centering
    \includegraphics[width=13cm]{img/plotted_I_K_VA/I_ShareVA-GFCF.pdf}
    \caption{\label{fig:I_ShareVA-GFCF}Share of GFCF in value added (\%), source: KLEMS}\todo{add appendix with country comparison graphs}
\end{minipage}
\end{figure} 
% latex table generated in R 3.3.0 by xtable 1.8-2 package
% Tue Jun 14 17:02:06 2016
\begin{table}[ht]
\centering
\begingroup\tiny
\begin{tabular}{rlllllll}
  \hline
 & Min & 1st Quantile & Median & Mean & 3rd Quantile & Max & NA \\ 
  \hline
    X1970 & 48194   & 1st Qu.: 58234   & Median : 79488   & Mean   :159545   & 3rd Qu.:144408   & Max.   :583849   & NA's   :5   \\ 
      X1971 & 49697   & 1st Qu.: 58334   & Median : 81558   & Mean   :165561   & 3rd Qu.:143621   & Max.   :623759   & NA's   :5   \\ 
      X1972 & 47784   & 1st Qu.: 61763   & Median : 80229   & Mean   :179374   & 3rd Qu.:155205   & Max.   :693666   & NA's   :5   \\ 
      X1973 & 48297   & 1st Qu.: 66136   & Median : 81712   & Mean   :192360   & 3rd Qu.:164470   & Max.   :755299   & NA's   :5   \\ 
      X1974 & 45476   & 1st Qu.: 66020   & Median : 80660   & Mean   :184386   & 3rd Qu.:160655   & Max.   :711220   & NA's   :5   \\ 
      X1975 & 44418   & 1st Qu.: 65459   & Median : 78902   & Mean   :169702   & 3rd Qu.:145473   & Max.   :642731   & NA's   :5   \\ 
      X1976 & 28415   & 1st Qu.: 59126   & Median : 74962   & Mean   :162408   & 3rd Qu.:154143   & Max.   :694524   & NA's   :4   \\ 
      X1977 & 30973   & 1st Qu.: 59260   & Median : 75052   & Mean   :173434   & 3rd Qu.:155529   & Max.   :779359   & NA's   :4   \\ 
      X1978 & 28536   & 1st Qu.: 58068   & Median : 78975   & Mean   :185384   & 3rd Qu.:159505   & Max.   :864548   & NA's   :4   \\ 
      X1979 & 30139   & 1st Qu.: 55573   & Median : 80575   & Mean   :190740   & 3rd Qu.:160273   & Max.   :906956   & NA's   :4   \\ 
      X1980 & 30940   & 1st Qu.: 55780   & Median : 82303   & Mean   :179863   & 3rd Qu.:146218   & Max.   :839859   & NA's   :4   \\ 
      X1981 & 30847   & 1st Qu.: 53829   & Median : 80485   & Mean   :174975   & 3rd Qu.:124070   & Max.   :841640   & NA's   :4   \\ 
      X1982 & 28213   & 1st Qu.: 53890   & Median : 77400   & Mean   :165413   & 3rd Qu.:129278   & Max.   :771605   & NA's   :4   \\ 
      X1983 & 27960   & 1st Qu.: 53328   & Median : 80196   & Mean   :172081   & 3rd Qu.:129381   & Max.   :821231   & NA's   :4   \\ 
      X1984 & 27519   & 1st Qu.: 51687   & Median : 87262   & Mean   :191472   & 3rd Qu.:140746   & Max.   :944671   & NA's   :4   \\ 
      X1985 & 28990   & 1st Qu.: 54745   & Median : 90920   & Mean   :202781   & 3rd Qu.:152165   & Max.   :999477   & NA's   :4   \\ 
      X1986 &  29413   & 1st Qu.:  59740   & Median :  89631   & Mean   : 208801   & 3rd Qu.: 161766   & Max.   :1011273   & NA's   :4   \\ 
      X1987 &  30521   & 1st Qu.:  65932   & Median :  96650   & Mean   : 214353   & 3rd Qu.: 167415   & Max.   :1022680   & NA's   :4   \\ 
      X1988 &  32419   & 1st Qu.:  74250   & Median : 108097   & Mean   : 222572   & 3rd Qu.: 173036   & Max.   :1048798   & NA's   :4   \\ 
      X1989 &  33681   & 1st Qu.:  82279   & Median : 113245   & Mean   : 229724   & 3rd Qu.: 176834   & Max.   :1073816   & NA's   :4   \\ 
      X1990 &  35125   & 1st Qu.:  85821   & Median : 107405   & Mean   : 228936   & 3rd Qu.: 175282   & Max.   :1069253   & NA's   :4   \\ 
      X1991 &  37864   & 1st Qu.:  89194   & Median : 108404   & Mean   : 238013   & 3rd Qu.: 190552   & Max.   :1005511   & NA's   :3   \\ 
      X1992 &  38043   & 1st Qu.:  94358   & Median : 108232   & Mean   : 244881   & 3rd Qu.: 188041   & Max.   :1052855   & NA's   :3   \\ 
      X1993 &  37458   & 1st Qu.:  89941   & Median : 132648   & Mean   : 248986   & 3rd Qu.: 231042   & Max.   :1133444   & NA's   :2   \\ 
      X1994 &  39281   & 1st Qu.:  94325   & Median : 141159   & Mean   : 265240   & 3rd Qu.: 246743   & Max.   :1223716   & NA's   :2   \\ 
      X1995 &  38809   & 1st Qu.: 104477   & Median : 180587   & Mean   : 297942   & 3rd Qu.: 356115   & Max.   :1303143   & NA's   :1   \\ 
      X1996 &  39791   & 1st Qu.: 110833   & Median : 186082   & Mean   : 319396   & 3rd Qu.: 363256   & Max.   :1431684   & NA's   :1   \\ 
      X1997 &  40502   & 1st Qu.: 121122   & Median : 191295   & Mean   : 336786   & 3rd Qu.: 367112   & Max.   :1563284   & NA's   :1   \\ 
      X1998 &  42460   & 1st Qu.: 132420   & Median : 202340   & Mean   : 367618   & 3rd Qu.: 392923   & Max.   :1747948   & NA's   :1   \\ 
      X1999 &  44306   & 1st Qu.: 147673   & Median : 214688   & Mean   : 400340   & 3rd Qu.: 421456   & Max.   :1954997   & NA's   :1   \\ 
      X2000 &  47532   & 1st Qu.: 147687   & Median : 234914   & Mean   : 430165   & 3rd Qu.: 444280   & Max.   :2126174   & NA's   :1   \\ 
      X2001 &  48370   & 1st Qu.: 160090   & Median : 244889   & Mean   : 439510   & 3rd Qu.: 434772   & Max.   :2110328   & NA's   :1   \\ 
      X2002 &  46709   & 1st Qu.: 177986   & Median : 255087   & Mean   : 439749   & 3rd Qu.: 417617   & Max.   :2061135   & NA's   :1   \\ 
      X2003 &  49951   & 1st Qu.: 186166   & Median : 251720   & Mean   : 453943   & 3rd Qu.: 421688   & Max.   :2151026   & NA's   :1   \\ 
      X2004 &  50290   & 1st Qu.: 198759   & Median : 263131   & Mean   : 482792   & 3rd Qu.: 435999   & Max.   :2307157   & NA's   :1   \\ 
      X2005 &  51788   & 1st Qu.: 213140   & Median : 295348   & Mean   : 519020   & 3rd Qu.: 470559   & Max.   :2504121   & NA's   :1   \\ 
      X2006 &  55402   & 1st Qu.: 230141   & Median : 331755   & Mean   : 570864   & 3rd Qu.: 523046   & Max.   :2714466   & NA's   :1   \\ 
      X2007 &  59031   & 1st Qu.: 251069   & Median : 411172   & Mean   : 615116   & 3rd Qu.: 567639   & Max.   :2841059   & NA's   :1   \\ 
   \hline
\end{tabular}
\endgroup
\caption{Summary total investment per year (product = GFCF, industr=TOT), over countries} 
\end{table}

\clearpage
For the countries under consideration, Austria, Czech Republic, Denmark,Spain, Finland, France, Germany, Italy, the Netherlands, Slovenia, Sweden, UK, USA, Japan, Australia, the portion of the annual national budgets allocated to investment in fixed assets, between 1970 and 2007, averages at $24.2\%$, reaches a minimum in the USA in 1992 with $16.5\%$ and three close maxima of about $35\%$ in Finland in 1975, in the Czech Republic in 1996, and in Spain in 2007. Over the period considered, averaging over all countries, investment in non-ICT assets decreases with $0.16\%$ of VA, whereas investment in ICT assets shows a steady increase of $0.06\%$ of VA per year \footnote{The changes per year in I/VA are not growth rates. They are the slope of the linear regression $I_{pc}/VA_{c} = trend_{pc} year + intercept + residual$. The trend is thus interpreted as }. Overall GFCF decreases on average with $0.097\%$ per year (view \cref{table:I_ShareVA-GFCF-global}). The steady increase in investment in ICT products is most recently pulled by software assets, which reaches a maximum of $2.9\%$ in Sweden in 2000 and to stay above $2.4\%$ since, closely followed by $2.5\%$ in Denmark in 2007 and $2.2\%$ in the USA in 2000. Since 1997, all countries in the sample allocate at least $0.5\%$ of VA to investment in software, and on average $1.6\%$ in 2007.

Investment in IT gradually increased throughout the seventies and eigthies, to settle on an average investment level of around $1\%$ of VA from the middle of the eighties, throughout the nineties and until 2007. The overall trend is an increase in IT assets of $0.014\%$ of VA per year. Denmark is an early investor in IT equipment surpassing $1.5\%$ in the second half of the seventies (and rarely dipping below this level since), peaking from 1984-1987 with I reaching $2.7\%$ - $2.9\%$ of VA. (What industries do this investing in Denmark?) 

Investment in communication technologies had already attained higher levels in 1970 then software and IT investment (see figure \ref{fig:I_ShareVA-gridCountries-TOT}). The average investment level has increased slightly over the years, with a peak around 2000, reaching $0.09\%$ of VA. (peak mobile infrastructure ? which industries ?). In this year the Czech Republic invests $1.5\%$ of VA. The overall average over all years and all countries is $0.06\%$ of VA.


For the USA, figure \ref{fig:I_ShareVA-USA-TOT} shows GFCF/VA has not often surpassed $20\%$ after 1976 up til 2007. The dip in 1992 seems to be mostly due to a reduction in Construction, both residential and other. Notably, from 1992 on, capital formation in residential structures keeps rising in the USA, to boom in 2005, at which point it reaches $6.5\%$ of VA, after which it dramatically goes down in the years after.

Investment is highest in 1970 at of VA. It is lowest in at  of VA. Over the next 38 years, for many countries, investment as a share of value added decreased at a yearly rate ranging from of $0.65\%$ - CZE, to $0.42\%$ - GER to around $0.10\%$ for 

\clearpage
\subsubsection{Construction}

The asset level Construction encompasses both residential (RStruc) and non-residential construction (OCon). On average over time and space, it accounts for $56\%$ of GFCF, with $25\%$ going to residential construction and  $31\%$ going to non-residential construction.
Investment in construction decreased over the period and country-sample considered by $0.11\%$ per year, with residential construction going down slightly faster then non-residential construction, $0.062\%$ and $0.050\%$ respectively. For a few countries, investment in construction does not go significantly decrease, or fluctuates more, on average (Sweden, Slovenia, UK, USA). In Spain, construction significantly increases on average by $0.17\%$ of VA per year, evolution which can be decomposed in 2 periods: during the first period, investment in construction is in line with the levels of the other countries, specifically in the eurozone ; from 1997 on, real estate investment increases by approx $1\%$ of VA per year, to reach about $24\%$ of VA in 2006 or $0.11\%$ of GFCF. Investment in construction starts rising from 1985 on, taking a dip in the middle of the nineties, to start the linear rise in 1997. 

Non-residential construction starts increasing rapidly in the early eighties, interrupted by 2 dips in the nominal level in the nineties, to continue rising rapidly together with residential construction after 1997. (see $analysis/plots/plottedData/I-ESP$)

There is a spread (max-min) in Con $I/VA$ of about $0.08 \%$ per year over countries, spread going up as the Spanish building bubble booms to $1.5 \%$ in 2005. Dynamics seem to be pretty close between some countries, such as a peak in the middle of the seventies, end of the eighties, a rise in the beginning of the years 2000. (Discuss context of peaks.)

The spread in $I/VA$ over countries and years is similar between RStruc and OCon, as is the standard deviation (see table \ref{table:I_ShareVA-GFCF-global}). The kurtosis of RStruc is much higher then that of OCon, indicating more outliers: Sweden, Spain, France, Denmark (\cref{table:I_ShareVA-GFCF-global}). 

The sectors investing in residential structures are the real estate sector ; investment in non-residential structures is done by Community and some other sectors \todo{Detailed analysis of which sectors invest in what.}

\begin{itemize}
\item For GFCF and every asset: what is the min and max growth rate (in absolute values) from year to year ? How abrupt does it go up/down? What is the relevant variable to measure nominal investment change (think psychology factor)?
\item Composition of GFCF. Average composition in 1970, 1995, 2007 in europe, usa, japan, \% of GFCF per asset.   Percentage construction etc : an average ? outliers ? 

\end{itemize}

%\input{img/summarizeData/Summary_OMach_TOT.tex}

Investment in travel equipment averages over the 38 years under consideration and all countries at $0.022 \%$ of VA. The overall trend in investment in travel equipment is not pronounced ($-0.005 \%$ of VA). Notable outliers are Australia before 1990 with an investment level of above $3 \%$ of VA and Czech Republic where investment in travel equipment sharply inncreases and surpasses $3 \%$ of VA after the year 2000.

TraEq $I/VA$ significantly decreases over the whole period in Australia, Finland, the UK and USA. It significantly increases in Sweden from 1993 on, the data are missing before that date, and in the Czech Republic, for which there is data from 1995 on. Kurtosis is negative for many countries over the 38 period considered, indicating a relatively constant investment level (compare mean/stddev between prods,ctries). 

Investment per VA in OMach decreases over the period considered for most countries (Australia, Austria, Czech Republic, Spain, Finland, France, Germany, Italy, the Netherlands, UK, USA). In Japan and Denmark there is no overall trend over the period considered.

\clearpage
\subsubsection{Other}

\input{img/summarizeData/Summary_SLR_Other_TOT.tex}
\vfill
\begin{figure}[!h]
\hfill\begin{minipage}{.85\textwidth}\centering
    \includegraphics[width=13cm]{img/plotted_I_K_VA/I_ShareVA-Other.pdf}
    \caption{\label{fig:I_ShareVA-GFCF}Share of GFCF in value added (\%), source: KLEMS}
\end{minipage}
\end{figure} 
%\input{img/summarizeData/Summary_Other_TOT.tex}

This category groups investment in Agriculture, Other intangiables and other products (describe more specifically).

The investment pattern is very different, with much higher levels in Japan, Australia and the Netherlands than the other countries. The overall downward slight trend reported in table \ref{table:I_ShareVA-GFCF-global} is probably mostly due to Australia. Australia very high in the 70's, from turning bush into agricultural productive capital, terrain?

\clearpage
\subsubsection{ICT}

\input{img/summarizeData/Summary_SLR_ICT_TOT.tex}
\vfill
\begin{figure}[!h]
\hfill\begin{minipage}{.85\textwidth}\centering
    \includegraphics[width=13cm]{img/plotted_I_K_VA/I_ShareVA-ICT.pdf}
    \caption{\label{fig:I_ShareVA-ICT}Share of ICT in value added (\%), source: KLEMS}
\end{minipage}
\end{figure} 
%\input{img/summarizeData/Summary_ICT_TOT.tex}

\clearpage

\input{img/summarizeData/Summary_SLR_CT_TOT.tex}
\vfill
\begin{figure}[!h]
\hfill\begin{minipage}{.85\textwidth}\centering
    \includegraphics[width=13cm]{img/plotted_I_K_VA/I_ShareVA-CT.pdf}
    \caption{\label{fig:I_ShareVA-CT}Share of CT in value added (\%), source: KLEMS}
\end{minipage}
\end{figure} 
%\input{img/summarizeData/Summary_CT_TOT.tex}
\clearpage
\input{img/summarizeData/Summary_SLR_IT_TOT.tex}
\vfill
\begin{figure}[!h]
\hfill\begin{minipage}{.85\textwidth}\centering
    \includegraphics[width=13cm]{img/plotted_I_K_VA/I_ShareVA-IT.pdf}
    \caption{\label{fig:I_ShareVA-IT}Share of IT in value added (\%), source: KLEMS}
\end{minipage}
\end{figure} 
%\input{img/summarizeData/Summary_IT_TOT.tex}
\clearpage
\input{img/summarizeData/Summary_SLR_Soft_TOT.tex}
\vfill
\begin{figure}[!h]
\hfill\begin{minipage}{.85\textwidth}\centering
    \includegraphics[width=13cm]{img/plotted_I_K_VA/I_ShareVA-Soft.pdf}
    \caption{\label{fig:I_ShareVA-Soft}Share of Soft in value added (\%), source: KLEMS}
\end{minipage}
\end{figure} 
%\input{img/summarizeData/Summary_Soft_TOT.tex}
\clearpage

\paragraph{CT} Investment in communications technology fluctuates between 0 and 1,5 $\%$ of VA for the observed countries, a light upward trend, is it significant ? For a number of countries, there is peak in investment a little before the year 2000 : UK, USA, CZE, SVN, FIN and t oa lesser extent SWE and JPN. What is this peak ? Wires for the internet ? Mobile phone isntallations, masts etc ?  
\paragraph{IT} Investment in IT took of in the 1980's, except for Denmark and Italy, which stayed around the 1970 level. FInlands investment in IT peaked in the mid-nineties and has declined since. Italy, Austria, France, Germany and Spain invested less in the 90's and 2000's then CZE, SVN, Denmark, UK, Sweden and Japan, Australia ; the US and the Netherlands are in between.  
\paragraph{Software} Investment in software has been going up steadily from $0\%$ to around $3\%$  of VA over the last 40 years. 




\subsection{Per product industry-stacked barcharts, countrygrid, I}
\clearpage
\begin{figure}[!h]
    \centering
    \includegraphics{img/plottedIndustries/I_RStruc.pdf}
    \caption{\label{fig:I_RStruc}source: KLEMS.}
\end{figure} 
\clearpage
\begin{figure}[!h]
    \centering
    \includegraphics{img/plottedIndustries/I_OCon.pdf}
    \caption{\label{fig:I_OCon}source: KLEMS.}
\end{figure} 
\clearpage
\begin{figure}[!h]
    \centering
    \includegraphics{img/plottedIndustries/I_TraEq.pdf}
    \caption{\label{fig:I_TraEq}source: KLEMS.}
\end{figure} 
\clearpage
\begin{figure}[!h]
    \centering
    \includegraphics{img/plottedIndustries/I_OMach.pdf}
    \caption{\label{fig:I_OMach}source: KLEMS.}
\end{figure} 
\clearpage
\begin{figure}[!h]
    \centering
    \includegraphics{img/plottedIndustries/I_Other.pdf}
    \caption{\label{fig:I_Other}source: KLEMS.}
\end{figure} 
\clearpage
\begin{figure}[!h]
    \centering
    \includegraphics{img/plottedIndustries/I_CT.pdf}
    \caption{\label{fig:I_CT}source: KLEMS.}
\end{figure} 
\clearpage
\begin{figure}[!h]
    \centering
    \includegraphics{img/plottedIndustries/I_IT.pdf}
    \caption{\label{fig:I_IT}source: KLEMS.}
\end{figure}
\clearpage
\begin{figure}[!h]
    \centering
    \includegraphics{img/plottedIndustries/I_Soft.pdf}
    \caption{\label{fig:I_Soft}source: KLEMS.}
\end{figure}


\subsection{Per country, product-stacked barcharts, I and I/VA}

\clearpage
\subsubsection{Austria} 
\begin{figure}[!h]
    \centering
    \includegraphics{img/plottedProducts-bar/I-AUT-TOT.pdf}
    \caption{\label{fig:I-AUT-TOT}Evolution of GFCF composition in Austria (millions of euros), source: KLEMS.}
\end{figure} 
\begin{figure}[!h]
    \centering
    \includegraphics{img/plottedProducts-bar/I_ShareVA-AUT-TOT.pdf}
    \caption{\label{fig:I_ShareVA-AUT-TOT}Evolution of GFCF composition in Austria. Investment as share of value added (\%), source: KLEMS.}
\end{figure} 

\clearpage
\subsubsection{The Netherlands} 
\begin{figure}[!h]
    \centering
    \includegraphics{img/plottedProducts-bar/I-NLD-TOT.pdf}
    \caption{\label{fig:I-NLD-TOT}Evolution of GFCF composition in the Netherlands (millions of euros), source: KLEMS.}
\end{figure} 
\begin{figure}[!h]
    \centering
    \includegraphics{img/plottedProducts-bar/I_ShareVA-NLD-TOT.pdf}
    \caption{\label{fig:I_ShareVA-NLD-TOT}Evolution of GFCF composition in the Netherlands. Investment as share of value added (\%), source: KLEMS.}
\end{figure} 

\clearpage
\subsubsection{Germany} 
\begin{figure}[!h]
    \centering
    \includegraphics{img/plottedProducts-bar/I-GER-TOT.pdf}
    \caption{\label{fig:I-GER-TOT}Evolution of GFCF composition in Germany (millions of euros), source: KLEMS.}
\end{figure} 
\begin{figure}[!h]
    \centering
    \includegraphics{img/plottedProducts-bar/I_ShareVA-GER-TOT.pdf}
    \caption{\label{fig:I_ShareVA-GER-TOT}Evolution of GFCF composition in Germany. Investment as share of value added (\%), source: KLEMS.}
\end{figure} 
\clearpage
\subsubsection{France} 
\begin{figure}[!h]
    \centering
    \includegraphics{img/plottedProducts-bar/I-FRA-TOT.pdf}
    \caption{\label{fig:I-FRA-TOT}Evolution of GFCF composition in France (millions of euros), source: KLEMS.}
\end{figure} 
\begin{figure}[!h]
    \centering
    \includegraphics{img/plottedProducts-bar/I_ShareVA-FRA-TOT.pdf}
    \caption{\label{fig:I_ShareVA-FRA-TOT}Evolution of GFCF composition in France. Investment as share of value added (\%), source: KLEMS.}
\end{figure} 

\clearpage
\subsubsection{Spain} 
\begin{figure}[!h]
    \centering
    \includegraphics{img/plottedProducts-bar/I-ESP-TOT.pdf}
    \caption{\label{fig:I-ESP-TOT}Evolution of GFCF composition in Spain (millions of euros), source: KLEMS.}
\end{figure} 
\begin{figure}[!h]
    \centering
    \includegraphics{img/plottedProducts-bar/I_ShareVA-ESP-TOT.pdf}
    \caption{\label{fig:I_ShareVA-ESP-TOT}Evolution of GFCF composition in Spain. Investment as share of value added (\%), source: KLEMS.}
\end{figure} 

\clearpage
\subsubsection{Italy} 
\begin{figure}[!h]
    \centering
    \includegraphics{img/plottedProducts-bar/I-ITA-TOT.pdf}
    \caption{\label{fig:I-ITA-TOT}Evolution of GFCF composition in Italy (millions of euros), source: KLEMS.}
\end{figure} 
\begin{figure}[!h]
    \centering
    \includegraphics{img/plottedProducts-bar/I_ShareVA-ITA-TOT.pdf}
    \caption{\label{fig:I_ShareVA-ITA-TOT}Evolution of GFCF composition in Italy. Investment as share of value added (\%), source: KLEMS.}
\end{figure} 

\clearpage
\subsubsection{Denmark}
\begin{figure}[!h]
    \centering
    \includegraphics{img/plottedProducts-bar/I-DNK-TOT.pdf}
    \caption{\label{fig:I-DKN-TOT}Evolution of GFCF composition in Denmark (millions danish krone), source: KLEMS.}
\end{figure} 
\begin{figure}[!h]
    \centering
    \includegraphics{img/plottedProducts-bar/I_ShareVA-DNK-TOT.pdf}
    \caption{\label{fig:I_ShareVA-DKN-TOT}Evolution of GFCF composition in Denmark. Investment as share of value added (\%), source: KLEMS.}
\end{figure} 
Denmark is an early investor in IT equipment surpassing $1.5\%$ in the second half of the seventies (and rarely dipping below this level since), peaking from 1984-1987 with I reaching $2.7\%$ - $2.9\%$ of VA. (What industries do this investing in Denmark?) 
 Denmarks capital stock Other construction in the 1970s very high, been declining since. Why?

\clearpage
\subsubsection{Finland}
\begin{figure}[!h]
    \centering
    \includegraphics{img/plottedProducts-bar/I-FIN-TOT.pdf}
    \caption{\label{fig:I-FIN-TOT}Evolution of GFCF composition in Finland (millions of euros), source: KLEMS.}
\end{figure} 
\begin{figure}[!h]
    \centering
    \includegraphics{img/plottedProducts-bar/I_ShareVA-FIN-TOT.pdf}
    \caption{\label{fig:I_ShareVA-FIN-TOT}Evolution of GFCF composition in Finland. Investment as share of value added (\%), source: KLEMS.}
\end{figure} 

\clearpage
\subsubsection{Sweden}
\begin{figure}[!h]
    \centering
    \includegraphics{img/plottedProducts-bar/I-SWE-TOT.pdf}
    \caption{\label{fig:I-SWE-TOT}Evolution of GFCF composition in Sweden (millions swedish krone), source: KLEMS.}
\end{figure}
\begin{figure}[!h]
    \centering
    \includegraphics{img/plottedProducts-bar/I_ShareVA-SWE-TOT.pdf}
    \caption{\label{fig:I_ShareVA-SWE-TOT}Evolution of GFCF composition in Sweden. Investment as share of value added (\%), source: KLEMS.}
\end{figure} 

\clearpage
\subsubsection{UK}
\begin{figure}[!h]
    \centering
    \includegraphics{img/plottedProducts-bar/I-UK-TOT.pdf}
    \caption{\label{fig:I-UK-TOT}Evolution of GFCF composition in the United Kingdom (millions of pounds), source: KLEMS.}
\end{figure} 

\begin{figure}[!h]
    \centering
    \includegraphics{img/plottedProducts-bar/I_ShareVA-UK-TOT.pdf}
    \caption{\label{fig:I_ShareVA-UK-TOT}Evolution of GFCF composition in the United Kingdom. Investment as share of value added (\%), source: KLEMS.}
\end{figure} 

\clearpage
\subsubsection{Czech Republic} 
\begin{figure}[!h]
    \centering
    \includegraphics{img/plottedProducts-bar/I-CZE-TOT.pdf}
    \caption{\label{fig:I-CZE-TOT}Evolution of GFCF composition in the Czech Republic (millions of Czech Koruna), source: KLEMS.}
\end{figure}
\begin{figure}[!h]
    \centering
    \includegraphics{img/plottedProducts-bar/I_ShareVA-CZE-TOT.pdf}
    \caption{\label{fig:I_ShareVA-CZE-TOT}Evolution of GFCF composition in the Czech Republic. Investment as share of value added (\%), source: KLEMS.}
\end{figure} 


\clearpage
\subsubsection{SVN} 
\begin{figure}[!h]
    \centering
    \includegraphics{img/plottedProducts-bar/I-SVN-TOT.pdf}
    \caption{\label{fig:I-SVN-TOT}Evolution of GFCF composition in Slovenia (millions of Slovenian Tolar), source: KLEMS.}
\end{figure} 
\begin{figure}[!h]
    \centering
    \includegraphics{img/plottedProducts-bar/I_ShareVA-SVN-TOT.pdf}
    \caption{\label{fig:I_ShareVA-SVN-TOT}Evolution of GFCF composition in Slovenia. Investment as share of value added (\%), source: KLEMS.}
\end{figure} 

\clearpage
\subsubsection{Japan}
\begin{figure}[!h]
    \centering
    \includegraphics{img/plottedProducts-bar/I-JPN-TOT.pdf}
    \caption{\label{fig:I-JPN-TOT}Evolution of GFCF composition in Japan (millions of Yen), source: KLEMS.}
\end{figure} 
\begin{figure}[!h]
    \centering
    \includegraphics{img/plottedProducts-bar/I_ShareVA-JPN-TOT.pdf}
    \caption{\label{fig:I_ShareVA-JPN-TOT}Evolution of GFCF composition in Japan. Investment as share of value added (\%), source: KLEMS.}
\end{figure} 

\clearpage
\subsubsection{Australia}
\begin{figure}[!h]
    \centering
    \includegraphics{img/plottedProducts-bar/I-AUS-TOT.pdf}
    \caption{\label{fig:I-AUS-TOT}Evolution of GFCF composition in the Australia (millions of Australian dollars), source: KLEMS.}
\end{figure} 
\begin{figure}[!h]
    \centering
    \includegraphics{img/plottedProducts-bar/I_ShareVA-AUS-TOT.pdf}
    \caption{\label{fig:I_ShareVA-AUS-TOT}Evolution of GFCF composition in the Australia. Investment as share of value added (\%), source: KLEMS.}
\end{figure} 

\clearpage
\subsubsection{US}
\begin{figure}[!h]
    \centering
    \includegraphics{img/plottedProducts-bar/I-USA-TOT.pdf}
    \caption{\label{fig:I-USA-TOT}Evolution of GFCF composition in the United States (millions of dollars), source: KLEMS.}
\end{figure} 
\begin{figure}[!h]
    \centering
    \includegraphics{img/plottedProducts-bar/I_ShareVA-USA-TOT.pdf}
    \caption{\label{fig:I_ShareVA-USA-TOT}Evolution of GFCF composition in the United States. Investment as share of value added (\%), source: KLEMS.}
\end{figure} 




%---------------------------------------------------------------------------------

\clearpage
\subsection{Per industry composition of investment, USA}

\begin{figure}[!h]
    \centering
    \includegraphics{img/plotted_perCountry/I-VA-Product-Total-USA.pdf}
    \caption{\label{fig:I-VA-Product-Total-USA}Evolution of the composition of capital formation, total industry, USA. Investment as share of value added (\%), source: KLEMS.}
\end{figure} 

\begin{figure}[!h]
    \centering
    \includegraphics{img/plotted_perCountry/I-VA-Product-Agri-USA.pdf}
    \caption{\label{fig:I-VA-Product-Agri-USA}Evolution of the agriculture sector's composition of capital formation, USA. Investment as share of value added (\%), source: KLEMS.}
\end{figure} 

\begin{figure}[!h]
    \centering
    \includegraphics{img/plotted_perCountry/I-VA-Product-Mining-USA.pdf}
    \caption{\label{fig:I-VA-Product-Mining-USA}Evolution of the mining sector's composition of capital formation, USA. Investment as share of value added (\%), source: KLEMS.}
\end{figure} 
\begin{figure}[!h]
    \centering
    \includegraphics{img/plotted_perCountry/I-VA-Product-Manufacturing-USA.pdf}
    \caption{\label{fig:I-VA-Product-Manufacturing-USA}Evolution of the real estate sector's composition of capital formation, USA. Investment as share of value added (\%), source: KLEMS.}
\end{figure} 
\begin{figure}[!h]
    \centering
    \includegraphics{img/plotted_perCountry/I-VA-Product-Elec Gas Wtr-USA.pdf}
    \caption{\label{fig:I-VA-Product-Elec Gas Wtr-USA}Evolution of the electricity \& gas sector's composition of capital formation, USA. Investment as share of value added (\%), source: KLEMS.}
\end{figure} 
\begin{figure}[!h]
    \centering
    \includegraphics{img/plotted_perCountry/I-VA-Product-Construction-USA.pdf}
    \caption{\label{fig:I-VA-Product-Construction-USA}Evolution of the construction sector's composition of capital formation, USA. Investment as share of value added (\%), source: KLEMS.}
\end{figure} 

\begin{figure}[!h]
    \centering
    \includegraphics{img/plotted_perCountry/I-VA-Product-Sale-USA.pdf}
    \caption{\label{fig:I-VA-Product-Sale-USA}Evolution of the retail sector's composition of capital formation, USA. Investment as share of value added (\%), source: KLEMS.}
\end{figure} 

\begin{figure}[!h]
    \centering
    \includegraphics{img/plotted_perCountry/I-VA-Product-Transport Communication-USA.pdf}
    \caption{\label{fig:I-VA-Product-Transport Communication-USA}Evolution of the transport \& communication sector's composition of capital formation, USA. Investment as share of value added (\%), source: KLEMS.}
\end{figure} 

\begin{figure}[!h]
    \centering
    \includegraphics{img/plotted_perCountry/I-VA-Product-Finance-USA.pdf}
    \caption{\label{fig:I-VA-Product-Finance-USA}Evolution of the finance sector's composition of capital formation, USA. Investment as share of value added (\%), source: KLEMS.}
\end{figure} 

\begin{figure}[!h]
    \centering
    \includegraphics{img/plotted_perCountry/I-VA-Product-Real estate-USA.pdf}
    \caption{\label{fig:I-VA-Product-Real estate-USA}Evolution of the real estate sector's composition of capital formation, USA. Investment as share of value added (\%), source: KLEMS.}
\end{figure} 

\begin{figure}[!h]
    \centering
    \includegraphics{img/plotted_perCountry/I-VA-Product-Real Estate Business-USA.pdf}
    \caption{\label{fig:I-VA-Product-Real Estate Business-USA}Evolution of the real estate business sector's composition of capital formation, USA. Investment as share of value added (\%), source: KLEMS.}
\end{figure} 

\begin{figure}[!h]
    \centering
    \includegraphics{img/plotted_perCountry/I-VA-Product-Community-USA.pdf}
    \caption{\label{fig:I-VA-Product-Community-USA}Evolution of the community sector's composition of capital formation, USA. Investment as share of value added (\%), source: KLEMS.}
\end{figure} 

\section{Gross fixed capital formation}

Gross fixed capital formation is about 22\% of GDP\footnote{World Bank Data: \url{http://data.worldbank.org/indicator/NE.GDI.FTOT.ZS?end=2014&start=1960&view=chart}.}.
\Cref{fig:shareGfcfGdp} shows the values for selected countries and group of countries.
Between 1960 and 2014, GFCF was between 21.5\% and 25.5\%.
OECD at 20.
GFCF in France is at 22 (19.5-26.5\% in 1970-2014).
GFCF in the USA is at 19.5 (18-24.5\% in 1960-2014).
GFCF in China is at 45\% (15-45\% in 1960-2014, lowest since 2000 is 35 in 2000).
GFCF in India is at 29 (13-34\% in 1960-2014, lowest since 2000 is 24 in 2002).

\begin{figure}[!h]
    \centering
    \includegraphics{img/gfcfShareInGdp.pdf}
    \caption{\label{fig:shareGfcfGdp}Share of GFCF in GDP (\%), source: World Bank}
\end{figure}

\begin{figure}
    \centering
    \includegraphics{img/ruddock2009-constructionInGfcf.pdf}
    \caption{Share of construction in GFCF (\%)}
    From Lewis: Quantifying the GDP–construction relationship
    in Economics for the Modern Built Environment
    edited by Les Ruddock
\end{figure}

\begin{figure}
    \centering
    \includegraphics[trim={1cm 9cm 1cm 7cm},clip]{img/indiaShareConstructionGFCF.png}
    \caption{Share of construction in GFCF (\%)}
    From India's Economic Development Since 1947 By Uma Kapila
\end{figure}

\clearpage

\subsection{Data analysis}

\begin{figure}[!h]
    \centering
    \includegraphics[height=10cm]{img/usaTotAbsolute-1.pdf}
    \includegraphics[width=\textwidth]{img/usaTotAbsolute-2.pdf}
    \caption{usaTotAbsolute-1}
\end{figure}

\begin{figure}
    \centering
    \includegraphics[height=10cm]{img/usaTotRelative-1.pdf}
    \includegraphics[width=\textwidth]{img/usaTotRelative-2.pdf}
    \caption{usaTotRelative-1}
\end{figure}

\begin{figure}
    \centering
    \includegraphics[height=10cm]{img/usaYearsAbsolute-1.pdf}
    \includegraphics[width=\textwidth]{img/usaYearsAbsolute-2.pdf}
    \caption{usaYearsAbsolute-1}
\end{figure}

\begin{figure}
    \centering
    \includegraphics[height=10cm]{img/usaYearsRelative-1.pdf}
    \includegraphics[width=\textwidth]{img/usaYearsRelative-2.pdf}
    \caption{usaYearsRelative-1}
\end{figure}

\clearpage






\bibliographystyle{cbe}
\bibliography{klems.bib}

\listoftodos

\end{document}
