
\fbox{
    \parbox{\textwidth}{
        \textbf{CA vs. PCA:}\newline
            \begin{itemize}
                \item[ANSWER Frank]
                    CA is a form of PCA where distance is not euclidian norm but $\chi_2$.
                    What that means is that data is weighted by row totals and column totals (everything is to be understood as barycenters).
        \item[ANSWER 1]
            PCA works on the values where as CA works on the relative values.
            Both are fine for relative abundance data of the sort you mention (with one major caveat, see later).
            With \% data you already have a relative measure, but there will still be differences.
            Ask yourself do you want to emphasise the pattern in the abundant species/taxa (i.e. the ones with large \%cover), or do you want to focus on the patterns of relative composition?
            If the former, use PCA.
            If the latter use CA.
            What I mean by the two questions is would you want (50, 20, 10) and (5,  2,  1) to be considered different or the the same?
            PCA would consider these very different because of the Euclidean distance used, but CA would consider these two samples as being very similar because the have the same       relative profile.

            The big caveat here is the closed compositional nature of the data.
            If you have a few groups (Sand, Silt, Clay, for example) that sum to 1 (100\%) then neither approach is correct and you could move to a more appropriate analysis via Aitchison's Log-ratio   PCA which was designed for closed compositional data.
            (IIRC to do this you need to centre by rows and columns, and log transform the data.) There are other approaches too.
            If you use R, then one book that would be useful is
            Analyzing Compositional Data with R.
            \end{itemize}
    }
}

